\documentclass[10pt, aspectratio=169]{beamer}

\usetheme{metropolis}
\definecolor{alertc}{HTML}{008cd9}
\definecolor{myDarkTeal}{HTML}{1c2d30}
\setbeamercolor{alerted text}{fg=alertc}
\setbeamercolor{normal text}{fg=myDarkTeal,bg=white}
\metroset{block=fill}
\usepackage{appendixnumberbeamer}

\usepackage{booktabs}
\usepackage[scale=2]{ccicons}
\usepackage{array}
\newcolumntype{L}[1]{>{\raggedright\arraybackslash}m{#1}}
\newcolumntype{C}[1]{>{\centering\arraybackslash}m{#1}}
\newcolumntype{R}[1]{>{\raggedleft\arraybackslash}m{#1}}

\usepackage{siunitx}

\usepackage{xspace}
\newcommand{\themename}{\textbf{\textsc{metropolis}}\xspace}

\usepackage{amsmath}
\def\RR{\mathbb{R}}
\def\Pr{\mathrm{Pr}}
\DeclareMathOperator*{\argmax}{arg\,max}
\DeclareMathOperator*{\argmin}{arg\,min}
\usepackage{algorithm}
\usepackage{algpseudocode}

\usepackage{natbib}
\bibliographystyle{abbrvnat}
\setcitestyle{numbers,square}
\usepackage{graphicx}
\usepackage{xcolor}
\usepackage{hyperref}

\definecolor{mDarkTeal}{HTML}{23373b}
\definecolor{mLightBrown}{HTML}{EB811B}
\definecolor{mDarkBrown}{HTML}{B85002}
\definecolor{mLightGreen}{HTML}{14B03D}
\setbeamertemplate{itemize/enumerate subbody begin}{\normalsize}

\title{Computational detection of a high-risk DLBCL group}
\subtitle{A summary of my master thesis}
\date{October 15th, 2024}
\author{Lukas Geßl}
\institute{Chair of Statistical Bioinformatics, University of Regensburg}

\begin{document}

\maketitle

\section{The goals of my thesis}

\begin{frame}{}
  Develop a binary classifier that predicts the risk of DLBCL patients: high (PFS < 2 years) or low 
  (PFS $\geq$ 2 years). 

  In more detail:
  \begin{itemize}
    \item \textbf{Performance}: rate of positive predictions $\geq \num{0.10}$ and precision significantly 
      (\num{95}\%-CI) above the precision of the IPI on an independent, prospective data set. Make 
      the model transferable from one data set to another.
    \item \textbf{Heuristics and recommendations for the future}: Which model classes for which 
      hyperparameters are worth training and validating? Which models can we validate reliably and 
      perform strongly on new data? What are the requirements for the training and test data?
  \end{itemize}
\end{frame}

\section{A quick dive into the methods}

\begin{frame}{The big picture}
  I did the well-known train-validate-test approach.
  \begin{itemize}
    \item \textbf{Training and validation} on the training data. Fit several candidate models, 
      validate them (cross validation, out-of-bag predictions) and choose a single model, the best 
      one according to validation.
    \item \textbf{Testing} on the test data. Evaluate the performance of the chosen model.
  \end{itemize}
\end{frame}

\begin{frame}{The smaller picture: candidate models}
  \begin{itemize}
    \item All fit models are Gauss, logistic, Cox models or random forests or compositions thereof 
      (late integration of IPI, already-existent and own signatures).
    \item For training, I tuned the time cutoff that separates high- and low-risk patients.
    \item A-priori feature selection: Include gene-expression levels? Format of certain features?
    \item Loss-function-based feature selection: elastic-net, usually LASSO regularization.
    \item Enforce zero-sum constraint for gene-expression levels to gain cross-platform 
      transferability \cite{zerosum16,transplatform17}.
    \item Add combinations of at most $n_\text{combi}$ categorical features to the predictor if 
      sufficiently prevalent.
  \end{itemize}
\end{frame}


\section{The results}

\begin{frame}{Meet the data}
  \begin{table}[ht]
\small
\centering
\begin{tabular}{L{.18\textwidth}R{.2\textwidth}R{.28\textwidth}R{.18\textwidth}}
  \hline
  & \textbf{Schmitz \cite{schmitz18}} & \textbf{Reddy \cite{reddy17}} & \textbf{Lamis test \cite{staiger20}} \\
  \hline
  \textbf{\# samples} & 229 & 604 & 466 \\
  \textbf{\# genes} & 25066 & 13302 & 145 \\
  \textbf{technology} & RNA-seq & RNA-seq & NanoString \\
  \textbf{high risk [\%]} & 36.6 & 31.5\footnotemark & 24.3 \\
  \textbf{IPI-45 prev. [\%]} & 12.9 & 21.6 & 17.0 \\
  \textbf{IPI-45 prec. [\%]} & 65.2 & 54.1 & 38.2 \\
  \textbf{other features\footnotemark} & signature ``genetic subtype'' & 
    genetic events: high expression, translocation, mutation & LAMIS signature \\
  \hline
\end{tabular}
\end{table}

\end{frame}

\begin{frame}{}
  Use one of the three cohorts as training data, the respective remaining two cohorts as test data.

  \begin{table}[ht]
    \centering
    \small
    \begin{tabular}{lrrr}
        \hline
        & \textbf{Schmitz} & \textbf{Reddy} & \textbf{Staiger} \\
        \hline
        \textbf{Schmitz} & \num{12.9}, \num{65.2} & \num{17.7}, \num{59.6} & \num{17.1}, \num{50.7} \\
        \textbf{Reddy} & \num{17.8}, \num{71.1} & \num{21.6}, \num{54.1} & \num{18.0}, \num{53.2} \\
        \textbf{Staiger} & \num{17.4}, \num{75.7} & \num{22.5}, \num{50.4} & \num{17.0}, \num{38.2} \\
        \hline
    \end{tabular}
    \caption{Prevalence and precision.
        Diagonal entries $(i, i)$ hold prevalence, precision of the $\text{tIPI}$ on 
        cohort $i$. Off-diagonal entries $(i, j)$ hold prevalence, precision on cohort $j$ of 
        the best model trained and validated on cohort $i$.}\label{tab:inter-prev-prec}
\end{table}
\end{frame}

\begin{frame}{Dito for lower limit of 95\%-CI of precision}
  \begin{table}[ht]
    \centering
    \begin{tabular}{lrrr}
        \hline
        & \textbf{Schmitz} & \textbf{Reddy} & \textbf{Staiger} \\
        \hline
        \textbf{Schmitz} & \num{65.2} & \num{48.6} & \num{38.7} \\
        \textbf{Reddy} & \num{54.1} & \num{54.1} & \num{41.5} \\
        \textbf{Staiger} & \num{58.5} & \num{40.9} & \num{38.2} \\
        \hline
    \end{tabular}
    \caption{Lower limit of the \num{95}\%-confidence interval of the precision.
        Diagonal entries $(i, i)$ hold the precision of the $\text{tIPI}$ on cohort $i$. 
        Off-diagonal entries hold the lower limit of the \num{95}\%-confidence interval of the 
        precision on cohort $j$ of the best model trained and validated on cohort $i$.}
        \label{tab:inter-prec-ci}
\end{table}
  
  In the following, we focus on the best model trained and validated on the Reddy cohort, denoted 
  $m_\text{Reddy}^*$.
\end{frame}

\begin{frame}{What does $m_\text{Reddy}^*$ look like?}
  A logistic model, trained with LASSO regularization, the time cutoff separating into low- and 
  high-risk patients is \num{2.3} (versus \num{2.5}).

  Features:
  \begin{itemize}
    \item No gene-expression levels a-priori directly included.
    \item Single continuous feature included a priori: LAMIS score.
    \item Categorical features: gender, LDH ratio > 1, number of extranodal sites > 1, Ann Arbor 
      stage > 1, age > 60, ECOG > 1, IPI group, and combinations of two of them.
    \item Sparse model: \num{13} features have non-zero coefficients.
  \end{itemize}
\end{frame}

\begin{frame}{How reliable is the validation for the models trained on the Reddy data?}
  \begin{figure}[ht]
    \centering
    \includegraphics[width=0.9\textwidth]{figs/inter_val_test_reddy.pdf}
    \caption{Validation versus test error. The dashed gray line is the 
        identity line. The training survival cutoff $T$, 
        $n_\text{combi}$, elastic-net regularization strength $\lambda$ (for GLMs), 
        feature subsample size $m$ and minimal node size $n_\text{min}$ (for RFs) have already been 
        optimized with respect to the validation error.}
  \end{figure}
\end{frame}

\begin{frame}{On thresholding the continuous output of $m_\text{Reddy}^*$}
  \begin{figure}[ht]
    \centering
    \includegraphics[width=0.9\textwidth]{figs/inter_output_prec_reddy.pdf}
    \caption{Prevalence versus precision for all possible ways to threshold $m^*_\text{Reddy}$ for 
      a prevalence in $[\num{0.10}, \num{0.50}]$.  The dashed gray line marks the precision of the 
      $\text{tIPI}$ on the respective cohort.}
  \end{figure}
\end{frame}

\begin{frame}{Dito for the lower limit of the 95\%-CI of the precision}
  \begin{figure}[ht]
    \centering
    \includegraphics[width=0.9\textwidth]{figs/inter_output_prec_ci_reddy.pdf}
    \caption{Prevalence versus lower limit of the \num{95}\%-CI of the precision for all possible 
      ways to threshold $m^*_\text{Reddy}$ for a prevalence in $[\num{0.10}, \num{0.50}]$. The 
      dashed gray line marks the precision of the $\text{tIPI}$ on the respective cohort.}
  \end{figure}
\end{frame}

\begin{frame}{}
  Find the details on \url{https://github.com/lgessl/master-thesis}.
\end{frame}

\begin{frame}[standout]
      \centering
      Thank you! Questions?
\end{frame}

\appendix

\begin{frame}[allowframebreaks]{References}
  \small
  \bibliography{../lit.bib}
\end{frame}

\end{document}
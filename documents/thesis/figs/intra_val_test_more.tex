\begin{figure}
    \centering
    \includegraphics[width=1.0\textwidth]{../../results/intra_val_test_more.pdf}
    \caption{Validation versus test error in intra-trial experiments for models predicting from the 
        full range of a priori selected features for all three data sets in the rows. The dashed 
        gray line is the identity line. The notation 
        $c_1$-$c_2$ encodes a model of class $c_1$ as the early model $f_1$ nested into a model of 
        class $c_2$ as the late model $f$ according to Alg. \ref{alg:nested-pcv}.
        ``GE'' is an acronym for gene expression and refers to the presence of all available 
        gene-expression levels in the predictor, while ``no GE'' refers to the absence of all of 
        them. Both columns show 
        the same points, but highlight different groupings. The elastic-net 
        regularization strength $\lambda$, training survival cutoff $T$, $n_\text{combi}$ and the 
        random-forest hyperparameters $B$, $m$ and $n_\text{min}$ have already been tuned. The 
        validation error for the $\text{tIPI}$ is in fact also a test error on independent data, 
        namely our training cohort.}
    \label{fig:intra-val-test-more}
\end{figure}
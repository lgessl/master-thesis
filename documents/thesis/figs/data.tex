\begin{table}
    \centering
    \begin{tabular}{lrrr}
        \hline
        & \textbf{Schmitz} & \textbf{Reddy} & \textbf{Staiger} \\
        \hline
        \textbf{prospective trial} & no & no & yes \\
        \textbf{response} & PFS & OS & PFS \\
        \textbf{\# samples} & \num{229} & \num{604} & \num{466} \\
        \textbf{\# genes with expression level} & \num{25066} & \num{13302} & \num{145} \\
        \textbf{technology} & RNA-seq & RNA-seq & NanoString \\
        \textbf{high risk [\%]} & \num{36.6} & \num{31.5} & \num{24.3} \\
        \textbf{$\text{prev}(\text{tIPI})$ [\%]} & \num{12.9} & \num{21.6} & \num{17.0} \\
        \textbf{$\text{prec}(\text{tIPI})$ [\%]} & \num{65.2} & \num{54.1} & \num{38.2} \\
        \hline
    \end{tabular}
    \caption{Overview on used data sets. All datasets include the five IPI features in their 
        thresholded format, the IPI score and group, gender, cell of origin thresholded into 
        ABC-like, GCB-like, unclassified, and, added 
        by us, the LAMIS score and group. For the Schmitz and Staiger data, we use the high-risk 
        definition of 
        MMML-Predict: PFS below two years. Because the Reddy data reports only overall survival 
        (OS), no PFS, we define high risk as overall survival below \num{2.5} years there, leading 
        to a comparable high-risk proportion. ``Technology'' refers to the technology used to measure 
        gene-expression levels.}
        \label{table:data}
\end{table}
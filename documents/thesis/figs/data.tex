\begin{table}
    \centering
    \begin{tabular}{lrrr}
        \hline
        & \textbf{Schmitz \cite{schmitz18}} & \textbf{Reddy \cite{reddy17}} & \textbf{Lamis test \cite{staiger20}} \\
        \hline
        \textbf{prospective trial} & no & no & yes \\
        \textbf{outcome} & PFS & OS & PFS \\
        \textbf{\# samples} & \num{229} & \num{604} & \num{466} \\
        \textbf{\# genes} & \num{25066} & \num{13302} & \num{145} \\
        \textbf{technology} & RNA-seq & RNA-seq & NanoString \\
        \textbf{high risk [\%]} & \num{36.6} & \num{31.5}\footnotemark & \num{24.3} \\
        \textbf{IPI-45 prev. [\%]} & \num{12.9} & \num{21.6} & \num{17.0} \\
        \textbf{IPI-45 prec. [\%]} & \num{65.2} & \num{54.1} & \num{38.2} \\
        \hline
    \end{tabular}
    \caption{Overview on used data sets. All datasets include the IPI features in thresholded form, 
        gender, cell of origin, and, added by us, the Lamis signature. The Schmitz and Lamis test 
        data use the high-risk defintion of MMML-Predict: progression-free survival below two years; 
        because the Reddy data reports only overall, no progression-free survival, we define high risk 
        as overall survival below \num{2.5} years, leading to a comparable high-risk proportion.}
        \label{table:data}
\end{table}
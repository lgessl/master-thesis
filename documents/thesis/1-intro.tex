\chapter{Introduction} \label{chap:intro}

\section{Diffuse large B-cell lymphoma}

Diffuse large B-cell lymphoma (DLBCL) is the most common type of B-cell lymphoma in adults, 
representing almost 30\% of all diagnoses of non-Hodgkin's lymphoma. 
This aggressive and heterogeneous group of 
lymphoid neoplasms exhibits
diverse phenotypic, genetic and clinical characteristics. The clinical presentation of 
DLBCL varies significantly, with differences in tumor load and patients' performance status, leading 
to varied outcomes \cite{dlbcl-review21}.

Despite being an aggressive and fatal disease, DLBCL is highly curable with the 
application of intensive immunochemotherapy even in the elderly population. The standard treatment 
for DLBCL is immunochemotherapy in the R-CHOP (rituximab, cyclophosphamide, doxorubicin, 
vincristine, and prednisone) regimen. R-CHOP-like therapies
have significantly improved survival rates, curing approximately two-thirds of the patients. 
However, the remaining one-third of the patients, especially 
those with relapsed or refractory disease, continue to face poor outcomes and in most cases finally 
succumb to their disease \citep{glass17}. This underscores the clinical 
need for an accurate, robust, affordable and easy-to-use tool that can identify patients at high risk 
for treatment failure early in their treatment course. Clinicians can guide those patients labeled 
as high-risk by the tool to alternative treatments like immunotherapy with CAR-T cells 
\cite{wang20}.

To this end, the International Prognostic Index for non-Hodgkin's lymphoma (IPI) was established in 
the 1990s. It incorporates five binary clinical features one can measure cheaply and 
reliably without batch effects: Is the patient older than \num{60} years? Is the cancer advanced 
(Ann Arbor Stage III or IV)? Does the patient's blood have a higher-than-normal lactate dehydrogenase (LDH) 
level? Is the patient already 
bedridden (performance status > 1)? Is the patient's cancer in more than one extranodal site? The 
\textit{IPI score} is then the number of positive answers to these questions, an integer between 0 
and 5. The IPI is the result of a rigorous statistical analysis 
of a large dataset of \num{1872} patients: Out of twelve candidate features, the IPI inventors
first selected those features that were independently and significantly associated with survival 
(namely the five above-mentioned features), and fit a Cox proportional-hazards model to them.
Since the relative risks for all five features turned out to be similar, they simplified the model 
by just counting the number of present risk factors~\cite{ipi93}. 

The IPI is a simple and robust clinical tool used globally to predict risk and guide 
treatment decisions in DLBCL patients. It has been the cornerstone of risk assessment for the last 
three decades, no alternative has gained widespread acceptance outside of clinical 
trials~\citep{ipi-stay-strong}. 
At the same time, pharma-sponsored randomized trials in the whole DLBCL population have failed to
improve R-CHOP \cite{susanibar21}. The IPI fails to identify a high-risk DLBCL subpopulation that is 
large enough to trigger research and to enable clinical trials for new treatments that outperform 
R-CHOP on this subpopulation: In the prospective trial comprising \num{466} patients used as test 
set in \citep{staiger20}, only \num{3.4}\% of patients have the maximum IPI score of 5. All other 
cohorts defined via $\text{IPI} \geq i, i = 0, 1, 2, 3, 4$, have a low high-risk content, i.e.,  
the proportion of patients with 
\textit{progression-free survival} (PFS) less than two years in all these cohorts is below 
\num{50}\%. PFS is the time from the start of the treatment to tumor progression or death from any 
cause \cite{saad09}.

For this thesis, we define high-risk DLBCL patients as those patients with PFS below two years. 
Two years is a time 
frame accepted by both patients and clinicians, which makes roughly a quarter of DLBCL patients 
high-risk \cite{staiger20}. It is also the threshold used in the MMML-Predict project we will 
introduce next.

\section{The MMML-Predict project} \label{sec:intro-mmml}

Renowned lymphoma experts from across academic Germany -- clinical trialists, biostatisticians, 
bioinformaticians, lymphoma pathologists and translational lymphoma biologists -- have formed the 
consortium MMML-Predict to develop and roll out a robust, simple-to-use and cost-effective 
prognostic tool for DLBCL 
which yields a clinically more relevant high-risk group \cite{mmml-chapuy,mmml-idw}. This tool, the 
MMML-Predictor, aims to allow
patients and clinicians early in the treatment cycle to make an informed decision if they want to 
continue with the standard R-CHOP-like treatment or switch to a novel, more experimental treatment. 

MMML-Predict will enroll \num{200} patients in a prospective trial at first 
diagnosis into a training cohort and collect various clinical and molecular risk features.
They include clinical scores (like the IPI), gene-expression-derived features (like cell of origin) 
and genetic aberrations (like the translocation of MYC). 
In a novel approach, MMML-Predict will measure dynamic response determinants during treatment. It is 
unknown if these features capture similar or different 
risks. Combining them may finally bring to the patients' bedside the significant progress in the 
understanding of DLBCL biology that researchers have made over the last decades. The training 
cohort will be the data foundation to build the MMML-Predictor.

There are plenty of biological terms in the last paragraph, and we want to hold on for a moment to 
introduce those that will become relevant over the course of this thesis.
Many of the above-mentioned features rely on measuring \textit{gene-expression levels} for a set 
of genes. 
For a single gene, its expression level reflects the number of RNA transcripts derived from 
this gene in a tissue, in our case a lymphoma biopsy. Measuring gene-expression levels for all 
or at least a high number of the genes that an organism expresses, yields a gene-expression 
profile. \textit{Cell-of-origin signatures} indicate if the gene-expression profile 
of a tumor is more similar to that of a germinal-center B cell (GCB) or an activated B cell (ABC), 
which refer to two states of a B cell over its life cycle; GCB-like DLBCLs have 
significantly better survival than ABC-like DLBCLs \cite{abc-gcb00}.
Biologists call predictive models that 
predict from genetic features, especially gene-expression levels, \textit{signatures}. The status 
of the genes MYC, BCL2, BCL6 and TP53 is known to predict survival in DLBCL and has been subject 
to extensive research in the last years
\cite{aukema11, horn13, hummel06, johnson09, klapper08, kretzmer15, rosenwald19, zenz17}.

The group around Markus Loeffler in Leipzig plans to evaluate the readily trained MMML-Predictor on 
a test cohort of another \num{100} patients enrolled for this project. The new classifier has to 
label at least \num{10}\% of 
patients in the test cohort as high-risk. Stated differently, we require a rate of positive 
predictions or -- synonymously -- a \textit{prevalence} of at least \num{10}\%. At the same time, the 
\text{precision} -- i.e.\ the proportion of \textit{truly} high-risk patients among the patients 
\textit{labeled} as high-risk -- must be significantly above that of the group defined by
$\text{IPI} \geq 4$. 

In more detail, being high-risk or not is a Bernoulli random variable. 
The precision is the success probability of this random variable under the condition that the 
MMML-Predictor predicts a high risk. The test cohort comprises independent, identically distributed 
(i.i.d) samples. They stay independent under the condition of being labeled high-risk by the 
MMML-Predictor. The precision on the test cohort is therefore tailored for the Clopper-Pearson 
method, which calculates exact $\gamma$-confidence intervals (CIs) for the success probability 
of i.i.d.\ Bernoulli random variables for an arbitrary confidence level $\gamma \in (0, 1)$
\cite{clopper34}.
By significantly outperforming the precision of the IPI, we mean that the \num{95}\%-CI
of the precision of the MMML-Predictor must not include the 
precision of the group determined by $\text{IPI} \geq 4$.

\section{The role of this thesis inside MMML-Predict}

Inside MMML-Predict, Rainer Spang's group in Regensburg will develop the MMML-Predictor. 
This is 
a supervised-learning task with a classification problem with binary response: PFS of less 
than two years -- or high-risk DLBCL -- is the positive group, PFS of more than two 
years -- or low-risk DLBCL -- is the negative group. Since, as of July 2024, the MMML-Predict 
training cohort has not yet arrived in Regensburg, this thesis will imitate the train-test scenario 
of MMML-Predict on already-available DLBCL data sets. 

This thesis has two main goals.
\begin{itemize}
\item First, we want to show that, with data including traditional clinical and modern 
molecular features, we can deliver a model that identifies a DLBCL subpopulation with poor survival 
that comprises at least \num{10}\% of patients and holds significantly more high-risk patients than 
that identified by the IPI on independent test data. 
With an eye to rolling out the MMML-Predictor in clinical practice, 
we want to demonstrate that we can design this model in such a way that one can transfer it from 
one measurement protocol to another without seriously compromising its performance.
\item Second, we want to develop heuristics and recommendations to answer the question which 
candidate models under which circumstances are worth training and validating. We want to find out 
which models we can reliably validate and which perform well on independent test data and what are 
the requirements for the data to do so. These findings are meant for future MMML-Predict researchers 
to avoid overfitting in the internal validation and submitting a model that convinces 
in validation, but disappoints on the test cohort in Leipzig.
\end{itemize}
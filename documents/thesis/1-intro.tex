\chapter{Introduction} \label{chap:intro}

\section{Diffuse large B-cell lymphoma: treatment and state-of-the-art risk prediction}

Diffuse large B-cell lymphoma (DLBCL) is the most common type of B-cell lymphoma in adults, 
representing almost 30\% of all diagnoses of non-Hodgkin's lymphoma. 
This aggressive and heterogeneous group of 
lymphoid neoplasms exhibits
diverse phenotypic, genetic, and clinical characteristics. The clinical presentation of 
DLBCL varies significantly, with differences in tumor load and patients' performance status, leading 
to varied outcomes \cite{dlbcl-review21}.

Despite being an aggressive and fatal disease, DLBCL is highly curable with the 
application of intensive immunochemotherapy even in the elderly population. The standard treatment 
for DLBCL has long been immunochemotherapy in the R-CHOP (rituximab, cyclophosphamide, doxorubicin, 
vincristine, and prednisone) regimen. R-CHOP-like therapies
have significantly improved survival rates, curing approximately two-thirds of the patients. 
However, the remaining one-third of patients, especially 
those with relapsed or refractory disease, continue to face poor outcomes and in most cases finally 
succumb to their disease \citep{glass17}. This underscores the clinical 
need for an accurate, robust, affordable, and easy-to-use tool that can identify patients at high risk 
for treatment failure early in their treatment course, ideally within the first three cycles of 
induction chemotherapy. Clinicians can guide those patients labeled as high-risk by the tool to 
alternative treatments like salvage chemotherapy and CAR-T cells. 

To this end, the International Prognostic Index for non-Hodgkin's lymphoma (IPI) was established in 
the 1990s. It incorporates five binary clinical features one can measure cheaply and 
reliably without batch effects: Is the patient older than \num{60} years? Is the cancer advanced 
(Ann Arbor Stage III or IV)? Does the patient have a higher-than-normal lactate dehydrogenase (LDH) 
level? Is the patient already 
bedridden (performance status > 1)? Is the patient's cancer in more than one extranodal site? The 
\textit{IPI score} is then the number of positive answers to these questions, an integer between 0 
and 5. The IPI is the result of a rigorous statistical analysis 
of a large dataset of \num{1872} patients: out of twelve candidate features, the IPI inventors
first selected those features that were independently and significantly associated with survival 
(namely the five above-mentioned features), and fit a Cox proportional-hazards model to them;
since the relative risks for all five features turned out to be similar, they simplified the model 
by just counting the number of present risk factors~\cite{ipi93}. 

The IPI is a simple yet robust clinical tool used globally to predict risk and guide 
treatment decisions in DLBCL patients. It has been the cornerstone of risk assessment for the last 
three decades, no alternative has gained widespread acceptance outside of clinical 
trials~\citep{ipi-stay-strong}. 
Despite its effectiveness in large cohorts, the IPI and other individual 
biomarkers do not reliably predict the clinical course for each patient, particularly failing to 
identify those at high risk for early treatment failure who may benefit from alternative therapeutic 
approaches. E.g., in the prospective trial comprising \num{466} patients used as test set in 
\citep{staiger20}, only \num{3.4}\% of patients have the maximum IPI score of 5 -- this proportion 
is too low 
to gain special attention in clinical practice and to incentivize the pharmaceutical industry to 
develop new treatments. All other cohorts defined via $\text{IPI} \geq i, i = 0, 1, 2, 3, 4$, lack 
precision: the proportion of patients with \textit{progression-free survival} (PFS) less than two 
years is below \num{50}\% -- too low to persuade a patient to undergo an experimental treatment 
instead of a standard, R-CHOP-like therapy. PFS is the time from the start of the treatment to tumor 
progression or death from any cause \cite{saad09}.

For this thesis, we define high-risk DLBCL patients as those patients with PFS below two years. 
Two years is a time 
frame accepted by both patients and clinicians, which makes roughly a quarter of DLBCL patients 
high-risk \cite{staiger20}. It is also the threshold used in the MMML-Predict project we will 
introduce next.

\section{The MMML-Predict project} \label{sec:intro-mmml}

Renowned lymphoma experts from across academic Germany -- clinical trialists, biostatisticians, 
bioinformaticians, lymphoma pathologists and translational lymphoma biologists -- have formed the 
consortium MMML-Predict to develop and roll out a robust, simple-to-use and cost-effective 
prognostic tool for DLBCL 
which yields a clinically more relevant high-risk group \cite{mmml-chapuy,mmml-idw}. This tool, the 
MMML-Predictor, aims to 
patients and clinicians early in the treatment cycle to make an informed decision if they want to 
continue with the standard R-CHOP-like treatment or switch to a novel, more experimental treatment. 

In a discovery phase, MMML-Predict will enroll \num{200} patients in a prospective trial at first 
diagnosis and collect various clinical and molecular risk features.
They include clinical scores (like the IPI), gene-expression-derived features (like cell of origin) 
and genetic aberrations (like the translocation MYC). 
In a novel approach, MMML-Predict will measure dynamic response determinants during treatment. It is 
unknown if these features capture similar or different 
risks. Combining them may finally bring to the patients' bedside the significant progress in the 
understanding of the DLBCL biology researchers have made over the last decades.

There are plenty of biological terms in the last paragraph, and we want hold on for a moment to 
introduce those that will become relevant over the course of this thesis.
Many of the above-mentioned features rely on measuring \textit{gene-expression levels} for a set 
of genes. 
For a single gene, its expression level reflects the number of RNA transcripts derived from 
this gene in a tissue, in our case a lymphoma biopsy. Measuring gene-expression levels for all 
or at least a high number of the genes that an organism expresses, yields a gene-expression 
profile. \textit{Cell-of-origin signatures} indicate if the gene-expression profile 
of a tumor is more similar to that of a germinal-center B-cell (GCB) or an activated B-cell (ABC), 
which refer to two states of a B cell over its life cycle; GCB-like DLBCLs have 
significantly better survival than ABC-like DLBCLs \cite{abc-gcb00}.
Biologists call predictive models that 
predict from genetic features, especially gene-expression levels, \textit{signatures}. The status 
of the genes MYC, BCL2, BCL6 and TP53 are known to predict survival in DLBCL and have been subject 
to extensive research in the last years
\cite{hummel06, zenz17, aukema11, horn13, johnson09, klapper08, kretzmer15, rosenwald19}.

The group around Markus Loeffler in Leipzig plans to evaluate the readily trained MMML-Predictor on 
a test cohort of another \num{100} patients enrolled for this project for whom only those features 
used in the MMML-Predictor will be measured. The new classifier has to label at least \num{10}\% of 
patients in the test cohort as high-risk; stated differently, we require a rate of positive 
predictions or -- synonymously -- a precision of at least \num{10}\%. At the same time, the 
precision -- i.e.\ the proportion of \textit{truly} high-risk patients among the patients 
\textit{labeled} as high-risk -- must be \textit{significantly} above that of the group defined by
$\text{IPI} \geq 4$. 

In more detail, being high-risk or not is a Bernoulli random variable. 
The precision is the success probability of this random variable under the condition that the 
MMML-Predictor predicts a high risk. The test cohort comprises independent, identically distributed 
(i.i.d) samples. They stay independent under the condition of being labeled high-risk by the 
MMML-Predictor. The precision on the test cohort is therefore tailored for the Clopper-Pearson 
method, which calculates exact $\gamma$-confidence intervals (CIs) for the success probability 
of i.i.d.\ Bernoulli random variables for an arbitrary confidence level $\gamma \in (0, 1)$
\cite{clopper34}.
By significantly outperforming the precision of the IPI, we mean that the \num{95}\%-CI
of the precision of the new classifier according to the Clopper-Pearson method must not include the 
precision of the group determined by $\text{IPI} \geq 4$. According to data with \num{2721} 
patients pooled from prospective trials of the German High-Grade Non-Hodgkin Lymphoma Study Group 
(DSHNHL), this precision of the IPI is at around \num{35}\%. 

\section{The role of this thesis within MMML-Predict}

Within MMML-Predict, Rainer Spang's group in Regensburg will develop the MMML-Predictor. 
This is 
a supervised-learning task with a classification problem with binary response: PFS of less 
than two years -- or high-risk DLBCL -- is the positive group, PFS of more than two 
years -- or low-risk DLBCL -- is the negative group. Part of the task is figuring out which features
are part of the model input, the predictor.
While the number of samples is rather small, 
the number of available features is large thanks to high-throughput measurements like RNA-seq or 
NanoString. We have an enormous amount of freedom in how we design and train the classifier. 
Hence, we need to deal with the curse of high dimensions if we want to use the high-dimensional part 
of the data and, 
more importantly, we need to take care that in our internal validation, usually a cross-validation,
we do not overfit the data. While we should be able to tackle the first problem with regularization,
for the latter one we need a trustworthy internal validation strategy and, most importantly, we 
must not validate too many models in the first place.

Since, as of July 2024, the MMML-Predict training cohort has not yet arrived in Regensburg, this thesis
will imitate the train-test scenario of MMML-Predict on already-available DLBCL data sets. 

This thesis has two main goals.
\begin{itemize}
\item First, we want to show that, with data including traditional clinical and modern 
molecular features, we can indeed deliver 
the desired model, which yields a larger and more precise high-risk group of DLBCL patients than 
IPI and does so significantly. With an eye to rolling out the MMML-Predictor in clinical practice, 
we want to demonstrate that we can design this model in such a way that one can transfer it from 
one lab protocol to another without seriously comprised performance.
\item Second, we want to develop heuristics and recommendations to answer the question which candidate 
models are worth training and validating and -- more importantly -- which are not. For this, we 
need to find out which models we can reliably validate 
and which perform well on independent test data. These findings will guide us on how to keep 
the number of candidate models we fit to the MMML-Predict training data low. This helps
us avoid overfitting in the internal validation and submitting a model that convinces 
in the validation, but disappoints on the test cohort in Leipzig.
\end{itemize}
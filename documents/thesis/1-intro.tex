\chapter{Introduction}

\section{Diffuse large B-cell lymphoma: treatment and state-of-the-art risk prediction}

Diffuse large B-cell lymphoma (DLBCL) is the most common type of B-cell lymphoma in adults, 
accounting for approximately 30\% of all diagnoses. This aggressive and heterogeneous group of 
lymphoid neoplasms typically originates from malignant transformed germinal center (GC) B cells, 
exhibiting diverse phenotypic, genetic, and clinical characteristics. The clinical presentation of 
DLBCL varies significantly, with differences in tumor load and patient performance status, leading 
to varied outcomes.

Despite being an aggressive and, if left untreated, fatal disease, DLBCL is a highly curable disease with the 
application of intensive immunochemotherapy even in the elderly population. The standard treatment 
for DLBCL has long been immunochemotherapy with the R-CHOP regimen. This 
regimen has significantly improved survival rates, with approximately two-thirds of patients 
achieving a cure. However, the remaining one-third of patients, especially 
those with relapsed or refractory disease, continue to face poor outcomes \citep{glass17}. For patients who do not 
respond to initial treatment, options include salvage chemotherapy and newer therapies like CAR-T cells. 
Approximately one-third of patients with DLBCL succumb to their disease, particularly those with 
relapsed or refractory conditions, for whom cure rates remain low. This underscores the clinical 
need for an accurate, robust, affordable, and easy-to-use tool that can identify patients at high risk 
for treatment failure early in their treatment course, ideally within the first three cycles of 
induction chemotherapy. 

To this end, the International Prognostic Index for non-Hodgkin's lymphoma (IPI) was established in the 
1990s. It incorporates five clinical binary clinical features one can measure cheaply and reliably without batch 
effects: Is the patient older than 60 years? Is the cancer advanced (Ann Arbor Stage III or IV)? Does 
the patient have a higher-than-normal lactate dehydrogenase (LDH) level? Is the patient already 
bedridden (performance status > 1)? Is the patient's cancer in more than one extranodal site? The 
IPI is then the number of positive answers to these questions, an integer between 0 and 5. Although 
primitive and arbitrary at first glance, the IPI is the result of a rigorous statistical analysis 
of a large dataset of \num{1872} patients: out of twelve candidate features, the IPI inventors
first selected those features that were independently and significantly associated with survival 
(namely the five above mentioned features), and fit a Cox proportional hazards model to them;
since the relative risks for all five features turned out to be similar, they simplified the model 
by just counting the number of present risk factors~\cite{ipi93}. 

The IPI is a simple yet robust clinical tool used globally to predict risk and guide 
treatment decisions in DLBCL patients. It has been the cornerstone of risk assessment for the last 
three decades, no alternative has gained widespread acceptance outside of clinical 
trials~\citep{ipi-stay-strong}. 
Despite its effectiveness in large cohorts, the IPI and other individual 
biomarkers do not reliably predict the clinical course for each patient, particularly failing to 
identify those at high risk for early treatment failure who may benefit from alternative therapeutic 
approaches. E.g., in the prospective trial comprising \num{466} patients used as test set in 
\citep{staiger20}, only \num{3.4}\% of patients have the maximum IPI score of 5 -- too few
to gain special attention in clinical practice and to incentivize the pharmaceutical industry to 
develop new treatments. All other cohorts defined via $\text{IPI} \geq i, i = 0, 1, 2, 3, 4$, lack 
precision: the proportion of patients with progression-free survival less than two years is below 
\num{50}\% -- too few to persuade a patient to undergo an experimental treatment instead of the 
standard R-CHOP regimen.

For this thesis, we define high-risk DLBCL patients as those who face cancer progression 
within two years after the start of the treatment. Two years is a time frame accepted by both 
patients and clinicians, which makes roughly a fourth of DLBCL patients high-risk \cite{staiger20}.
It's also the threshold used in the MMML-Predict we will introduce next.

\section{The MMML-Predict project}

Renowned lymphoma experts from across academic Germany -- clinical trialists, biostatisticians, 
bioinformaticians, lymphoma pathologists and translational lymphoma biologists -- have formed the 
consortium MMML-Predict to develop and roll out a new, robust, simple-to-use, cost-effective and 
parsimonious prognostic tool for DLBCL 
which yields a clinically more relevant high-risk group. This tool, the MMML-Predictor, will allow 
patients and clinicians early in the treatment cycle to make an informed decision if they want to 
continue with the standard R-CHOP treatment or switch to novel, more experimental treatments. 

In a discovery phase, MMML-Predict will enroll 200 patients in a prospective trial at first diagnosis and 
collect all clinical and molecular risk features that alone predict an unfavorable outcome, 
including clinical scores (like the IPI), gene-expression based factors (like cell-of-origin signatures,
immune scores) and genetic determinants (like MYC, BCL2, TP53, germline and somatic signatures).
As a novel approach, they will measure dynamic response determinants during treatment (PET-CT 
and liquid biopsy-based MRD detection). It is unknown if these features capture similar or different 
risks; combining them may finally bring the significant progress in the understanding of the DLBCL 
biology we have made over the last decades to the patients' bedside.

The group around Markus Loeffler in Leipzig will evaluate the readily trained MMML-Predictor on a 
test cohort of another 100 patients enrolled for this project for whom only those features used 
in the MMML-Predictor will be measured. The new classifier has to achieve a rate of positive 
predictions -- or prevalence -- of at least \num{15}\% and a precision (for high-risk) significantly above that of 
$\text{IPI} \geq 4$ on the test cohort, meaning the \num{95}\% confidence interval of the new 
classifier's precision according to the Clopper-Pearson method must not include the precision of 
the group determined by $\text{IPI} \geq 4$. In other prospective trials, this precision of the IPI 
is at around \num{35}\%; taking this number, a prevalence of \num{15}\% and the test cohort size into 
account, a calculation of the MMML-Predict consortium suggests that a precision of at least \num{50}\%
of the MMML-Predictor suffices.

\section{The role of this thesis within MMML-Predict}

Inside MMML-Predict, Rainer Spang's group is responsible to develop the MMML-Predictor. This is 
a supervised-learning task with a binary classification problem: progression-free survival less than 
two years -- or high-risk DLBCL -- is the positive group, progression-free survival more than two 
years -- or low-risk DLBCL -- is the negative group. We have a small number of samples, a large 
number of features and an enormous amount of freedom in how we design and train the classifier. 
Hence, we need to deal with the curse of high dimensions if we want to use the high-dimensional part 
of the data directly (and not indirectly via late integration of already-existent signatures) and, 
more importantly, we need to take care that in our internal validation (usually a cross-validation)
we do not overfit the data. While we should be able to tackle the first problem with regularization,
for the latter one we need a trustworthy internal validation strategy and, most importantly, we 
must not validate too many models in the first place.

Since, as of July 2024, the MMML-Predict train cohort hasn't yet arrived in Regensburg, this thesis
will imitate the train-test scenario of MMML-Predict on already-available DLBCL data sets. This thesis 
has two main goals:
\begin{enumerate}
    \item We want to show that with data including traditional clinical and modern 
        molecular features, both possibly already condensed to signatures, we can indeed deliver 
        the desired model, which yields a larger and more precise high-risk group of DLBCL patients.
        With an eye to rolling out the MMML-Predictor in clinical practice, we want to demonstrate 
        that we can design this model in such a way that one can transfer it from one platform to
        another without losing its predictive power.
    \item We want to develop heuristics and recommendations to answer the question which candidate 
        models (including their hyperparameters) are worth training and validating and--more 
        importantly--which are not. For this, we need to infer which models we can reliably validate 
        and which perform well on independent test data. These findings will guide us on how to keep 
        the number of candidate models we fit to the MMML-Predict training data low, thereby helping 
        us to avoid overfitting in the internal validation and to submitting a model that convinces 
        in the validation, but disappoints on the test cohort in Leipzig.
\end{enumerate}
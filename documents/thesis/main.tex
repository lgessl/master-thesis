\documentclass[10.5pt, a4paper, oneside]{report}
\usepackage{layout}
\usepackage[margin=102pt, textwidth=390pt]{geometry}

\usepackage{polyglossia}
\setmainlanguage{english}
\setotherlanguage{german}
\usepackage{fontspec}
\setmainfont{TeX Gyre Pagella}
\setsansfont{Fira Sans}[Scale=MatchLowercase]
\setmonofont{Inconsolata}[Scale=MatchLowercase]
\usepackage{sectsty}
\allsectionsfont{\sffamily}

\usepackage{amsmath}
\usepackage{amsfonts}
\usepackage{mathtools}
\DeclareMathOperator*{\argmin}{arg\,min}
\DeclarePairedDelimiter\ceil{\lceil}{\rceil}
\DeclarePairedDelimiter\floor{\lfloor}{\rfloor}
\DeclarePairedDelimiter\nint{\lfloor}{\rceil}
\def\RR{\mathbb{R}}
\def\Pr{\mathrm{Pr}}
\def\NN{\mathbb{N}}
\def\im{\mathrm{im}}
\def\cond{\,|\,}
\def\sgn{\mathrm{sgn}}
\def\lamishigh{$\text{LAMIS}_{\text{high}}$}
\def\lamislow{$\text{LAMIS}_{\text{low}}$}
\def\tbeta{\tilde{\beta}}
\def\hbeta{\hat{\beta}}

\usepackage{algorithm}
\usepackage{algpseudocode}
\usepackage{caption}
\usepackage{subcaption}
\captionsetup[table]{position=bottom}
\captionsetup[subtable]{font=normalsize}

\usepackage{siunitx}

\usepackage{natbib}
\bibliographystyle{abbrvnat}
\setcitestyle{numbers,square}

\usepackage{graphicx}
\usepackage{xcolor}
\usepackage{hyperref}

\begin{document}
    \begin{titlepage}
        \centering
        \vspace*{1cm}
        
        \huge
        \textsf{
        \textbf{Computational detection of a high-risk DLBCL group}
        }
        
        \vspace{1.5cm}
        
        \large
        \textbf{Lukas Geßl}
        
        \vfill
        
        \large
        A thesis presented for the degree of\\
        Master of Science
        
        \vspace{0.8cm}
        
        \begin{tabular}{rl}
            First Supervisor: & Prof.\ Dr.\ Harald Garcke \\
            Second Supervisor: & Prof.\ Dr.\ Rainer Spang \\
        \end{tabular}
        
        \vspace{0.8cm}
        
        Department of Mathematics\\
        University of Regensburg\\
        \vspace{0.8cm}
        \today
        
    \end{titlepage}

    \begin{abstract}
        Chemotherapy with R-CHOP is the standard treatment for diffuse large B-cell lymphoma, 
        the most common type of non-Hodgkin lymphoma, achieving a cure for about two thirds of 
        patients. Survival for the remaining third with refractory or relapsed disease, however, 
        remains poor. Pharma-sponsored randomized trials in the whole DLBCL population to date have 
        failed 
        to improve R-CHOP. The International Prognostic Index (IPI), the only widely accepted 
        risk-assessment tool for DLBCL and an easy clinical test, fails to identify 
        a high-risk DLBCL subpopulation that is 
        large and precise enough to trigger research and enable clinical trials for new treatments 
        that outperform R-CHOP on this subpopulation. 

        This thesis aims to develop a computational method that identifies DLBCL patients with 
        progression-free survival (PFS) below two years with higher prevalence and significantly 
        higher precision than the IPI and to show this on independent data. It also deals with the 
        question under which circumstances we can do so reliably.

        After introducing into DLBCL, the IPI together with its shortcomings and MMML-Predict, the 
        project that this thesis is part of, in the first chapter, we describe the statistical 
        frame and models as well as our software in the second chapter. The third chapter applies 
        the method to three different data sets and one big data set comprised of these three 
        data sets and 
        shows that we can indeed deliver a model with the desired properties. Analysis after 
        freezing the models and unlocking the test data suggest that, for a reliable internal 
        validation and high test performance, 
        (a) data sets with a large number of samples, even if 
        they result from combining somewhat different, partly non-prospective data sets sets, 
        (b) relying on already-existing molecular signatures rather than fitting new ones and 
        (c) deploying simple, generalized linear models that can handle batch effects
        play a key role. The fourth chapter 
        discusses how we can develop even better models in the future, especially for MMML-Predict. 

        We conclude that with currently available data and statistical and computational methods, 
        we can identify a DLBCL subpopulation with poor survival that is larger and 
        significantly more precise than that identified by the IPI. With more data and new and 
        more accurately measured features in the future, we expect to be able to further improve the 
        performance of such models.
    \end{abstract}

    % \layout
    \tableofcontents
    \chapter{Introduction}

\section{Diffuse large B-cell lymphoma: treatment and state-of-the-art risk prediction}

Diffuse large B-cell lymphoma (DLBCL) is the most common type of B-cell lymphoma in adults, 
accounting for approximately 30\% of all diagnoses. This aggressive and heterogeneous group of 
lymphoid neoplasms typically originates from malignant transformed germinal center (GC) B cells, 
exhibiting diverse phenotypic, genetic, and clinical characteristics. The clinical presentation of 
DLBCL varies significantly, with differences in tumor load and patient performance status, leading 
to varied outcomes.

Despite being an aggressive and, if left untreated, fatal disease, DLBCL is a highly curable disease with the 
application of intensive immunochemotherapy even in the elderly population. The standard treatment 
for DLBCL has long been immunochemotherapy with the R-CHOP regimen. This 
regimen has significantly improved survival rates, with approximately two-thirds of patients 
achieving a cure. However, the remaining one-third of patients, especially 
those with relapsed or refractory disease, continue to face poor outcomes \citep{glass17}. For patients who do not 
respond to initial treatment, options include salvage chemotherapy and newer therapies like CAR-T cells. 
Approximately one-third of patients with DLBCL succumb to their disease, particularly those with 
relapsed or refractory conditions, for whom cure rates remain low. This underscores the clinical 
need for an accurate, robust, affordable, and easy-to-use tool that can identify patients at high risk 
for treatment failure early in their treatment course, ideally within the first three cycles of 
induction chemotherapy. 

To this end, the International Prognostic Index for non-Hodgkin's lymphoma (IPI) was established in the 
1990s. It incorporates five clinical binary clinical features one can measure cheaply and reliably without batch 
effects: Is the patient older than 60 years? Is the cancer advanced (Ann Arbor Stage III or IV)? Does 
the patient have a higher-than-normal lactate dehydrogenase (LDH) level? Is the patient already 
bedridden (performance status > 1)? Is the patient's cancer in more than one extranodal site? The 
IPI is then the number of positive answers to these questions, an integer between 0 and 5. Although 
primitive and arbitrary at first glance, the IPI is the result of a rigorous statistical analysis 
of a large dataset of \num{1872} patients: out of twelve candidate features, the IPI inventors
first selected those features that were independently and significantly associated with survival 
(namely the five above mentioned features), and fit a Cox proportional hazards model to them;
since the relative risks for all five features turned out to be similar, they simplified the model 
by just counting the number of present risk factors~\cite{ipi93}. 

The IPI is a simple yet robust clinical tool used globally to predict risk and guide 
treatment decisions in DLBCL patients. It has been the cornerstone of risk assessment for the last 
three decades, no alternative has gained widespread acceptance outside of clinical 
trials~\citep{ipi-stay-strong}. 
Despite its effectiveness in large cohorts, the IPI and other individual 
biomarkers do not reliably predict the clinical course for each patient, particularly failing to 
identify those at high risk for early treatment failure who may benefit from alternative therapeutic 
approaches. E.g., in the prospective trial comprising \num{466} patients used as test set in 
\citep{staiger20}, only \num{3.4}\% of patients have the maximum IPI score of 5 -- too few
to gain special attention in clinical practice and to incentivize the pharmaceutical industry to 
develop new treatments. All other cohorts defined via $\text{IPI} \geq i, i = 0, 1, 2, 3, 4$, lack 
precision: the proportion of patients with progression-free survival less than two years is below 
\num{50}\% -- too few to persuade a patient to undergo an experimental treatment instead of the 
standard R-CHOP regimen.

For this thesis, we define high-risk DLBCL patients as those who face cancer progression 
within two years after the start of the treatment. Two years is a time frame accepted by both 
patients and clinicians, which makes roughly a fourth of DLBCL patients high-risk \cite{staiger20}.
It's also the threshold used in the MMML-Predict we will introduce next.

\section{The MMML-Predict project}

Renowned lymphoma experts from across academic Germany -- clinical trialists, biostatisticians, 
bioinformaticians, lymphoma pathologists and translational lymphoma biologists -- have formed the 
consortium MMML-Predict to develop and roll out a new, robust, simple-to-use, cost-effective and 
parsimonious prognostic tool for DLBCL 
which yields a clinically more relevant high-risk group. This tool, the MMML-Predictor, will allow 
patients and clinicians early in the treatment cycle to make an informed decision if they want to 
continue with the standard R-CHOP treatment or switch to novel, more experimental treatments. 

In a discovery phase, MMML-Predict will enroll 200 patients in a prospective trial at first diagnosis and 
collect all clinical and molecular risk features that alone predict an unfavorable outcome, 
including clinical scores (like the IPI), gene-expression based factors (like cell-of-origin signatures,
immune scores) and genetic determinants (like MYC, BCL2, TP53, germline and somatic signatures).
As a novel approach, they will measure dynamic response determinants during treatment (PET-CT 
and liquid biopsy-based MRD detection). It is unknown if these features capture similar or different 
risks; combining them may finally bring the significant progress in the understanding of the DLBCL 
biology we have made over the last decades to the patients' bedside.

The group around Markus Loeffler in Leipzig will evaluate the readily trained MMML-Predictor on a 
test cohort of another 100 patients enrolled for this project for whom only those features used 
in the MMML-Predictor will be measured. The new classifier has to achieve a rate of positive 
predictions -- or prevalence -- of at least \num{15}\% and a precision (for high-risk) significantly above that of 
$\text{IPI} \geq 4$ on the test cohort, meaning the \num{95}\% confidence interval of the new 
classifier's precision according to the Clopper-Pearson method must not include the precision of 
the group determined by $\text{IPI} \geq 4$. In other prospective trials, this precision of the IPI 
is at around \num{35}\%; taking this number, a prevalence of \num{15}\% and the test cohort size into 
account, a calculation of the MMML-Predict consortium suggests that a precision of at least \num{50}\%
of the MMML-Predictor suffices.

\section{The role of this thesis within MMML-Predict}

Inside MMML-Predict, Rainer Spang's group is responsible to develop the MMML-Predictor. This is 
a supervised-learning task with a binary classification problem: progression-free survival less than 
two years -- or high-risk DLBCL -- is the positive group, progression-free survival more than two 
years -- or low-risk DLBCL -- is the negative group. We have a small number of samples, a large 
number of features and an enormous amount of freedom in how we design and train the classifier. 
Hence, we need to deal with the curse of high dimensions if we want to use the high-dimensional part 
of the data directly (and not indirectly via late integration of already-existent signatures) and, 
more importantly, we need to take care that in our internal validation (usually a cross-validation)
we do not overfit the data. While we should be able to tackle the first problem with regularization,
for the latter one we need a trustworthy internal validation strategy and, most importantly, we 
must not validate too many models in the first place.

Since, as of July 2024, the MMML-Predict train cohort hasn't yet arrived in Regensburg, this thesis
will imitate the train-test scenario of MMML-Predict on already-available DLBCL data sets. This thesis 
has two main goals:
\begin{enumerate}
    \item We want to show that with data including traditional clinical and modern 
        molecular features, both possibly already condensed to signatures, we can indeed deliver 
        the desired model, which yields a larger and more precise high-risk group of DLBCL patients.
        With an eye to rolling out the MMML-Predictor in clinical practice, we want to demonstrate 
        that we can design this model in such a way that one can transfer it from one platform to
        another without losing its predictive power.
    \item We want to develop heuristics and recommendations to answer the question which candidate 
        models (including their hyperparameters) are worth training and validating and--more 
        importantly--which are not. For this, we need to infer which models we can reliably validate 
        and which perform well on independent test data. These findings will guide us on how to keep 
        the number of candidate models we fit to the MMML-Predict training data low, thereby helping 
        us to avoid overfitting in the internal validation and to submitting a model that convinces 
        in the validation, but disappoints on the test cohort in Leipzig.
\end{enumerate}
    \chapter{Methods} \label{chap:methods}

Finding the best possible model for our given task will not be possible from just theoretical 
considerations; we will have to fit several models to our data and demonstrate the performance of 
the chosen one convincingly. Section \ref{sec:train-val-test} will lay out the state-of-the-art 
train-validate-test paradigm for this. In section \ref{sec:candidate-models}, we will introduce 
the candidate models and the hyperparameters governing their fitting process. We will start with 
model-agnostic hyperparameters before we go on to present well-known model classes and finally 
introduce a method that lets us train compositions of multiple models.

\section{Training, validation and testing}\label{sec:train-val-test}

The design of MMML-Predict with its train and test cohort, where the people developing the predictor 
never get to see the test cohort, sets the scene for the standard two-step approach in 
supervised-learning tasks: We use the train cohort to fit multiple models (training) and choose the model
among these models that we have confidence performs best on new data (validation). We then touch 
the test cohort for the first time as we evaluate the chosen model on it to persuade the outside 
world we have come up with model worth deploying.

Since we want to have the same conditions for data used in this thesis as later in MMML-Predict,
we randomly split given data $(\mathbf{x}, y)$ with predictor matrix $\mathbf{x}$ and corresponding 
response $y$ into a train cohort $(\mathbf{x}_\text{train}, y_\text{train})$ and a test cohort 
$(\mathbf{x}_\text{test}, y_\text{test})$. The following assumes a single, split data set.

\subsection{Training and validation}\label{subsec:train-val}

To be able to discuss some probabilistic caveats of validation later, we introduce some formal 
notation. We start with a set of tuples of hyperparameters $H$, where every $h \in H$ defines a model
up to its parameters. Determining the parameters of a model, by definition, is the job of the 
algorithm optimizing a given loss function; we refer to this as fitting. Every $h \in H$ specifies 
a candidate model and there is a one-to-one mapping between $H$ and the set of candidate models.

For every hyperparameter tuple $h \in H$, we
\begin{itemize}
    \item fit the model to the train cohort subject to $h$ and provide validated predictions for 
        every sample in the train cohort.
        A validated prediction for a sample is a prediction made by a model fit 
        subject to $h$ to data that does not include this sample. We obtain a vector of validated 
        predictions $\hat{y}_\text{train} = \text{val}(h)$ of the same shape as $y_\text{train}$.
        The most common method to obtain validated predictions is a $k$-fold cross validation.
        This means we randomly assign the training samples into $k$ subsets, called folds, and then actually 
        fit $k+1$ models, one model on all samples and for every $i = 1, \ldots, k$, we train a model
        on all samples \textit{except} the $i$-th fold and obtain its cross-validated predictions 
        \textit{on} the $i$-th fold. Doing this for all folds, 
        we obtain a cross-validated prediction for every sample.
    \item Afterwards, we calculate the validation error 
        $\text{err}(y_\text{train}, \hat{y}_\text{train})$.
\end{itemize}

Finally, we select the model $m^*$ with minimal validated error defined by the hyperparameter tuple 
\begin{align}
    h^* = \argmin_{h \in H} \ \text{err}(y_\text{train}, \text{val}(h)).
\end{align}

$\text{val}$ is not a deterministic mapping. The procedure to calculate validated predictions often 
involves randomness, e.g. how we split the training samples into folds in a cross validation and 
many fitting algorithms like that of random forests resort to randomness. The computer does this 
independently between the $\text{val}(h)$. Therefore, we treat the validation errors
$\text{err}(y_\text{train}, \text{val}(h))$ as independent 
real random variables. It is a well known property of independent, identically distributed (i.i.d.) 
real random variables $X_i, i \in \NN$, that their extreme values are notoriously unstable: for all 
$t \in \RR$ with $P(X_1 \geq t) < 1$, 
\begin{align}
    \Pr\left( \min_{1 \leq i \leq n} X_i \geq t \right) = \Pr(X_1 \geq t)^n \to 0 \quad
    \text{as } n \to \infty.
\end{align}

To avoid this disaster, which is nothing but overfitting the validated predictions to the training 
data, we need $\text{err}(y_\text{train}, \text{val}(h))$ with clearly distinct distributions. 
How can we achieve this this? First, we can choose $H$ in such a way that the fit models are 
notably distinct. There is, however, a trade-off: the rougher and smaller $H$ is, the more distinct 
the models and their validated errors are, but the higher the chance is that we miss out on a very good 
model. Vice versa, the larger and more granular $H$ is, the more likely one of the $h \in H$ 
specifies a very performant model, but the less likely validation is to reveal this model because,
due to near-i.i.d.\ validation errors in subsets of $H$, another, actually worse model might sneak 
in a greatly underestimated validation error. Second, differrent methods to calculate validated
predictions can leverage differences in the distribution of the 
$\text{err}(y_\text{train}, \text{val}(h))$ differently. We will not just use cross validation, but 
also so-called out-of-bag predictions in the case of random forests in this thesis. Third, the 
magic bullet of machine learning also helps cure this problem: more training samples will make it 
harder to overfit the validated predictions.

What do we conclude from this for practice? First, we aim to choose $H$ in a smart and lean way. 
We do not want to have big subsets of $H$ that specify more or less the same model. This is hard 
to foresee, but criteria may be theoretical considerations and experience by both us and others 
like default or automatically generated hyperparameters. Second, once we have fit and validated 
all models specified in $H$ and tested the chosen one, we can unlock the test data for all other 
models to analyze the relationship between validated and tested errors. We can look out for subsets 
of $H$ with validated errors strikingly deviating from tested errors or with high test errors and 
exclude them from $H$ in the future. Third, we try to increase the number of samples in training 
and test data by combining data sets. This comes with some caveats we will deal with later, but 
we hope that more reliable validation errors and better models will compensate for this.

\subsection{Testing}

We calculate the predictions $m^*(X_\text{test}) = \hat{y}_\text{test}$ of the best validated model 
$m^*$ on the test cohort and estimate its performance on independent data via 
\begin{align}
    \text{err}(y_\text{test}, \hat{y}_\text{test}).
\end{align}

\subsection{Choice of the error function}\label{sec:error-function}

As stated in section \ref{sec:intro-mmml}, the MMML-Predictor should detect a high-risk group 
with an as-high-as-possible precison under the constraint that its prevalence must surpass 
\num{10}\%. Since an error should be the lower, the better the model is, we choose 
$\text{err}$ to be the minimum of the negative precisions with a prevalence of at least 17\%. We 
add seven percentage points to the \num{10}\% prevalence as it increases statistical power and makes 
the error function more robust as we will see next. 

Usually, we need 
to take a minimum over \textit{several} precisions because most models do not output the final 
classification.
Instead, they return a continuous score where a higher score means a higher vote of being
in the positive class, in our case being high-risk. This continuous score needs thresholding
via $\hat{y}_i > t$ for some $t \in \RR$.
We notice that as we increase $t$ and thereby decrease the prevalence, the obtained 
adjacent precision values fluctuate less and less; for high $t$ and low 
prevalences, fluctuations of up to \num{50} percentage points would be possible, but requiring a 
prevalence of at least 17\% caps such fluctuations at 
\begin{align}
    \frac{1}{\ceil*{\num{0.17} \cdot n_\text{test}}}
\end{align}
for $n_\text{test}$ samples in the test cohort as an easy calculation shows, cf. also Fig.\ 
\ref{fig:inter-output-prec} in practice. This gives us an error function that is tailored 
for our problem and robust.

As indicated by the notation $\text{err}(y_\text{train}, \hat{y}_\text{train})$ and 
$\text{err}(y_\text{test}, \hat{y}_\text{test})$, we optimize $t$ on the true outcomes 
of the train cohort when validating and we optimize it again on the true outcomes of the test cohort 
when testing. Strictly speaking, $t$ is a hyperparameter, which we optimize on the 
test cohort during testing; this sounds delicate. The results however will show that the optimal 
choice for $t$ (or at least an 
almost optimal choice) on both train and test cohort correponds to a prevalence of 
slightly above 17\% such that one can agree on the following when it comes to the MMML-Predict data: 
choose $t$ as the 17\%-quantile of the continuous model output on the test cohort.

Similarly, given some cohort, we threshold the IPI in such a way that it maximizes the precision under 
the constraint of a prevalence of at least \num{10}\% on this cohort and call this truly 
binary classifier thresholded IPI, in short tIPI. Usually this amounts to thresholding the IPI via 
$\text{IPI} \geq 4$. The IPI only needs to deliver the minimum 
attention-triggering prevalence of \num{10}\% while its challengers, our models, need a somewhat 
higher prevalence to gain statistical power against the IPI as explained before. Also optimizing 
the IPI's threshold for a given cohort ensures a fair competition.

In general, given data $(\mathbf{x}, y)$, we denote the prevalence of a model $m$ correspoding to 
$\text{err}(y, m(\textbf{x}))$ by $\text{prev}_{\mathbf{x}, y}(m)$ and the corresponding precision 
by $\text{prec}_{\mathbf{x}, y}(m) = -\text{err}(y, m(\textbf{x}))$. When it is clear from the 
context which data set or subset of a data set we are talking about, we just write $\text{prev}(m)$ 
and $\text{prec}(m)$.

\subsection{Inter-technical variability} 
One could instead suggest to already optimize the output threshold of our models on the cross-validated 
predictions. While this promises a stricter train-test regime at first glance, fixing the output 
threshold once and for all neglects \textit{inter-technical variability}. 

An always-present problem in bioinformatics, inter-technical variability refers to the fact that 
one and the same sample measured on different platforms, in different labs or at different times, 
in short: under different protocols, may lead to different values for the same feature. 
Values for the same feature and sample measured in different batches, even if the lab, preparation 
and technology are the same, may differ. We will therefore also use the term batch effects for 
inter-technical variability. Every data set adheres to its own protocol, so we need to deal with 
inter-technical variability whenever we combine data sets or have a model predict on a data set 
other than the one it was trained on.

If we 
measure the same $p$ features of the $n$ patients in $\mathbf{x} \in \RR^{n \times p}$ again under 
another protocol, we end up with another predictor 
$\mathbf{z} \in \RR^{n \times p}$.
For logarithmized gene-expression levels,
we can often well model the discrepancies with two indepedent biases, sample-wise effects $\theta_i$
and feature-wise effects $\omega_j$, leading to
\begin{align}\label{eq:inter-tech-exact}
    \Delta_{ij} = \mathbf{z}_{ij} - \mathbf{x}_{ij} = \theta_i + \omega_j + \epsilon_{ij}
\end{align}
for residues $\epsilon_{ij}$. Assuming accurate modeling, i.e. small residues, we can well approximate
\begin{align}
    \mathbf{z}_{ij} \approx \tilde{\mathbf{z}}_{ij} = \mathbf{x}_{ij} + \theta_i + \omega_j.
\end{align}
We can now apply an ordinary linear model with parameters $(\beta_0, \beta)$ to $\mathbf{z}$ 
and obtain
\begin{align} \label{eq:inter-tech}
\begin{split}
    \beta_0 + \sum_{j=1}^p \beta_j \mathbf{z}_{ij} &\approx \beta_0 + \sum_{j=1}^p \beta_j \tilde{\mathbf{z}}_{ij} \\
    &= \beta_0 + \sum_{j=1}^p \beta_j \mathbf{x}_{ij} + \sum_{j=1}^p \beta_j \theta_i + \sum_{j=1}^p \beta_j \omega_j.
\end{split}
\end{align}
The third summand is zero under the zero-sum constraint $\sum_{j = 1}^p \beta_j = 0$. The fourth 
summand is constant across all samples and can be absorbed by the intercept, cf.\ 
\cite{transplatform17}. 

Under these assumptions, which are pretty realistic,
going from one protocol to another just leads to a constant shift of the model 
output; even for generalized linear models, i.e. a Gauss model composed with some monotic link function,
inter-technical variability does not change the ordering of the samples. All one needs to do to 
obtain a final classification is calibrating the threshold for the output scores. However, this 
also demonstrates that while generalized linear models can cope with inter-technical variability 
pretty well, there is no point in using the same output threshold across all protocols.

\section{Candidate models}\label{sec:candidate-models}

\subsection{Model-agnostic hyperparameters}\label{subsec:model-agnostic}

Model-agnostic hyperparameters are those hyperparameters we can set and tune for every model. They 
concern the predictor matrix $\mathbf{x} \in \RR^{n \times p}$ and the response vector $y \in 
(\RR \times \{ 0, 1 \})^n \cup \{ 0, 1 \}^n$. We can provide the response in two formats. In the 
first case, the response has two entries per sample: the first one is the time to event, in our 
case PFS, the second one is the censoring status and is 1 if the event did occur at this time and 
0 if the sample was censored at this time before the event could occur; more on this in subsection 
\ref{subsec:core-models} when we present the Cox model. In the second case, the response is a binary 
vector with 1 encoding the positive class, high-risk DLBCL,
and 0 encoding the negative class, low-risk DLBCL. Note that in the latter case we need 
to discard samples censored before two years, but this is typically only a small proportion of the 
samples in the data set.

\subsubsection{A-priori feature selection}

Not every feature in the data set should be part of the predictor matrix $\mathbf{x}$. Including 
features measured toward the end of a patinent's therapy or even after it, is cheating and we 
should definitely exclude them. Excluding any of the remaining features is hard to justify: If we 
know for a feature that it alone is associated with the response, we include it in $\mathbf{x}$ 
without doubt.
On the other hand, if a feature alone shows no link to the response, the right model might still be 
able to leverage it in combination with other features; but at this step, we do not know if such 
a model exists and if it is among our candiate models. The only remaining option for handpicking 
features a priori is to brute-force the problem and try out all $2^p$ combinations of features.
Even if we trust on regularization to do its job for the high-dimensional part (typically several 
hundreds or even thousands of gene expression levels), just \num{10} remaining features would still
leave us with $2^{10} = 1024$ combinations to try out -- per model class. While this may be 
computationally feasible, validation would be a statistical fiasco as laid out in subsection 
\ref{subsec:train-val}. Therefore, we reduce this quickly exploding number of choices to only 
a handful and have all the other feature selection happen during model fitting. 

\paragraph{Gene-expression data}
The first and most important decision to make is: do we want to include the high-dimensional
gene-expression part of the data at all? If we do not, we suddenly have $p \ll n$ instead of 
$p > n$ or even $p \gg n$. This strongly effects the models we fit, hence also $H$ and 
validation.

\paragraph{Features in different formats}
Second, we need to make a couple of more decisions regarding the format of some features.
\begin{itemize}
    \item Concerning the IPI, one can include the five IPI features in their original, continuous 
        form; discretized as by the IPI inventors; the IPI score as a single continuous feature; 
        or the so-called IPI group, a partioning of the IPI score into three groups. Similarly, 
        widely accepted thresholds may exist for other clinical and genetic features.
    \item For already existing models, we can either include their continuous output or threshold 
        it. E.g., cell-of-origin signatures can be hresholded into ABC, GCB and unclassified 
        subtypes. Data sets of only provide this grouping and not the raw continuous score. The 
        above-mentioned IPI group thresholds the IPI model. 
\end{itemize}

E.g., imagine a data set provides the five IPI features in continuous format, from which we can 
easily infer all other mentioned formats, and the continuous output of some gene-expression 
signature for which its inventors also provide a preferred threshold. If we want to decide for 
exactly one format per feature, this leaves us with $4 \cdot 2$ possibilities for this simple case. 
If we want try out every combination of features in any format, the number of possibilities jumps to 
$2^4 \cdot 2^2 = 64$. 

Our solution here again is to be generous and include all formats of a feature in $\mathbf{x}$ a 
certain model may benefit from. This sentence usually 
simplifies to do: include all widely used formats of a feature in $\mathbf{x}$. E.g., generalized 
linear models cannot threshold continuous features, so a feature in its thresholded format may give 
a rough contribution to the prediction while the same feature in its continuous form may further 
refine it. Moreover, even decision trees can benefit from a feature additionally provided in its 
thresholded format if this threshold has been inferred on a much bigger data set since the decision 
tree may not be able to find the threshold itself.

\subsubsection{Imputation}

If values in $\mathbf{x}$ are missing, i.e. written as \texttt{NA}, actions we can take fall into two 
categories: we discard the part of data affected by missing values or we replace the missing values 
with some realistic estimate.

\paragraph{Discarding part of the data}

In a first step, we discard a feature if it is not available for a large proportion of the samples.
This enables reasonable computing.

\paragraph{Imputing}

At this point, we dichotomize a categorical 
feature with $c$ categories into $c-1$ binary dummy features; if for a sample this categorical feature is not 
available, all $c-1$ dummy features will be \texttt{NA}. This yields a modified $\mathbf{x}$ that 
bears exactly the same information as before. For \textit{every} column in $\mathbf{x}$, we calculate the 
mean of the available features in $\mathbf{x}$ and replace the \texttt{NA} values with it. 
For a missing categorical feature, every dummy feature therefore holds the marginal Bernoullie 
probability that the feature is in the $k$-th category. This is efficient, transparent and easily 
applicable to new data.

\subsubsection{Adding combinations of discrete features to the predictor}

Next, we add all combinations of at most $n_\text{combi}$ categorical features that are positive in a 
share of at least $s_\text{min}$ patients to $\mathbf{x}$; e.g., we add a column ``female \& ABC-type 
tumor'' if at least \num{5}\% of patients have this property. We always choose $s_\text{min}
= 5\%$. We set $n_\text{combi} = 1$ for models that facilitate interactions between features 
themselves, otherwise we set $n_\text{combi} = 3$.

Technically, we need to combine the binary features we derived from distinct categorical features 
in the previous step. We realize this by multiplying the corresponding columns in 
$\mathbf{x}$ element-wise. We thereby keep treating the value of a binary feature as a Bernoullie
probability under the assumption that the binary dummy features involved in the combination are 
independent, which is not always justified. The average of a combined (Bernoullie) feature over all 
samples, for which we apply the threshold $s_\text{min}$, therefore approximates the expected value 
of the combined feature.

\subsubsection{Tuning the definion of ``high risk''}

Defining a patient as high-risk if and only if the PFS is less than two years
is a clinically accepted, yet quite arbitrary decision. It may not be the time cutoff that 
separates PFS best from a biological point of view. Therefore, we can provide the fitting 
algorithm a refined response $y$, governed by the \textit{training survival cutoff} $T > 0$.
\begin{itemize}
    \item For binary response $y \in \{0, 1\}^n$, we set $y_i = 1$ if the PFS 
        of patient $i$ is less than $T$ and $y_i = 0$ otherwise. The higher $T$ is, the more 
        samples we need to discard due to censoring.
    \item For response $y \in (\RR \times \{0, 1\})^n$ with time to event and censoring, we censor 
        all samples with time to event exceeding $T$ at $T$.
\end{itemize}
Patients may separate much more clearly for some $T \neq 2$ and as long as we come up 
with a positive cohort that comprises at least 15\% of the samples of the test data, this allows us 
to fulfill our task more easily. We stress that $T$ only influences how we train models. For 
validation and testing, a PFS of \num{2} years stays the crucial cutoff.

\subsection{Core models}\label{subsec:core-models}

All models trained, validated and tested in this thesis at the core consist of ordinary linear, 
logistic and Cox proportional-hazards models with additional properties, as well as random forests.
In this chapter, we want to introduce the design of these models and the hyperparameters governing 
their fitting process.

Formally, we deal with a probability space $(\Omega, \mathcal{A}, P)$ that we do not and cannot 
specify any further because we only get in touch with two random variables 
\begin{align}
    X = (X_1, \ldots, X_p): \Omega \to \RR^p \text{ and } Y: \Omega \to \RR,
\end{align}
the predictor and the response, respectively. More 
precisely, we observe independent training samples $(x_i, y_i) \in \RR^{p+1}, i = 1, \ldots, n$, 
distributed according to $(X, Y)$; $x_i$ is the $i$-th row of the predictor matrix $\mathbf{x}$ and
$y_i$ is the $i$-th entry of the response vector $y$. The i.i.d.\ test samples follow the same 
distribution as the training samples, but in this subsection everything is about training. If we 
say samples in this subsection, we always refer to the training samples.

\subsubsection{Generalized linear models}\label{subsubsec:glm}

Here, we work with binary response, i.e. $\mathrm{im}(Y) \subset \{0, 1\}$. Both ordinary 
linar models and logistic regression models are generalized linear models (GLMs). In 
a GLM, $Y$ follows an exponential-family distribution and there is an invertible link function 
$g: \RR \to \RR$ and parameters $(\beta_0, \beta) \in \RR^{p+1}$ such that
\begin{align}
    g(E(Y \cond X = x)) = \beta_0 + \sum_{j=1}^p \beta_j x_j \quad \text{for all } x \in \RR^p,
\end{align}
where the $E(Y \cond X = x)$ is the expected value of $Y$ given $X = x$. For the above samples 
$(x_i, y_i)$, we obtain
\begin{align}
    \mu_i = E(Y \cond X = x_i) = g^{-1}\left(\beta_0 + \sum_{j=1}^p \beta_j x_{ij}\right).
\end{align}
Relating $\mu_i$ to $y_i$ via a log likelihood will allow us to fit $(\beta_0, \beta)$ to 
$(x_i, y_i)_{i = 1, \ldots, n}$. We can express the \textit{linear predictor} 
$\sum_{j=1}^p \beta_j x_{ij}$ in terms of the Euclidean scalar product as $x_i^T \beta$, where 
$x_i^T$ denotes the transpose of the column vector $x_i$.

\paragraph{Gauss model}
Here, $g = \text{id}$. $Y$ and $Y \cond X = x_i$ for all $i = 1, \ldots, n$ follow a normal
distribution with fixed standard deviation $\sigma > 0$ (a property called homoscadasticity), but 
varying mean. The log likelihood of $\mu_i$ is
\begin{align}
\begin{split}
    \ell(\mu_i; y_i) &= \log\left( \frac{1}{\sqrt{2\pi}\sigma} 
        \exp \left( -\frac{1}{2\sigma^2}(y_i - \mu_i)^2 \right) \right) \\
    &= -\frac{1}{2\sigma^2}(y_i - \mu_i)^2 - \log \left( \sqrt{2\pi}\sigma \right).
\end{split}
\end{align}
Up to a constant shift and re-scaling, which does not affect maximizing the log likelihood, this
is the well-known squared error.

\paragraph{Logistic model}
Here, the linear predictor is the log-odds for $Y = 1$ over $Y = 0$ given $X = x$, i.e.
\begin{align}
    g: \mu \mapsto \log\left( \frac{\mu}{1 - \mu} \right), \text{ hence } g^{-1}: \eta \mapsto
    \frac{1}{1 + \exp(-\eta)}.
\end{align}
Y and $Y \cond X = x_i$ for all $i = 1, \ldots, n$ follow a Bernoulli distribution with parameter 
$p$ and $\mu_i$, respectively. The log likelihood of $\mu_i$ is
\begin{align}
    \ell(\mu_i; y_i) &= \log\left( \mu_i^{y_i} (1 - \mu_i)^{1 - y_i} \right) 
    = y_i \log(\mu_i) + (1 - y_i) \log(1 - \mu_i). 
\end{align}

\paragraph{Loss function}
Using $\mu_i = g^{-1}(\beta_0 + x_i^T \beta)$, in both cases we can express the 
log likelihood 
for $(\beta_0, \beta)$. For our i.i.d.\ samples $(x_i, y_i)_{i = 1, \ldots, n}$, we therefore obtain 
\begin{align}
    L(\beta_0, \beta) &= \sum_{i=1}^n \ell(\mu_i; y_i) 
    = \sum_{i=1}^n \ell\left( g^{-1}\left( \beta_0 + x_i^T \beta \right); y_i \right).
\end{align}
We can augment $L$ with two hyperparameters, sample weights $w_i > 0$, $i = 1, \ldots, n$, and the 
zero-sum constraint with respect to zero-sum weights $u_j \geq 0$, $j = 1, \ldots, p$, yielding 
\begin{align} \label{eq:loss-glm-no-lasso}
    L(\beta_0, \beta) &= \sum_{i=1}^n w_i \ell\left( g^{-1}\left( \beta_0 + x_i^T \beta \right); 
    y_i \right) \quad \text{subject to } \sum_{j=1}^p \beta_j u_j = 0.
\end{align}
While we do not deviate from default value $1/n$ for the sample weights in the results chapter 
\ref{chap:results}, we will discuss in chapter \ref{chap:discussion} how we can use them in the 
future in a natural way without getting lost in the vast amount of freedom one has to choose them. 

Fitting a model to our data means finding a model parameter that minimizes the loss function. Here,
\begin{align}
    (\hat{\beta_0}, \hat{\beta}) = \argmin_{(\beta_0, \beta) \in \RR^{p+1}} L(\beta_0, \beta).
\end{align}

We want to demonstrate what the zero-sum constraint is good for in a more general setting. Imagine, 
part of the features suffer from sample-wise 
shifts. We denote these features by $J \subset \{1, \ldots, p\}$ and the shifts by $s_i \in R$, 
$i = 1, \ldots, n$. We now set the zero-sum weights  
\begin{align}
    u_j =
    \begin{cases}
        c & \text{if } j \in J, \\
        0 & \text{else},
    \end{cases}
\end{align}
for some $c > 0$ (usually $c = 1$); we can express this conviently with the help of the indicator 
function $\chi_J$ as $u_j = \chi_J(j) c$. By the zero-sum constraint, $\sum_{j \in J} c \beta_j = 0$  
and hence $\sum_{j \in J} \beta_j = 0$. In this situation, the linear predictor 
\begin{align}
    \beta_0 + \sum_{j=1}^p \beta_j (x_{ij} + s_i \chi_J(j))
    = \beta_0 + x_i^T \beta + \sum_{j \in J} \beta_j s_i = \beta_0 + x_i^T \beta,
\end{align}
as $\sum_{j \in J} \beta_j s_i = s_i \sum_{j \in J} \beta_j = 0$, is invariant. If $J$ consists of 
exactly the features holding gene-expression levels, we can transfer our model to other protocols 
and only need to re-calibrate the intercept, as shown in Eq. \eqref{eq:inter-tech}. Moreover, we 
can switch off the zero-sum constraint for all features by setting $u_j = 0$ for all 
$j = 1, \ldots, p$. The zero-sum constraint is cheap in terms of model complexity: it only removes 
one degree of freedom; but it is expensive in terms of computational complexity: enforcing it in
every step of the coordinate descent leads to a considerably higher computation time. 

As we have stressed several times already, the threshold that makes a GLM a binary classifier 
is rather a property of the underlying data set than of the model itself. The essential information 
the modelling function contributes to the classification of a cohort of samples is therefore the 
ordering of the output scores. Because $g$ and therfore also $g^{-1}$ are monotonically increasing, 
the essential part of the model is the linear predictor $x_i^T \beta$. For this reason, when we 
talk about the output of a GLM, we always mean the output of the linear predictor. For us, this 
makes $g$ more a tweak in the loss function, important for training, than a part of the modelling 
function.

We are not done, yet, with $L$: the coordinate descent minimizes 
Eq.\ \eqref{eq:loss-glm-no-lasso} with a regularization term added. Before we discuss this, we want 
to introduce a model that is closely related to GLMs.

\subsubsection{Cox proportional-hazards model}\label{subsubsec:cox}

For the Cox proportional-hazards model -- in short: Cox model -- the response variable $Y$ 
measures the time \textit{until} the event of interest occurs (``event'') or the time 
\textit{after} which the event occurs (``censoring''). Another binary random variable 
$\delta: \Omega \to \RR$ encodes 
which of these two options is the case: $\delta(\omega) = 1$ for the event, $\delta(\omega) = 0$ 
for censoring. This pays tribute to an important characteristic of survival trials: some
participants drop out of the trial before the event could happen -- e.g. the trial terminated, the 
patient decided to withdraw or died from another cause -- or the event luckily never happens.

\paragraph{Hazard function} To understand what the Cox model predicts, we need another random 
variable $T: \Omega \to \RR$, the time to the event. Unlike $Y$, $T$ is not affected by censoring. 
We require $T$ to have a density $f_T: \RR \to \RR$ and define its hazard function $h: \RR \to \RR$
via
\begin{align}
    h(t) = \lim_{\Delta t \to 0} \frac{P(t \leq T < t + \Delta t \cond T \geq t)}{\Delta t}.
\end{align}
We can interprete $h(t)$ as the instantaneous rate at which the event is occurring at time $t$, 
given that the event has not occured before time $t$. Let $F_T$ be the distribution function of 
$T$. Then 
we can write
\begin{align}
    \frac{P(t \leq T < t + \Delta t \cond T \geq t)}{\Delta t} = 
    \frac{F_T(t + \Delta t) - F_T(t)}{\Delta t \cdot (1 - F_T(t))}.
\end{align}
Hence, 
\begin{align}
    h(t) = \lim_{\Delta t \to 0} \frac{F_T(t + \Delta t) - F_T(t)}{\Delta t \cdot (1 - F_T(t))} 
    = \frac{f_T(t)}{1 - F_T(t)} = \frac{f_T(t)}{S_T(t)},
\end{align}
where $S(t) = 1 - F_T(t)$ is the survival function, and $h(t)$ is well-defined as long as $S_T(t)
> 0$.

\paragraph{Conditional hazard} The Cox model proposed in \cite{cox72} wants to get a hand on 
\textit{conditional} hazards.
Intuitively, it is clear that the population with one value of $X$ can have a vastly different 
hazard function than the population with another value of $X$ if $X$ has predictive power for $T$.
Formally, however, it is hard to define a conditional hazard function: $X$ may very well 
follow a continuous distribution, hence we condition on an event of probability $0$ when we write 
$P(t \leq T < t + \Delta t \cond T \geq t, X = x)$ -- it is hard to define this in a natural
and straightforward way. In practice, the measurements in $X$ often fluctuate around their true 
values anyway, so it is sensible to condition on a small neighborhood of $x$, say an $\epsilon$-ball 
around $x$ with respect to the Euclidean norm $|\cdot|_2$ for some small $\epsilon > 0$. This 
amounts to
\begin{align}
    h(t \cond X = x) = \lim_{\Delta t \to 0} \frac{P(t \leq T < t + \Delta t \cond T \geq t, 
    |X - x|_2 < \epsilon)}{\Delta t}
\end{align}
or -- equivalently -- $h(t \cond X = x)$ is the hazard of $T$ restricted to $(\Omega, \mathcal{A}, 
P)$ conditioned on $\Omega_x = \{ \omega \in \Omega: |X(\omega) - x| < \epsilon \}$; unlike before, 
we usually now have $P(\Omega_x) > 0$.

\paragraph{Proportional hazards} The Cox model assumes \textit{proportional} conditional hazards, 
i.e. 
\begin{align}
    h(t \cond X = x) = \lambda_x h_0(t) 
\end{align}
for some (unspecified) baseline hazard $h_0$ and $\lambda_x > 0$. Moreover, it assumes $\lambda_x$ depends 
on $x$ via 
\begin{align}\label{eq:cox-proportional-factor}
    \lambda_x = \exp(x^T \beta),
\end{align}
for parameters $\beta \in \RR^p$.

\paragraph{Partial log likelihood}
Given our independent samples $(x_i, y_i, \delta_i) \in \RR^{p+2}$, $i = 1, \ldots, n$, distributed 
according to $(X, Y, \delta)$, let $E \subset \{ 1, \ldots, n \}$ refer to the non-censored samples 
($\delta_i = 1$). For $i \in E$, let $R_i$ denote the set of samples where no event has occurred 
until $y_i$, i.e. $y_j \geq y_i$ for all $j \in R_i$. For fixed $i \in E$ and time to event $y_i$,
conditionally on the risk set $R_i$, the probability that the event occurs on sample $i$ as 
observed is
\begin{align}
    \frac{h(y_i \cond X = x_i)}{\sum_{j \in R_i} h(y_i \cond X = x_j)} = 
    \frac{\exp(x_i^T \beta) h_0(y_i)}{\sum_{j \in R_i} \exp(x_j^T \beta) h_0(y_i)} = 
    \frac{\exp(x_i^T \beta)}{\sum_{j \in R_i} \exp(x_j^T \beta)}
\end{align}
according to \cite[Eq. (12)]{cox72}.

Taking all events into account yields the partial likelihood 
\begin{align}\label{eq:cox-partial-lh}
    \tilde{\ell}(\beta) = \prod_{i \in E} \frac{\exp(x_i^T \beta)}{\sum_{j \in R_i} \exp(x_j^T 
    \beta)}.
\end{align}
Because $T$ is continuous, there is a zero probability for two distinct samples having the same 
event time, so we need not resort to one of the more sophisticated likelihoods capable of handling ties.
Analogously to Eq. \eqref{eq:loss-glm-no-lasso}, we can equip the log likelihood with sample 
weights $w_i > 0$ and zero-sum weights $u_j \geq 0$ such that the refined log partial likelihood reads 
\begin{align}\label{eq:cox-log-lh}
\begin{split}
    L(\beta) &= \sum_{i \in E} w_i x_i^T \beta - \log \left( \sum_{j \in R_i} w_j \exp(x_j^T
    \beta) \right) \\
    & \text{subject to } \sum_{j=1}^p \beta_j u_j = 0.
\end{split}
\end{align}
As with GLMs, we can leverage the zero-sum constraint to reach shift-invariance.

There are more analogies to GLMs. In Eq.\ \eqref{eq:cox-proportional-factor}, we also compose the 
linear predictor with an increasing function. Given a cohort of samples, we are still most 
interested in the ordering of the output scores and the linear predictor $x_i^T \beta$ already 
uniquely determines this ordering. Again, when we talk about the output of a Cox model, we mean the 
output of the linear predictor. In light of all these similarities,
in this thesis we refer to both GLMs in the typical sense and Cox models as GLMs.

\paragraph{Hazard ratio}

We can use the Cox model to calculate a hazard ratio (HR) between two groups of samples. Suppose 
$g_i \in \{0, 1\}$ encodes this grouping. In our case, $g_i = f(x_i)$ is always the output of 
a binary classifier $f$. We now fit a univariate Cox model to $(g_i, y_i, \delta_i)$, $i = 1, 
\ldots, n$, by maximizing 
the partial likelihood in Eq.\ \eqref{eq:cox-log-lh} with respect to $\beta$ and obtain its scalar 
coefficient $\hat{\beta}$. By Eq.\ \eqref{eq:cox-proportional-factor}, $\exp(\hat{\beta})$
is the relative risk increase of being in group $g_i = 1$ compared to group $g_i = 0$ and we call 
it hazard ratio.

We can view the maximum partial-likelihood estimate $\hat{\beta}$ as a random variable that depends 
on i.i.d.\ sampling $n$ traning samples. In analogy with maximum-likelihood estimation, there are 
good arguments to believe that the maximum \textit{partial} log-likelihood estimator $\hat{\beta}$
is
\begin{itemize}  
    \item \textit{consistent}, i.e., with growing sample size $n$, $\hat{\beta}$ converges in 
        probability to the true coefficient of the Cox model $\beta_0$, and
    \item \textit{asymptotically normal}, i.e., with growing sample size, $\hat{\beta}$ converges 
        in distribution to a normal distribution with mean $\beta_0$ and a standard deviation we 
        can estimate with the help of the second derivative of the partial log likelihood with 
        respect to $\beta$ evaluated at $\hat{\beta}$ \cite[8.1--8.4]{klein03}.
\end{itemize}
This justifies to treat $\hat{\beta}$ as normal with known mean and variance and allows us to
calculate confidence intervals for the HR $\exp(\hat{\beta})$ as well as a two-sided p-value for 
the null-hypothesis $\exp(\hat{\beta}) = 1$, i.e. identical risk between the two groups. 

\subsection{Elastic-net regularization} \label{subsec:elastic-net}

Modern molecular measurements are a big factor in our hope that MMML-Predict can succeed and improve 
the IPI. These measurements, like gene expression levels, show up as hundreds if not thousands of 
features in our data, meaning we usually have $p > n$ if not $p \gg n$. In this situation, we 
cannot uniquely determine the parameters of any of the models by minimizing the loss function: 
the models have way more degress of freedom than we have samples, they will find biologically 
meaningless and non-reproducible structures in the training data and generalize poorly to new 
data.

A solution to this is to restrict the freedom of the model to the extent that it is forced to 
learn the gist from the data. For that reason, we want to penalize model complexity. Elastic-net 
regularization, a widely adopted method proposed by Zou and Hastie \cite{elasticnet05}, does this 
job for us. The elastic net generalizes two well-established regularization methods,
\begin{itemize}
    \item ridge regularization \cite{ridge70}, which forces the parameter vector $\beta$ to 
        stay inside a ball around the origin by limiting the squared $\ell_2$ norm 
        of the parameters, and 
    \item LASSO (least absolute shrinkage and selection operator) regularization \cite{lasso18}, 
        which forces $\beta$ to stay inside a diamont centered at the origin by limiting the 
        $\ell_1$ norm of the parameters,
\end{itemize}
by combining them with a weight factor $\alpha \in [0, 1]$ into 
\begin{align}
    p_{\text{enet}, \alpha}(\beta) = \frac{1-\alpha}{2} |\beta|_2^2 + \alpha |\beta|_1.
\end{align}
Muliplied with the regularization strength $\lambda \geq 0$, we can add it to the negative 
(refined) log likelihood to obtain the regularized loss function
\begin{align}
    \mathcal{L}(\beta_0, \beta) = -L(\beta_0, \beta) + \lambda p_{\text{enet}, \alpha}(\beta).
\end{align}
Note that in the Cox case, $\mathcal{L}$ does not depend on $\beta_0$ and we just ignore $\beta_0$ 
in the above equation. Fitting a model to training data means finding a minimizer $(\hat{\beta_0}, 
\hat{\beta})$ of the loss function $\mathcal{L}$.

\paragraph{Effect on correlated features}
While both ridge and LASSO regularization shrink the coefficients, they act differently in the case 
of correlated features. Ridge regression tends to force the coefficients of correlated features to 
similar values while LASSO tends to pick one of them and set the others to zero. E.g, in the 
extreme case of $k$ identical features, the LASSO arbitrarily picks one of them with, say, 
coefficient $a$; the ridge regression, meanwhile, assigns every of the $k$ features the coefficient 
$a/k$. Relying on very few features comes with the advantage of sparse models, for which we 
can cheaply generate new data, but it is very sensitive to measurement errors in one of the picked 
features; in the latter case, ridge regularization would average out errors with the help of other 
correlated features. Elastic-net regularization balances the sparsity of the LASSO with the 
robustness of the ridge regularization, especially for values of $\alpha$ close to $1$ 
\cite{elasticnet05}.

\paragraph{Feature-wise penalty weights}
We refine the elastic-net regularization one last time and introduce \textit{a priori} defined 
feature weights $v_i \geq 0$ yielding
\begin{align}
    p_{\text{enet}, \alpha}(\beta) = \sum_{j=1}^p v_j \left( \frac{1-\alpha}{2} \beta_j^2 +
    \alpha |\beta_j| \right).
\end{align}
The default value is $w_j = 1$ for all $i = 1, \ldots, p$. Deviating from the default can be useful 
in at least two scenarios. First, sometimes we are so convinced of the predicitve 
power of a feature in our model that we set $v_j = 0$ for it to all but ensure a non-zero 
coefficient for it; the IPI (in some format) could be such a feature. 

\paragraph{Standardizing $X$} 
The second application of the penalty weights is an aspect of a bigger topic: standardizing the
predictor. We can use the weights to standardize $X$ by setting $v_j$ to an estimate of the standard 
deviation of feature $j$. Why 
should we do this? Let us consider two features $X_1$ and $X_2$ with standard deviation $\sigma_1$
and $\sigma_2$, respectively. To change the output of the model by $1$ when the feature changes by 
one standard deviation and all other features keep their values, $X_1$ needs the coefficient 
$\beta_1 = 1/\sigma_1$ and $X_2$ needs $\beta_2 = 1/\sigma_2$. For $\alpha = 1$, 
$p_{\text{enet}, \alpha}$ penalizes both coefficients equally and it still does so approximately 
for $\alpha$ slightly below $1$. Thanks to standardization of $X$, $\beta_1$ and $\beta_2$ contribute 
$1$ to the elastic-net penalty. Standardization strives to garuantee equal justice for features 
on different scales. Standardization works, however, contrary to the zero-sum idea: thanks to the 
zero-sum constraint, the fitting process and the final model both are invariant under sample-wise 
shifts on a subset of features if we set the zero-sum weights accordingly. Sample-wise shifts 
change the standard deviation of the affected features. With standardization, in 
contrast, this changes the cost function, which probably results in a different model. 
Consequently, when deploying the zero-sum constraint in some way, one should stick to the default 
value $v_j = 1$ for the features for which we wish to have shift-invariance; these are usually the 
gene-expression levels, which are on comparable scales anyway.

\subsubsection{Random forests}

More precisely, we talk about classification random forests (RFs), hence $Y$ is binary in this subsection 
with $\im(Y) \subset \{ \pm 1 \}$, where $1$ still encodes the positive class. When we say random 
forests in this thesis, we mean classification 
random forests. Random forests are an ensemble of classification trees -- in short: 
trees -- and aggregate the classification of every constituent tree.

\paragraph{Trees}
A tree is a simple function
\begin{align}
    T = \sum_{m=1}^M c_m \chi_{R_m}: \RR^p \to \{ \pm 1 \}
\end{align}
with $c_m \in \{\pm 1\}$ and the $R_m$ being disjoint rectangle sets, i.e. $R_m = \prod_{j=1}^p 
(a_j, b_j]$ for $a_j, b_j \in \RR \cup \{ \pm \infty \}$. It aims to predict the conditional 
majority class 
\begin{align}\label{eq:rf-major}
    \sgn(E(Y \cond X)) = T(X);
\end{align}
$\sgn$ is the signum function.

How can we use our samples $(x_i, y_i)$, $i = 1, \ldots, n$, to learn a tree $T$? While we can 
trivially fulfill Eq. \eqref{eq:rf-major} exactly for our samples, this would only lead to terribly 
overfit trees. Instead, we confine ourselves with simpler, rougher trees that will make errors on 
training samples, but generalize better. 
In the algorithm further below, we will only need to calculate an error for the samples inside a 
rectangle set $R \subset \RR^p$. Let $n(R) = |\{ 1 \leq i \leq n: x_i \in R \}|$ be the number of 
samples in $R$ and
\begin{align}
    \hat{p}_R = \frac{|\{ 1 \leq i \leq n: x_i \in R, y_i = 1 \}|}{n(R)}
\end{align}
be the proportion of samples in $R$ with positive outcome. The error measure -- in the context of 
trees usually termed impurity measure -- we use in this thesis
is the Gini impurity $Q(R) = 2 \hat{p}_R (1 - \hat{p}_R)$; it is low for pure $R$ dominated by 
samples with either positive or negative outcome and grows quadratically as the outcome of the 
samples in $R$ gets more and more imbalanced. The impurity measure is a hyperparameter: other 
options include the misclassification error and the cross entropy. To govern the complexity $T$,
we demand that every rectangle set $R_m$ contain at least $n_\text{min}$ samples. Even this 
constraint still leaves us with a computationally infeasible number of possible trees, and this is 
why we fit $T$ with a greedy algorithm: it recursively partitions every rectangle set along a 
feature in two subsets in such a way that the impurity measure gets minimal until every resulting 
rectangle set comprises less than $n_\text{min}$ samples. Formalizing binary partitioning, for a 
rectangle set $R$, a feature $j \in \{1, \ldots, p\}$ and split point $s \in \RR$, we define 
the pair of half planes
\begin{align}
    R^{(1)}_{j, s} = \{x \in R: x_j \leq s\} \text{ and } R^{(2)}_{j, s} = \{x \in R: x_j > s\}.
\end{align}
With respect to the impurity measure, there are only finitely many split points of interest since 
we only have finitely many samples, and hence we need to consider no more than $n \cdot p$ when 
considering $R$ for partioning. All of this allows us to compactly describe a procedure to grow a 
tree in Alg.\ \ref{alg:tree}.
  \begin{algorithm}
    \caption{Growing a tree by recursive binary partitioning} \label{alg:tree}
    \begin{algorithmic}[1]
      \Function{Tree}{$(\mathbf{x}_i, y_i)_{i = 1, \ldots, n}$; $Q$, $n_\text{min}$}
        \State $\mathcal{R} \gets \emptyset$ \Comment{Final partioning}
        \State $\mathcal{I} \gets \{ \RR \}$ \Comment{Intermediate partioning}
        \While{$\mathcal{I} \neq \emptyset$}
          \State Choose some $R \in \mathcal{I}$
          \State $\mathcal{I} \gets \mathcal{I} \setminus \{ R \}$
          \State $(j_*, s_*) \gets \argmin_{j, s} n\left(R_{j,s}^{(1)}\right) Q\left(R_{j,s}^{(1)}\right) + 
            n\left(R_{j,s}^{(2)}\right) Q\left(R_{j,s}^{(2)}\right)$ \label{alg:tree:split}
          \For{$\ell = 1, 2$}
            \If{$n\left(R_{j_*, s_*}^{(\ell)}\right) < n_{\text{min}}$}
              \State $\mathcal{R} \gets \mathcal{R} \cup \left\{ R_{j_*, s_*}^{(\ell)} \right\}$
              \State $c_R \gets 1$ if $\hat{p}_R > \num{0.5}$ else \num{-1}
            \Else
              \State $\mathcal{I} \gets \mathcal{I} \cup \left\{ R_{j_*, s_*}^{(\ell)} \right\}$
            \EndIf
          \EndFor
        \EndWhile
        \State \Return $\sum_{R \in \mathcal{R}} c_R \chi_R$
      \EndFunction
    \end{algorithmic}
  \end{algorithm}

\paragraph{Forests}
Trees are known to be notoriously noisy, meaning if we fix $x \in \RR^p$ the variance 
$V_\mathbf{z}(T(x))$ is quite high; the index $\mathbf{z}$ makes explicit that we consider $T(x)$ as 
a function -- or better: random variable -- of i.i.d.\ drawing training samples $\mathbf{z}$ (and 
then fitting a tree to them according to Alg.\ \ref{alg:tree}). On the other hand, if grown deep,
trees have pretty low bias, meaning for fixed $x \in \RR^p$ the expectation $E_\mathbf{z}(T(x))$ 
is close to $E(Y \cond x)$. High variance and low bias makes trees 
ideal candidates for a method called bagging, which averages the predictions of many noisy, 
approximately unbiased models with the goal to reduce their variance. Given trees $T_b$, $b = 1, 
\ldots, B$, fit to identically distributed training data $\mathbf{z}_b$, we denote the new model by 
\begin{align}
    \overline{T} = \frac{1}{B} \sum_{b=1}^B T_b.
\end{align}

We again fix some $x \in \RR^p$. Since the expected value is linear and the $T_b(x)$ are identically 
distributed, we have $E_\mathbf{z}(\overline{T(x)}) = E_\mathbf{z}(T_1)$ and we see that bagging 
leaves the bias unchanged. Consequently, our hope rests on reducing the variance. Say, every 
$T_b(x)$ -- again understood as a random variable dependent on sampling the training data -- has 
variance $\sigma^2$. If the $T_b(x)$ are independent, the variance is additive and we have 
$V_\mathbf{z}(\overline{T}(x)) = \sigma^2/B$, which tends to zero as $B \to \infty$. In practice, 
the $T_b(x)$ are not totally independent as they have training samples in common: we do not i.i.d.\ 
sample from the whole population $(X, Y)$, but draw a bootstrap sample of size $n$ from our training 
data $(x_i, y_i)$, $i = 1, \ldots, n$. Instead providing every 
tree its own separate, tiny, but independent set of training samples would seriously increase the 
bias. Bootstrapping -- uniformly sampling without replacement -- is the key concept behind bagging, 
explaining why \textit{bootstrap aggregation} is a synonym for it. If in this 
situation the $T_b(x)$ have a pairwise correlation of $\rho > 0$, we have 
\begin{align}
    V_\mathbf{z}\left(\overline{T}(x)\right) = \rho \sigma^2 + \frac{1-\rho}{B} \sigma^2,
\end{align}
which tends to $\rho \sigma^2$ as $B \to \infty$, cf. \cite[Eq. (15.1)]{elem-stat-learn01}.

Thus, in addition to bagging, random forests strive to reduce to correlation between the trees 
without increasing the variance and bias too much. We achieve this by randomly selecting a subset
of $m \leq p$ features every time we split a rectangle into two sub-rectangles when growing the 
tree in Alg.\ \ref{alg:tree}. In more detail, the indices $j$ over which we take the minimum in 
line \ref{alg:tree:split} only range over 
these $m$ features. We denote the modified function with the appended parameter $m$ by 
\textsc{RfTree}. Alg.\ \ref{alg:forest} applies these ideas all together.

  \begin{algorithm}
    \caption{Growing a forest with bagging and random feature selection} \label{alg:forest}
    \begin{algorithmic}[1]
      \Function{RandomForest}{$(x_i, y_i)_{i = 1, \ldots, n}$; $B$, $m$, $Q$, $n_\text{min}$}
        \State $\mathcal{E} <- \emptyset$ \Comment{Ensemble to be filled} 
        \For{$b = 1, \ldots, B$}
          \State Draw a bootstrap sample $(\tilde{x}_i, \tilde{y}_i)_{i = 1, \ldots, n}$ from 
            $(x_i, y_i)_{i = 1, \ldots, n}$
          \State $T \gets$ \Call{RfTree}{$(\tilde{x}_i, \tilde{y}_i)_{i = 1, \ldots, n}$; $Q$, 
            $n_\text{min}$, $m$}
          \State $\mathcal{E} \gets \mathcal{E} \cup \{T\}$
        \EndFor
        \State \Return $\sgn \left( \frac{1}{B} \sum_{T \in \mathcal{E}} T \right)$
      \EndFunction
    \end{algorithmic}
  \end{algorithm}

Widely used default values are $m = \floor{\sqrt{p}}$ and $n_\text{min} = 1$. As for $B$, we have 
shown above that increasing it reduces the variance at constant bias; in practice, therefore, only 
computation time limits the number of trees in the forest. We can go for a high $B$ as fitting the 
constituent trees is an embarrassingly parallel problem. We can 
also validate random forests very efficiently by facilitating \textit{out-of-bag (OOB)} samples: 
For each sample $(x_i, y_i)$, we can construct its random forest predictor by averaging only those 
trees for which $(x_i, y_i)$ was not part of the bootstrapped training sample -- no need for a time 
consuming cross validation. The predicition of this random forest predictor for $x_i$ is the OOB 
prediction for sample $i$. When we say validated prediction for a random forest, we mean the 
OOB prediction.

\subsection{Nested models}\label{subsec:nested-models}

Given some \textit{early} models -- e.g. core models -- $f_i: \RR^p \to \RR$, $i = 1, \ldots, m$, 
we can nest them into another, \textit{late} model $f: \RR^m \to \RR$ and obtain a new model 
$g = f \circ (f_1, \ldots, f_m)$. 

\subsubsection{Early models trained on another data set}
Often, the early models have been trained on another, independent data set, so we observe their 
output as features in our data set. Examples are the cell-of-origin signature, the IPI score or the 
LAMIS. Such $f_i$ are merely projections onto a feature.

\subsubsection{Early models trained on the same data set}
If we need to fit some of the early models to the training data ourselves, getting reliable 
cross-validated predictions
for $f$ becomes trickier. Without loss of generality, we assume the only early model left to be fit 
is $f_1$. We want to fit $f_1$ according to a hyperparameter tuple $h_1$ and $f$ according to $h_2$, 
giving us the hyperparameter tuple $h \in H$ for the nested model. How do 
we get realistic cross-validated predictions $\hat{y}_i$, $i \in F_\ell$, for the $\ell$-th fold? 
We start by fitting $f_1$ to $(x_i, y_i)$, $i \in \{ 1, \ldots, n \} \setminus F_\ell = F_\ell^c$ 
subject to $h_1$.
The question reduces to: which values for the output of $f_1$, denoted $o$, do we provide the 
algorithm fitting $f$ to $(o_i, f_2(x_i), \dots, f_m(x_i); y_i)$, $i \in F_\ell^c$?

The choice $o_i = f_1(x_i)$ is most likely too optimistic since $f_1$ overfits its training 
data to some extent. Consequently, $f$ will put too much trust in them and will generalize poorly, 
as the cross-validated predictions of $f$ will already show. We want to tackle this problem by 
using cross-validated predictions for $f_1$, instead. The theoretically cleanest approach 
is to do an \textit{inner} cross validation for $f_1$, i.e. another cross-validation for fitting 
$f_1$ to $(x_i, y_i)$, $i \in F_\ell^c$, and use the cross-validated predictions from \textit{this} 
cross validation for $o$. However, this increases time complexity by a factor of $k$ for a $k$-fold 
cross-validation. Because we fit $f_1$ to $(x_i, y_i)$, $i = 1, \ldots, n$, in a cross validation 
anyway, we prefer to use the cross-validated predictions we get in this process for $o$. 
Alg.\ \ref{alg:nested-pcv} writes this down clearly.

\begin{algorithm}
    \begin{algorithmic}[1]
        \Function{NestedPseudoCV}{$(x_i, y_i)_{i = 1, \ldots, n}$; $h = (h_1, h_2)$, $k$}
            \State Fit $f_1$ to $(x_i; y_i)$, $i = 1, \ldots, n$, subject to $h_1$ , yielding 
                validated predictions $o$.
            \State Fit $f$ to $(o_i, f_2(x_i), \ldots, f_m(x_i); y_i)$, 
                $i = 1, \ldots, n$, subject to $h_2$, yielding validated predictions $\hat{y}$.
            \State $g \gets f \circ (f_1, \ldots, f_n)$
            \State \Return $(\hat{y}, g)$
        \EndFunction
    \end{algorithmic}
    \caption{Nested pseudo cross validation.} \label{alg:nested-pcv}
\end{algorithm}

The gain in time complexity compared to the approach involving an inner cross validation comes at a 
little price: a not thoroughly clean cross validation for $g$. To see this, we assume that the 
assignment of samples into folds in the cross validation for $f_1$ and $f$ is the same -- a very 
realistic assumption as we 
usually choose $k = n$. We fix a sample $i$ and another sample $j$ that is in another fold. 
Sample $i$'s cross-validated prediction $\hat{y}_i$ for $g$ -- or equivalently its cross-validated 
prediction for $f$ -- depends on the cross-validated 
prediction $o_j$ of sample $j$ for $f_1$. $o_j$ in turn depends on sample 
$i$. All in all, $\hat{y}_i$ very slighly depends on $(x_i, y_i)$.

Next, we want to talk about how we tune the hyperparameters of a nested model. In practice, one thinks of 
the hyperparameter tuples $H$ as partitioned into subsets. We summarize the
hyperparameters of a model that make it most distinct to other models in its \textit{architecure}. 
For a core model, the architecture concerns its class (like logistic model or random forest) and 
the a priori selected features. For a nested model, its architecure is defined by the architectures 
of the involved core models and the hierarchy that nests them into one another.

For a given architecture of a nested model, we need to tune the hyperparameter tuple $h = (h_1, h_2)$ 
of 
Alg.\ \ref{alg:nested-pcv}, i.e., we repeately call \textsc{NestedPseudoCV} as we vary $h \in H_0$
with $H_0$ being the set of tried out hyperparameter tuples for this model architecture. What 
should $H$ look like? We have candidate 
hyperparameter tuples $H_{0, 1}$ for $f_1$ and $H_{0, 2}$ for $f$. An unprejudiced strategy is to 
set $H = H_{0, 1} \times H_{0, 2}$. Compared to non-nested models, we try out a considerably 
higher number of hyperparamter tuples -- roughly a factor of $|H_{0, 1}| \approx |H_{0, 2}|$ and 
the validated errors of nested models have a higher chance 
of being overestimated, cf.\ subsection \ref{subsec:train-val}. We therefore settle on a greedier 
strategy: we first tune $h_1 \in H_{0, 1}$ and then go on to tune $h_2 \in H_{0, 2}$. 

$f_1$ often deals with high-dimensional input. E.g., $f_1$ is a Gauss, logistic or Cox model 
predicting from 
several thousand RNA-seq gene-expression levels. Hence, we undoubtedly need elastic-net 
regularization in its training and need to tune the regualarization strength $\lambda$. 
With this alone, we quickly end up with several hundred hyperparameters to be tuned for $f_1$. 
With $H_{0,1}$ being so big, good validated performances have one thing in common: they are 
probably too optimistic, the cross-validated predictions are too accurate. This means we still 
have not gotten entirely rid of the problem that $f$ puts too much trust into $f_1$'s output 
misled by too optimistic values in the training. This problem will propagate into another
problem: too optimistic cross-validated predictions for $g$ and an overestimated 
cross-validated performance of $g$. It is hard to quantify the effect, so we will have a very close 
eye on models nested according to Alg.\ \ref{alg:nested-pcv} in chapter \ref{chap:results}. 

\subsubsection{Nested models and inter-technical variability}

Considering a nested model $g = f \circ (f_1, \ldots, f_m)$, what happens if the output of some of 
the early models $f_i$ suffers from inter-technical variability? Without loss of generality, let 
only $f_1$ be such an early model. Moreoever, let both $f_1$ and $f$ be a GLM, which we describe 
with the linear predictor as $f_1: \RR^p \to \RR, x \mapsto x^T \beta$ and $f: \RR^m \to \RR, 
x \mapsto x^T \gamma$. Let $\beta$ fulfill the zero-sum constraint, i.e. $\sum_{j=1}^p \beta_j = 0$.
As before, $\mathbf{z} \in \RR^{n \times p}$ holds the same features for the same samples as the 
predictor matrix $\mathbf{x}$, but measured under a different protocol.
We assume that Eq.\ \ref{eq:inter-tech-exact} holds with small residues $\epsilon_{ij}$. In this 
thesis, $f_1$ will always be a gene-expression signature, which makes this a realistic assumption. 
With Eq.\ \ref{eq:inter-tech}, we obtain
\begin{align}\label{eq:inter-tech-nested}
\begin{split}
    g(z_i) &= (f \circ (f_1, \ldots, f_m))(z_i) \\ 
    &= (z_i^T \beta, f_2(z_i), \ldots, f_m(z_i))^T \gamma \\
    &\approx (x_i^T \beta + c, f_2(x_i), \ldots, f_m(x_i))^T \gamma \\ 
    &= c \gamma_1 + (f \circ (f_1, \ldots, f_m))(x_i) = c \gamma_1 + g(x_i)
\end{split}
\end{align}
for some $c \in \RR$ and all samples $i = 1, \ldots, n$. The sample-wise shift in the output of 
$f_1$ propagates into a sample-wise shift in the output of $g$ -- qualitatively this is the same 
as in Eq.\ \ref{eq:inter-tech}. The residuals $\epsilon_{ij}$, $\beta$ and $\gamma$ impact the 
above approximation. From our 
perspective, where output ordering matters most, we can say: the better this approximation, the more 
monotonic the $g(z_i)$ are in $g(x_i)$. We do not elaborate on this theoretically any further; 
in our practice in section \ref{sec:inter-trial}, we will see that the approximation is indeed
very good.

While $f_1$ is always a GLM in this thesis, one might consider a random forest for $f$. Random 
forests, which predict solely by thresholding single features, are highly sensitive to shifts in 
the output of $f_1$. In this situation, we should rather make $f_1$ agnostic to inter-technical 
variability once and 
for all by thresholding its output on the respective batch of samples, but this risks losing the 
nuanced information a continous output provides. These considerations turn the simplicity of GLMs 
into an asset against more complicated models: for GLMs we can better foresee how they will react 
to inter-technical variability.

\section{Software}

We have designed the software that generates the results of this thesis in such a way that it 
can be the foundation of the machine-learning part of MMML-Predict as the project develops and 
other researchers take over. Beyond this, we have written code, wherever it was possible, in such 
a general way that one can apply it not just for the MMML-Predict problem, but for all equivalent 
and closely related problems.

\subsection{The R package \texttt{patroklos}}
The R package \texttt{patroklos} holds all of the highly reusable code of this thesis. It brings 
all the methods described in this chapter to life and is available on GitHub \cite{patroklos-gh} 
together with a 
\href{https://lgessl.github.io/patroklos/}{website}\footnote{\url{https://lgessl.github.io/patroklos}}. 
In fact, one can deploy 
\texttt{patroklos} to solve any binary classification problem in a supervised-learning manner.
\texttt{patroklos} provides some extra functionality for those problems among them where the 
reponse is a thresholded survival time, high-dimensional data is involved and one wants to maximize 
the precision of the classifier for a sufficently large positive cohort.

\paragraph{Training} 
\texttt{patroklos} allows the user to quickly specify model architectures and 
attach hyperparamters tuples to them, including model-agnostic hyperparameters to all of them. It 
provides a decorator to transform a function that fits merely one model for a single hyperparameter 
tuple into a function that fits and validates models for a variety of hyperparameter tuples. 
It supports fitting core models with the support of the R packages \texttt{zeroSum} 
\cite{zerosumR} and \texttt{ranger} \cite{ranger-gh} out of the box. \texttt{zeroSum} 
augments the popular \texttt{glmnet} package by the zero-sum constraint: it fits and 
tunes Gauss, logistic and Cox models in a cross validation and endows their loss function 
with elastic-net regularization and -- unlike \texttt{glmnet} -- the zero-sum 
constraint. The time-critical work happens in a fast, well-structured C++ backbone. 
\texttt{ranger} fits random forests fast, also thanks to a performant C++ backbone. 

\paragraph{Validation and testing} 
If fitting functions from existing R packages calculate a validated error of 
their fit models at all, they do so in a plethora of ways, usually using an error measure derived 
from the loss function they minimize; e.g., \texttt{zeroSum} uses the binomial deviance for 
logistic models and \texttt{ranger} uses the classicatication error for classification random 
forests. \texttt{patroklos} takes 
full control of validation and unifies it
by calculating the same error measure across all models and even more statistics of the model
with the help of the \texttt{AssScalar} R6 class. With the \texttt{Ass2d} R6 class, the 
user can plot one property of a truly binary classifier against another, which guides 
thresholding models with continuous output as in Fig.\ \ref{fig:inter-output-prec}.

\paragraph{Modular, extendible design} 
\texttt{patroklos} defines function interfaces so the user 
can easily wrap fitting functions implemented in other R packages or by the user to make 
them \texttt{patroklos}-compliant. The two R6 classes \texttt{Model} and \texttt{Data}
strive to abstract models from the data in such a way that the user can easily apply $H$ to new 
data.

\paragraph{Meta analysis} 
\texttt{patroklos} provides tools to analyze how the validated error depends on the test error and 
allows to highlight subgroups of model architectures to disclose fundamental flaws in validation 
and test performance.

\subsection{\texttt{patroklos} in action}

The Git repository that brings \texttt{patroklos} into action for this thesis is available in a 
public version on GitHub \cite{thesis-gh}. We try our best at presenting the results in a 
comprehensive and understandable way in the following chapter \ref{chap:results}. Yet, since code 
is more precise than freely written text, the Git repository rigorously defines everything we have 
done to obtain the results.

The public Git repository does not contain the data because part of it must not be publicly 
available. To reproduce the results of this thesis, the reader may request access to the scientific 
computing servers of the Chair of Statistical Bioinformatics, University of Regensburg, from 
\href{mailto:christian.kohler@ur.de}{Christian Kohler}\footnote{Email: 
\href{mailto:christian.kohler@ur.de}{\texttt{christian.kohler@ur.de}}}; on these servers, a mounted
volume holds all data sets.
    \chapter{Results} \label{chap:results}

In this triune chapter, we start with introducing three DLBCL data sets that include 
survival and gene expression features. Next, in intra-trial experiments, we split every of these 
three data sets into 
a train and test cohort, fit models for a variety of hyperparameter tuples in $H$ to the training 
cohort, validate them and test the best on the test cohort; this is less about presenting a 
high-performing model, but more about analyzing validated and tested errors to make $H$ slimmer and 
better for the future. This future plays out in the last part as we train and validate on one of 
the three data sets and test on another and deal with cross-platform variability in inter-trial 
experiments.

\section{The data}

Our DLBCL data sets are taken from papers by Schmitz et al.\ \cite{schmitz18}, Reddy et al.\ 
\cite{reddy17} and Staiger et al.\ \cite{staiger20}; see Table \ref{table:data} for key properties 
and comparison. We will refer to them as Schmitz, Reddy and Lamis test data, respectively.

\begin{table}
    \centering
    \begin{tabular}{lrrr}
        \hline
        & \textbf{Schmitz} & \textbf{Reddy} & \textbf{Staiger} \\
        \hline
        \textbf{prospective trial} & no & no & yes \\
        \textbf{response} & PFS & OS & PFS \\
        \textbf{\# samples} & \num{229} & \num{604} & \num{466} \\
        \textbf{\# genes with expression level} & \num{25066} & \num{13302} & \num{145} \\
        \textbf{technology} & RNA-seq & RNA-seq & NanoString \\
        \textbf{high risk [\%]} & \num{36.6} & \num{31.5} & \num{24.3} \\
        \textbf{$\text{prev}(\text{tIPI})$ [\%]} & \num{12.9} & \num{21.6} & \num{17.0} \\
        \textbf{$\text{prec}(\text{tIPI})$ [\%]} & \num{65.2} & \num{54.1} & \num{38.2} \\
        \hline
    \end{tabular}
    \caption{Overview on used data sets. All datasets include the five IPI features in their 
        thresholded format, the IPI score and group, gender, cell of origin thresholded into 
        ABC-like, GCB-like, unclassified, and, added 
        by us, the LAMIS score and group. For the Schmitz and Staiger data, we use the high-risk 
        definition of 
        MMML-Predict: PFS below two years. Because the Reddy data reports only overall survival 
        (OS), no PFS, we define high risk as overall survival below \num{2.5} years there, leading 
        to a comparable high-risk proportion. ``Technology'' refers to the technology used to measure 
        gene-expression levels.}
        \label{table:data}
\end{table}

\paragraph{Schmitz data}
The data by Schmitz et al.\ includes the five IPI features in their continuous form. 
By heuristically optimizing a novel genetic distinctiveness metric, Schmitz et al.\ 
clustered 574 DLBCL biopsy samples into four subtypes, called MCD, BN2, N1, EZB and ``other''. They 
unblinded the clinical data only after the clustering was complete and the 
model was frozen \cite[Appendix 1, pp. 16--18]{schmitz18}. The following, independent survival analysis 
unveiled significantly differing progression-free survival between the four subtypes (excluding 
``other'') according to a logrank test; BN2 and EZB subtypes have far better prognosis than MCD and 
N1 (2-year PFS rate of \num{81}\% and \num{75}\% as opposed to \num{39}\% and 
\num{20}\%, respectively). The IPI score did not vary significantly between the subtypes, 
indicating that the new classifier gives us additional information to predict survival.

There are two caveats: First, the genetic classifier saw the entire data set during training and 
this runs afoul of a strict train-test regime even if training was 
survival-agnostic. This only affects the intra-trial experiments as in the inter-trial experiments 
we do not include the genetic subtype as a feature. As a more important take away, we always need 
to carefully use features in a data set that are the output of some model trained on this data set. 
Second, this data set is not the result of a 
prospective, representative trial, but opportunistically collected samples from highly renowned 
U.S. hospitals, which preferably treat 
difficult cases. As a result, the high-risk proportion in the Schmitz data is at \num{36.6}\% -- 
compared to \num{24.3}\% in the prospective Lamis test data -- and the IPI reaches a precision of 
\num{65.2}\% for classifying high-risk patients at a prevalence of \num{12.9}\% -- compared to 
\num{38.2}\% at \num{17.0}\% in the Lamis test data. The IPI already meets the MMML-Predict goals, 
but we want to see if we can do even better in a such high-risk regime.

\paragraph{Reddy data}

Compared to a prospective study, also the Reddy data is enriched for high-risk patients and the 
IPI boasts a performance that already satisfies the MMML-Predict goals, even if a bit less 
convincingly than on the Schmitz data. After identifying \num{150} DLBCL driver genes, Reddy et al.\ 
trained a Cox model, termed genomic risk model,
that predicts overall survival (OS) from combinations of genetic events and gene-expression markers 
(cell of origin, MYC, and BCL2) and thresholded it into low, intermediate and low risk. This is a
model whose predictions we cannot use because their split into the train and test cohort is not 
clear, but that inspired us to use combinations of discrete features, cf.\ subsection 
\ref{subsec:model-agnostic}. Nevertheless, with high expression and translocation of MYC, BCL2 and 
BCL6, the data provides some of the input features of the genomic risk model as well as three binary 
clinical features: B symptoms at diagnosis, testicular and central-nervous-system involvement. B 
symptoms refer to the presence of the triad fever, night sweats and unintential weight 
loss.

\paragraph{Lamis test data}

The Lamis test data is composed of \num{466} patients enrolled in prospective clinical trials. 
Staiger et al.\ first determined \num{731} gene pairs with highly correlated gene expression levels 
between their train cohort -- \num{233} DLBCLs with gene expression levels built from the Affymetrix 
GeneChip technology -- and the Lamis test data -- \num{466} DLBCLs with gene expression levels built 
from the NanoString nCounter technology -- with the aid of six paired nCounter-GeneChip samples, cf.\ 
\cite[Supplementary Methods]{staiger20}.
Next, they learned a Cox model on the differences of the (logarithmized) gene expression levels 
from these gene pairs and the five thresholded IPI features using LASSO regularization. 
Afterwards, they removed the five IPI features from the model aiming to make it independent of the 
IPI. One can expand the differences of gene expression levels in the signature to obtain an ordinary 
gene-expression signature with coefficients corresponding to single genes, the Lamis. It is 
based on \num{17} genes, but dominated by just two genes, CSF1 and CPT1A. 

By dichotomizing the 
Lamis scores at the 75\%-quantile into their Lamis group (low or high), Staiger et al.\ present two 
groups on the Lamis test data with significantly differing PFS and OS; 
meanwhile, the IPI features, breaks in MYC, BCL2, BCL6, and cell of origin remained prognostic 
indicators independently of the Lamis group.

Since the Lamis coefficients fulfill the 
zero-sum property, we apply the Lamis unchanged on the two other data sets; according to Eq.\ 
\eqref{eq:inter-tech}, the Lamis scores have a data-set-dependent shift. We also add the Lamis 
group by dichotomizing the Lamis scores at the 75\%-quantile of the respective data set.

\section{Intra-trial experiments}

To gain first insights on our methods, we conduct intra-trial experiments sepratetly on the 
Schmitz, Reddy and Lamis test data. To this end, we split every data set into a train and test 
cohort. We do so uniformly at random, with two constraints: first, a ratio of 3 to 1 between train 
and test cohort and, second, the ratio between high-risk and low-risk patients in train cohort, 
test cohort and overall data set is the same. As shown in Table \ref{table:intra-trial}, the 
performance of the IPI score thresholded at 4 to classify high risk notably differ between the whole
data set and the subsampled test cohort. 

\begin{table}
    \centering
    \begin{tabular}{lrrr}
        \hline
         & \textbf{Schmitz} & \textbf{Reddy} & \textbf{Staiger} \\
        \hline
        \textbf{\# samples} & \num{58} & \num{151} & \num{117} \\
        \textbf{high risk [\%]} & \num{37.0} & \num{31.6} & \num{24.3} \\
        \textbf{$\text{prev}(\text{tIPI})$, $\text{prec}(\text{tIPI})$ [\%]} & \num{17.0}, 
            \num{50.0} & \num{19.2}, \num{42.1} & \num{39.2}, \num{38.7} \\
            \textbf{$\text{prev}(\mtc{i})$, $\text{prec}(\mtc{i})$ [\%]} & \num{35.1}, \num{68.4} & 
            \num{23.0}, \num{55.6} & \num{34.6}, \num{45.9} \\
        \textbf{precision \num{95}\%-CI $\mtc{i}$ [\%]} & $[\num{43.5}, \num{87.4}]$ & 
            $[\num{35.3}, \num{74.5}]$ & $[\num{29.5}, \num{63.1}]$ \\
        \textbf{HR $\mtc{i}$} & \num{1.00} & \num{1.04} & \num{13.6} \\
        \textbf{HR \num{95}\%-CI $\mtc{i}$} & $[\num{1.00}, \num{1.00}]$ & 
            $[\num{1.02}, \num{1.06}]$ & $[\num{1.99}, \num{93.2}]$ \\
        \textbf{logrank p $\mtc{i}$} & \num[scientific-notation=true]{3.69e-4} & 
            \num[scientific-notation=true]{1.82e-3} & 
            \num[scientific-notation=true]{9.38e-4} \\
        \hline
    \end{tabular}
    \caption{Statistics of intra-trial experiments on the respective test cohort. $\mtc{i}$ denotes the 
        model with minimal validation error on the respective training cohort.}
    \label{table:intra-trial}
\end{table}

\subsection{Model architectures}

We want to give a brief summary of the models we send into the race and take a closer look at the 
best one, $m^*$, on every data set.

\subsubsection{Candidates}

\paragraph{Gene expression levels only}
Models trained in a leave-one-out cross-validation only on the gene expression levels include the 
Gaussian, logistic and Cox model. Regarding noteworthy hyperparameter decisions, we both apply and 
do not apply standardization of the predictor; we do not demand the zero-sum constraint 
because we do not want to transfer our model to other data sets and want to safe computation time;
we regularize with elastic-net penalty factor $\alpha \in \{ \num{0.1}, 1 \}$; as troughout this 
chapter, we have \texttt{zeroSum} calculate a sequence of \num{100} regularization strengths 
$\lambda$ for us and stop early if the cross-validated error does not improve for \num{10} 
consecutive decreasing $\lambda$ values; we try out a whole bunch of time cutoffs $T$.

We train and validate a model for every combination of these hyperparameters.
Not taking into account the values of $\lambda$ (which are hard to foresee due to early stopping), 
this adds up to \num{88} models.

\paragraph{Core models with other features}
We use the validated performance of the above models to greedily restrict $H$: by looking at the 
top performing models, we restrict the number model classes (Gaussian, logistic, Cox) to just one 
or two; we restrict $T$ to one or two values, usually $\infty$ for Cox models and the value 
of high-risk definition for Gaussian and logistic models; by choosing $\alpha = 1$, we opt for 
LASSO regularization and its sparse models. 

We now add all available remaining features: for the 
IPI, we either add the five continuous features (if available), the five threholded features, the 
score as a continuous feature or all of them; for the Lamis, here and for the rest of the thesis we 
add the score as a continuous and the group as discrete features. For $s_\text{min} = \num{0.05}$ and 
$n_\text{combi} \in \{1, 2, 3, 4 \}$, we add combinations of discrete features to the predictor. 
Additionally, we train these models without gene expression levels, in which case we also use the 
random forest as a model class.

\subsection{}

\section{Inter-trial experiments}

\begin{table}[ht]
    \small
    \centering
    \begin{tabular}{lrrr}
        \hline
        & \textbf{Schmitz} & \textbf{Reddy} & \textbf{Lamis test} \\
        \hline
        \textbf{Schmitz} & (\num{12.9}/\num{65.2}) & (\num{17.7}/\num{59.6}) & (\num{17.1}/\num{50.7}) \\
        \textbf{Reddy} & (\num{17.8}, \num{71.1}) & (\num{21.6}/\num{54.1}) & (\num{18.0}/\num{53.2}) \\
        \textbf{Lamis test} & (\num{17.4}/\num{75.7}) & (\num{22.5}/\num{50.4}) & (\num{17.0}/\num{38.2}) \\
        \hline
    \end{tabular}
    \caption{Rows $i$ hold training cohorts, columns $j$ hold test cohorts. Diagonal 
        entries $(i, i)$ hold (prevalence/precision) of $\text{IPI} \geq 4$
        on cohort $i$. Off-diagonal entries $(i, j)$ hold (prevalence/precision) 
        on cohort $j$ of the best validated model $m^*_i$ trained on cohort $i$.}
    \label{tab:inter_trial}
\end{table}
    \chapter{Discussion} \label{chap:discussion}

The previous chapter showed that our best models can significantly outperform the IPI in both 
a high-risk and representative-risk setting. Moreover, we demonstrated that we can reliably validate 
our trained models and pick a near-optimal if not optimal one if we restrict the methods from 
chapter \ref{chap:methods} appropriately and increase the sample size.

\section{Heuristics for candidate models}

Key to ensuring realistic validated errors is excluding those candidate models from $H$ that sneak 
in a validated error considerably below their test error. In this thesis, we developed a series 
of heuristics that aim to do precisely that.

\paragraph{Nesting models}
We have seen that we should nest models predicting from hundreds or even thousands of 
gene-expression levels together with more features into another model with care. Our way to do 
this, which uses the cross-validated predictions of the early model to train the late model, 
delivers way too optimistic validated errors for the late model. The early model, usually involving 
much more features than training samples, has plenty of freedom to overfit and it will exploit this 
freedom when we tune its hyperparameters in a cross-validation. The algorithm of the late model then 
recognizes the output of the early model as a very predictive feature and is misled to make it 
a prominent feature in the final model. On independent test data, the input for the late model 
systematically differs from the training, causing it to generalize poorly.

Instead, we need to present the late model training data identically distributed to the test 
data. An easy and successful way to this is to not train and tune the early model ourselves, but have 
other people do this on an independent data set. In other words, we use already-existent signatures. 
Specifically, we did this with cell of origin and the LAMIS. The LAMIS fulfills the 
zero-sum constraint such that transferring it to a new data set with gene-expression levels 
measured under another protocol only results in a constant shift. When using the output of such 
a signature as a feature for one of our models, we have two options regarding its format.
First, we can threshold the signature -- as with cell of origin into ABC and GCB or the LAMIS into 
a high and low group --, thereby rendering the signature protocol-independent and ready to input 
into any late model; this, of course, also works with violated zero-sum constraint and 
$m^*_\text{Schmitz}$ from section \ref{sec:inter-trial} 
does this quite successfully. Second, we can leave the signature output untouched and provide it as a 
continuous feature to a GLM, for whose linear predictor a constant shift in an input feature results 
in constant shift of its output. The most convincing model of this thesis, $m^*_\text{Reddy}$ does 
this with the LAMIS score.

\paragraph{Gene-expression levels}
As a result, we refrain from incorporating gene-expression levels directly into the models trained 
by us, but only use their valuable information condensed into the output of already-existent models. 
With the curse of high dimensions gone, we have observed much more trustworthy validated errors. 
Furthermore, this reduces the training time and enables us to use more complex models.

\paragraph{More heuristics}

The inter-trial experiments suggest that model performance benefits from using a lower training 
survival cutoff $T$. Random forests, whose OOB predictions proved to yield a very accurate estimate 
of the test error, fare worse than GLMs; unlike GLMs, they cannot deal with systemically shifted 
features, which makes them unsuited to handle the continuous output of gene-expression signatures
across protocols.

\section{Increasing the sample size}

In addition to restricting the hyperparameter-tuple space $H$ carefully, we raised the number of 
samples by combining three data sets. We hoped to get both more representative train and test 
cohorts with less systematic differences and hence turn validated errors into better estimates 
for test errors. Whatever the reason, in the latter we succeeded. To combine the data, we had to 
make sacrifices: we discarded the features that are 
not present in all three data sets -- and this included valuable features like MYC translocations -- 
and we often trained models to predict PFS and then tested them on OS or vice versa. Still, our 
best validated models defied this as well as systematic, especially technological and prognostic 
differences between the data sets and, at roughly \num{15}\% prevalence, outperformed the IPI 
significantly on the prospective Staiger data.

In the inter-trial experiments, with more samples in the test cohort, we could lower the prevalence 
of our selected models to \num{10}\%, drastically raise the precision and still retain enough 
statistical power to significantly defeat the IPI in even more cases. For the best validated models, 
we have seen an overwhelmingly monotonic relationship between the prevalence and the precision, 
which made the \num{15}\%-quantile or -- even better -- \num{10}\%-quantile of the model output a 
near-optimal threshold. 

The results of this thesis strongly suggest that sample size is a crucial factor 
for our problem. Furthermore, we should be ready to make sacrifices -- less features, 
even a different kind of the response -- to increase the number of samples in both train and test 
cohort. 

\section{The future of MMML-Predict}\label{subsec:discussion-mmml}

With $m^*_\text{Schmitz}$ and even more $m^*_\text{Reddy}$, this thesis presents two 
models that meet all requirements of MMML-Predict: a prevalence of at least \num{10}\% and a 
precision above \num{50}\% and significantly above the IPI on an independent, prospective data set. 

And still, every percentage point we gain in prevalence will convince more clinicians, researchers 
and pharmaceutical companies to pay attention to the high-risk group identified by the MMML-Predictor.
Every percentage point we gain in precision will not just spur more interest of the above people and 
companies, but will also convince more patients to let the MMML-Predictor guide their treatment and 
will avoid both unnecessary and failed therapies. This thesis did not suggest that model performance 
has already saturated, so we can go for even more in the future.

How might we do this? This thesis demonstrated that we can use data from already-existent trials, 
even if they are not prospective and differ technologically, to train and validate a model and 
then successfully test it on data from a prospective trial as MMML-Predict is about to conduct one.
In the intra-trial experiments, we saw validation and test performance out of touch. We suspect 
that too small, therefore non-representative and systematically differing train and test cohorts 
could have played a key role in this. For the final MMML-Predictor, one might thus consider 
using more than \num{100} of the \num{300} samples registered for MMML-Predict, maybe even all 
\num{300}, for a sufficiently large, statistically more powerful test cohort. One might combine 
a series of available data sets into a big training cohort. With combining data sets come two 
caveats. 

First, we need to intersect over the sets of features, which amounts to discarding plenty 
of features. Nevertheless, many modern DLBCL data sets include gender, age, the five thresholded 
IPI features as clinical features and gene-expression levels as molecular features. More modern data 
sets also hold the double-hit and triple-hit status, which refers to the simultaneous translocation 
of MYC and either BCL2 or BCL6, or MYC, BCL2 and BCL6, respectively. From gene-expression 
features, we can calculate already-existent molecular signatures as well as the double-expressor and 
triple-expressor, which describe the phenomena analogous to double- and triple-hit status with 
overexpression instead of translocation of the three involved genes.

Second, we need to take care of batch effects. Among the features mentioned above, this concerns 
the gene-expression levels. The first option is to get rid of protocol effects by once and for all
thresholding the output of molecular signatures. Most signatures come with canonical thresholds, 
such as the group for the LAMIS, and we can calculate them on the respective data set even before 
combining. In this case, we can deploy quite any model as we no longer need to fight the curse 
of high dimensions or batch effects. The second option is to only apply zero-sum signatures (or 
scale the gene-expression levels to an $\ell_1$ norm of \num{1} across all samples). Using the 
output of these signatures as continuous features of a GLM again results in a protocol-dependent 
shift in output of the GLM. For training, we can add the data set the sample was taken from as a 
categorical 
feature to the predictor so we can correct for the data-set-dependent shifts in the loss function. 
After training, we remove the data-set feature from the model. In testing, 
we threshold the continuous output of the GLM at the $\alpha$-quantile for some $\alpha$, as we did 
in this thesis. A shortcoming of this testing procedure is that we always need a sufficiently large 
test cohort to be able estimate the $\alpha$-quantile reliably. An alternative approach might 
involve internal standards, i.e.\ a small number of samples one has measured for a test cohort for 
which we already know a good threshold and that we can measure again for any new protocol to shift 
the threshold accordingly. Even for MMML-Predict, this is a problem for the distant future.

Our last idea concerns the sample weights in loss functions. Almost every loss function is derived 
from the log likelihood of i.i.d.\ samples and thus sums over the training samples. Therefore, they 
offer to weight every summand with a sample weight, as we have seen in the loss functions of the 
presented GLMs in Eq.\ \eqref{eq:loss-glm-no-lasso} and \eqref{eq:cox-log-lh}. One can even provide 
sample weights for random forests. We have a huge amount of 
freedom in how we choose the sample weights and trying out too many choices risks torpedoing 
validation. We will now describe a single, natural choice. Let $q$ denote the 
proportion of high-risk samples in the training data. In the prospective Staiger data, we had 
$q = \num{24.3}$, which renders our classification problem imbalanced. Setting $w_i = 1/q$ for 
high-risk and $w_i = 1/(1-q)$ for low-risk samples perfectly balances the classification task in 
the sense that the sum of sample weights belonging to high-risk samples equals the sum of sample 
weights belonging to low-risk samples.

Looking back at this thesis, we can say with confidence: the state of MMML-Predict is strong and 
it is getting stronger.
    \bibliography{../lit.bib}

    \newpage
    \section*{Declaration of authorship}
    \begin{german}
        Ich habe die Arbeit selbständig verfasst, keine anderen als die angegebenen Quellen und 
        Hilfsmittel benutzt und bisher keiner anderen Prüfungsbehörde vorgelegt. Außerdem 
        bestätige ich hiermit, dass die vorgelegten Druckexemplare und die vorgelegte elektronische 
        Version der Arbeit identisch sind, dass ich über wissenschaftlich korrektes Arbeiten und 
        Zitieren aufgeklärt wurde und dass ich von den in § 26 Abs.\ 5 vorgesehenen Rechtsfolgen 
        Kenntnis habe. 

        \vspace{1.5cm}
        \noindent Regensburg, \today \hfill \underline{\hspace{6cm}}
    \end{german}

\end{document}
\documentclass[10.5pt, a4paper, oneside]{report}
\usepackage{layout}
\usepackage[margin=102pt, textwidth=390pt]{geometry}

\usepackage{polyglossia}
\setmainlanguage{english}
\setotherlanguage{german}
\usepackage{fontspec}
\setmainfont{TeX Gyre Pagella}
\setsansfont{Fira Sans}[Scale=MatchLowercase]
\setmonofont{Inconsolata}[Scale=MatchLowercase]
\usepackage{sectsty}
\allsectionsfont{\sffamily}

\usepackage{amsmath}
\usepackage{amsfonts}
\usepackage{mathtools}
\DeclareMathOperator*{\argmin}{arg\,min}
\DeclarePairedDelimiter\ceil{\lceil}{\rceil}
\DeclarePairedDelimiter\floor{\lfloor}{\rfloor}
\DeclarePairedDelimiter\nint{\lfloor}{\rceil}
\def\RR{\mathbb{R}}
\def\Pr{\mathrm{Pr}}
\def\NN{\mathbb{N}}
\def\im{\mathrm{im}}
\def\cond{\,|\,}
\def\sgn{\mathrm{sgn}}
\def\lamishigh{$\text{LAMIS}_{\text{high}}$}
\def\lamislow{$\text{LAMIS}_{\text{low}}$}
\def\tbeta{\tilde{\beta}}
\def\hbeta{\hat{\beta}}

\usepackage{algorithm}
\usepackage{algpseudocode}
\usepackage{caption}
\usepackage{subcaption}
\captionsetup[table]{position=bottom}
\captionsetup[subtable]{font=normalsize}

\usepackage{siunitx}

\usepackage{natbib}
\bibliographystyle{abbrvnat}
\setcitestyle{numbers,square}

\usepackage{graphicx}
\usepackage{xcolor}
\usepackage{hyperref}

\begin{document}
    \begin{titlepage}
        \centering
        \vspace*{1cm}
        
        \huge
        \textsf{
        \textbf{Computational detection of a high-risk DLBCL group}
        }
        
        \vspace{1.5cm}
        
        \large
        \textbf{Lukas Geßl}
        
        \vfill
        
        \large
        A thesis presented for the degree of\\
        Master of Science
        
        \vspace{0.8cm}
        
        \begin{tabular}{rl}
            First Supervisor: & Prof.\ Dr.\ Harald Garcke \\
            Second Supervisor: & Prof.\ Dr.\ Rainer Spang \\
        \end{tabular}
        
        \vspace{0.8cm}
        
        Department of Mathematics\\
        University of Regensburg\\
        \vspace{0.8cm}
        \today
        
    \end{titlepage}

    \begin{abstract}
        Chemotherapy with R-CHOP is the standard treatment for diffuse large B-cell lymphoma, 
        the most common type of non-Hodgkin lymphoma, curing about two thirds of 
        patients. Survival for the remaining third with refractory or relapsed disease, however, 
        remains poor. Pharma-sponsored randomized trials in the whole DLBCL population to date have 
        failed 
        to improve R-CHOP. The International Prognostic Index (IPI), the only widely accepted 
        risk-assessment tool for DLBCL and an easy clinical test, fails to identify 
        a high-risk DLBCL subpopulation that is 
        large and precise enough to trigger research and enable clinical trials for new treatments 
        that outperform R-CHOP on this subpopulation. 

        This thesis aims to develop a computational method that identifies DLBCL patients with 
        progression-free survival (PFS) below two years with higher prevalence and significantly 
        higher precision than the IPI and to show this on independent data. It also deals with the 
        question under which circumstances we can do so reliably.

        After introducing into DLBCL, the IPI together with its shortcomings and MMML-Predict, the 
        project that this thesis is part of, in the first chapter, we describe the statistical 
        frame and models as well as our software in the second chapter. The third chapter applies 
        the method to three different data sets and one big data set comprised of these three 
        data sets and 
        shows that we can indeed deliver a model with the desired properties. Analysis after 
        freezing the models and unlocking the test data suggest that, for a reliable internal 
        validation and high test performance, the following factors are crucial:
        (a) data sets with a large number of samples, even if 
        they result from combining somewhat different, partly non-prospective data sets, 
        (b) relying on already-existing molecular signatures rather than fitting new ones and 
        (c) deploying simple, generalized linear models that can handle batch effects. 
        The fourth chapter 
        discusses how we can develop even better models in the future, especially for MMML-Predict. 

        We conclude that with currently available data and statistical and computational methods, 
        we can identify a DLBCL subpopulation with poor survival that is larger and 
        holds significantly more high-risk patients than that identified by the IPI. With more 
        data and new and 
        more accurately measured features in the future, we expect to be able to further improve the 
        performance of such models.
    \end{abstract}

    % \layout
    \tableofcontents
    \chapter{Introduction} \label{chap:intro}

\section{Diffuse large B-cell lymphoma: treatment and state-of-the-art risk prediction}

Diffuse large B-cell lymphoma (DLBCL) is the most common type of B-cell lymphoma in adults, 
accounting for approximately 30\% of all diagnoses. This aggressive and heterogeneous group of 
lymphoid neoplasms typically originates from malignant transformed germinal center (GC) B cells, 
exhibiting diverse phenotypic, genetic, and clinical characteristics. The clinical presentation of 
DLBCL varies significantly, with differences in tumor load and patient performance status, leading 
to varied outcomes.

Despite being an aggressive and, if left untreated, fatal disease, DLBCL is a highly curable disease with the 
application of intensive immunochemotherapy even in the elderly population. The standard treatment 
for DLBCL has long been immunochemotherapy with the R-CHOP regimen. This 
regimen has significantly improved survival rates, with approximately two-thirds of patients 
achieving a cure. However, the remaining one-third of patients, especially 
those with relapsed or refractory disease, continue to face poor outcomes \citep{glass17}. For patients who do not 
respond to initial treatment, options include salvage chemotherapy and newer therapies like CAR-T cells. 
Approximately one-third of patients with DLBCL succumb to their disease, particularly those with 
relapsed or refractory conditions, for whom cure rates remain low. This underscores the clinical 
need for an accurate, robust, affordable, and easy-to-use tool that can identify patients at high risk 
for treatment failure early in their treatment course, ideally within the first three cycles of 
induction chemotherapy. 

To this end, the International Prognostic Index for non-Hodgkin's lymphoma (IPI) was established in the 
1990s. It incorporates five clinical binary clinical features one can measure cheaply and reliably without batch 
effects: Is the patient older than 60 years? Is the cancer advanced (Ann Arbor Stage III or IV)? Does 
the patient have a higher-than-normal lactate dehydrogenase (LDH) level? Is the patient already 
bedridden (performance status > 1)? Is the patient's cancer in more than one extranodal site? The 
IPI is then the number of positive answers to these questions, an integer between 0 and 5. Although 
primitive and arbitrary at first glance, the IPI is the result of a rigorous statistical analysis 
of a large dataset of \num{1872} patients: out of twelve candidate features, the IPI inventors
first selected those features that were independently and significantly associated with survival 
(namely the five above mentioned features), and fit a Cox proportional hazards model to them;
since the relative risks for all five features turned out to be similar, they simplified the model 
by just counting the number of present risk factors~\cite{ipi93}. 

The IPI is a simple yet robust clinical tool used globally to predict risk and guide 
treatment decisions in DLBCL patients. It has been the cornerstone of risk assessment for the last 
three decades, no alternative has gained widespread acceptance outside of clinical 
trials~\citep{ipi-stay-strong}. 
Despite its effectiveness in large cohorts, the IPI and other individual 
biomarkers do not reliably predict the clinical course for each patient, particularly failing to 
identify those at high risk for early treatment failure who may benefit from alternative therapeutic 
approaches. E.g., in the prospective trial comprising \num{466} patients used as test set in 
\citep{staiger20}, only \num{3.4}\% of patients have the maximum IPI score of 5 -- too few
to gain special attention in clinical practice and to incentivize the pharmaceutical industry to 
develop new treatments. All other cohorts defined via $\text{IPI} \geq i, i = 0, 1, 2, 3, 4$, lack 
precision: the proportion of patients with progression-free survival less than two years is below 
\num{50}\% -- too few to persuade a patient to undergo an experimental treatment instead of the 
standard R-CHOP regimen.

For this thesis, we define high-risk DLBCL patients as those who face cancer progression 
within two years after the start of the treatment. Two years is a time frame accepted by both 
patients and clinicians, which makes roughly a fourth of DLBCL patients high-risk \cite{staiger20}.
It's also the threshold used in the MMML-Predict we will introduce next.

\section{The MMML-Predict project} \label{sec:intro-mmml}

Renowned lymphoma experts from across academic Germany -- clinical trialists, biostatisticians, 
bioinformaticians, lymphoma pathologists and translational lymphoma biologists -- have formed the 
consortium MMML-Predict to develop and roll out a new, robust, simple-to-use, cost-effective and 
parsimonious prognostic tool for DLBCL 
which yields a clinically more relevant high-risk group. This tool, the MMML-Predictor, will allow 
patients and clinicians early in the treatment cycle to make an informed decision if they want to 
continue with the standard R-CHOP treatment or switch to novel, more experimental treatments. 

In a discovery phase, MMML-Predict will enroll 200 patients in a prospective trial at first diagnosis and 
collect all clinical and molecular risk features that alone predict an unfavorable outcome, 
including clinical scores (like the IPI), gene-expression based factors (like cell-of-origin signatures,
immune scores) and genetic determinants (like MYC, BCL2, TP53, germline and somatic signatures).
As a novel approach, they will measure dynamic response determinants during treatment (PET-CT 
and liquid biopsy-based MRD detection). It is unknown if these features capture similar or different 
risks; combining them may finally bring the significant progress in the understanding of the DLBCL 
biology we have made over the last decades to the patients' bedside.

The group around Markus Loeffler in Leipzig will evaluate the readily trained MMML-Predictor on a 
test cohort of another 100 patients enrolled for this project for whom only those features used 
in the MMML-Predictor will be measured. The new classifier has to achieve a rate of positive 
predictions -- or prevalence -- of at least \num{15}\% and a precision (for high-risk) significantly above that of 
$\text{IPI} \geq 4$ on the test cohort, meaning the \num{95}\% confidence interval of the new 
classifier's precision according to the Clopper-Pearson method must not include the precision of 
the group determined by $\text{IPI} \geq 4$. In other prospective trials, this precision of the IPI 
is at around \num{35}\%; taking this number, a prevalence of \num{15}\% and the test cohort size into 
account, a calculation of the MMML-Predict consortium suggests that a precision of at least \num{50}\%
of the MMML-Predictor suffices.

\section{The role of this thesis within MMML-Predict}

Inside MMML-Predict, Rainer Spang's group in charge to develop the MMML-Predictor. This is 
a supervised-learning task with a binary classification problem: progression-free survival less than 
two years -- or high-risk DLBCL -- is the positive group, progression-free survival more than two 
years -- or low-risk DLBCL -- is the negative group. We have a small number of samples, a large 
number of features and an enormous amount of freedom in how we design and train the classifier. 
Hence, we need to deal with the curse of high dimensions if we want to use the high-dimensional part 
of the data directly (and not indirectly via late integration of already-existent signatures) and, 
more importantly, we need to take care that in our internal validation (usually a cross-validation)
we do not overfit the data. While we should be able to tackle the first problem with regularization,
for the latter one we need a trustworthy internal validation strategy and, most importantly, we 
must not validate too many models in the first place.

Since, as of July 2024, the MMML-Predict train cohort hasn't yet arrived in Regensburg, this thesis
will imitate the train-test scenario of MMML-Predict on already-available DLBCL data sets. This thesis 
has two main goals:
\begin{enumerate}
    \item We want to show that with data including traditional clinical and modern 
        molecular features, both possibly already condensed to signatures, we can indeed deliver 
        the desired model, which yields a larger and more precise high-risk group of DLBCL patients.
        With an eye to rolling out the MMML-Predictor in clinical practice, we want to demonstrate 
        that we can design this model in such a way that one can transfer it from one platform to
        another without losing its predictive power.
    \item We want to develop heuristics and recommendations to answer the question which candidate 
        models (including their hyperparameters) are worth training and validating and -- more 
        importantly -- which are not. For this, we need to infer which models we can reliably validate 
        and which perform well on independent test data. These findings will guide us on how to keep 
        the number of candidate models we fit to the MMML-Predict training data low, thereby helping 
        us to avoid overfitting in the internal validation and to submitting a model that convinces 
        in the validation, but disappoints on the test cohort in Leipzig.
\end{enumerate}
    \chapter{Methods} \label{chap:methods}

Finding the best possible model for our given task will not be possible from just theoretical 
considerations; we will have to fit several models to our data and demonstrate the performance of 
the chosen one convincingly. Section \ref{sec:train-val-test} will lay the state-of-the-art 
train-validate-test paradigm for this. In section \ref{sec:candidate-models}, we will introduce 
the candidate models and the hyperparameters governing their fitting process. We will start with 
model-agnostic hyperparameters before we go on to present well-known model types and finally 
introduce a method that lets us train compositions of multiple models.

\section{Training, validation and testing}\label{sec:train-val-test}

The design of MMML-Predict with its train and test cohort, where the people developing the predictor 
never get to see the test cohort, pays tribute to the standard two-step approach in supervised-
learning tasks: We use the train cohort to fit multiple models (training) and choose the model
among these models that we have confidence performs best on new data (validation). We then touch 
the test cohort for the first time as we evaluate the chosen model on it to persuade the outside 
world we have come up with model worth deploying.

Since we want to have the same conditions for data used in this thesis as later in MMML-predict,
we randomly split given data $(\mathbf{x}, y)$ with predictor $\mathbf{x}$ and response $y$ into 
a train cohort $(\mathbf{x}_\text{train}), y_\text{train}$ and a test cohort $(\mathbf{x}_\text{test}, y_\text{test})$.
The following assumes a single, split data set.

\subsection{Training and validation}\label{subsec:train-val}

To be able to discuss some probabilistic caveats of validation later, we introduce some formal 
notation. We start with a set of tuples of hyperparameters $H$, where every $h \in H$ defines a model
up to its parameters; determining the parameters of a model, by definition, is the job of the 
algorithm optimizing a given loss function, we refer to this as fitting. There is a one-to-one 
mapping between $H$ and the set of candidate models.

For every hyperparameter tuple $h \in H$, we
\begin{enumerate}
    \item fit the model to the train cohort subject to $h$ in $k$-fold cross-validation. This means 
        we randomly assign the training samples into $k$ subsets, called folds, and then actually 
        fit $k+1$ models, one model on all samples and for every $i = 1, \ldots, k$, we train a model
        on all samples except the $i$-th fold and obtain its cross-validated predictions on the 
        $i$-th fold; these are predictions on new data. Doing this for all folds, we obtain a 
        cross-validated prediction for every sample and hence a vector of cross-validated predictions
        $\hat{y}_\text{train} = \text{cv}(h)$ of the same shape as $y_\text{train}$. There may be 
        other methods than ordinary cross
        validation in this step, but their purpose is always the same: yield a prediction for every 
        sample that looks like it was made on new data.
    \item We use the cross-validated predictions to calculate the validation error 
        $\text{err}(y_\text{train}, \hat{y}_\text{train})$.
\end{enumerate}

Finally, we select the model $m^*$ defined by the hyperparameter tuple 
\begin{align}
    h^* = \argmin_{h \in H} \ \text{err}(y_\text{train}, \text{cv}(h)).
\end{align}

We note that for all $h$, the validation error is a random variable, which e.g. depends on the 
random assignment of the fold, the train cohort and possibly random involved in the loss-function
optimizer. It is a well known property of independent,
identically distributed (i.i.d.) real random variables $X_i, i \in \NN$, that their extreme values 
are notoriously unstable in the sense that for all $t \in \RR$ with $P(X_1 \geq t) < 1$, 
\begin{align}
    \Pr\left( \min_{1 \leq i \leq n} X_i \geq t \right) = \Pr(X_1 \geq t)^n \to 0 \quad
    \text{as } n \to \infty.
\end{align}

While not all $\text{err}(y_\text{train}, \text{cv}(h)), h \in H$, are i.i.d. -- after all, they 
all depend on 
$(\mathbf{x}_\text{train}, y_\text{train})$ and some model are better suited for the problem than others --,
we can have situations where at least a subset of $H$ delivers validation errors close to being i.i.d.

\paragraph{Models merely guessing}
Imagine models not suited to deal with the problem at all. Their cross-validated predictions are 
merely guesses on $y_\text{train}$ and therefore i.i.d. There is a non-zero probability to guess
all true outcomes on the train cohort correctly and hence it is only a matter of the number of 
such merely guessing models until which get the minimal possible error for one of them. The number of 
such models can indeed grow large e.g. if we keep trying ever new hyperparameters for a model class 
that can per se not describe the dependence between $\mathbf{x}$ and $y$ at all.
As a practical advice, we should not spend too much time und put too much hope into improving 
bad models by hyperparamter tuning.

\paragraph{General scenario}
In general, the validation errors of the candidate models are independent to some extent because we fit them
subject to differing hyperparameters. As a result, the validation error of every model can 
fluctuate freely around its expected value to some extent. The larger $H$ gets, the 
more severely underestimated validation errors we will have in the ranking. This 
increases the odds that $m^*$ is just a model with a tremendously underestimated validation error, 
leading to a bad surprise on the test cohort. As a practical advice, we should be careful with 
trying too many model families and vastly differing hyperparameters for a given base model 
because both may give us quite independent models.

\paragraph{}
This underscores the need for a smart and lean choice of the candidate models or hyperparameter 
tuples, respectively. Exclusion criteria for candidate models may be theoretical considerations
and experience by both us and others. Concerning experience, this means we will not move away too 
far from default hyperparameters and we will develop heuristics in chapter \ref{chap:results} as 
move from data set to the next.

\subsection{Testing}

We calculate the predictions $m^*(X_\text{test}) = \hat{y}_\text{test}$ of the best validated model 
$m^*$ on the test cohort and estimate its performance on independent data via 

\begin{align}
    \text{err}(y_\text{test}, \hat{y}_\text{test}).
\end{align}

\subsection{Choice of the error function}\label{sec:error-function}

Taking into account the calculations in section \ref{sec:intro-mmml},
we want to optimize the precision of the high-risk group, which must comprise at least 15\% of the 
samples in our data. Since an error should be the lower, the better the model is, we choose 
$\text{err}$ to be the minimum of the negative precisions with a prevalence of at least 17\%. 

Usually, we need 
to take a minimum over \textit{several} precisions because most models do not output the final 
classification.
Instead they return a continuous score where a higher score means a higher probability of being
in the positive class, in our case being high-risk, and this continuous score needs thresholding
via $\hat{y}_i > t$ for some $t \in \RR$.
We notice that as we increase $t$ and thereby decrease the prevalence, the obtained 
adjacent precision values more and more depend on one another; for high $t$ and low 
prevalences, fluctuations of up to 50 percentage points would be possible, but requiring a 
prevalence of at least 17\% caps such fluctuations at 
\begin{align}
    \frac{1}{\ceil*{\num{0.17} \cdot n_\text{test}}}
\end{align}
for $n_\text{test}$ samples in the test cohort. This gives us an error function that is tailored 
for our problem, sufficiently far-sighted and robust.

As indicated by the notation $\text{err}(y_\text{train}, \hat{y}_\text{train})$ and 
$\text{err}(y_\text{test}, \hat{y}_\text{test})$, we optimize $t$ on the true outcomes 
of the train cohort when validating and we optimize it again on the true outcomes of the test cohort 
when testing. Strictly speaking, $t$ is a hyperparameter, which we optimize on the 
test cohort during testing; this sounds delicate. The results however will show that the optimal 
choice for $t$ (or at least an 
almost optimal choice) on both train and test cohort correponds to a prevalence of 
slightly above 17\% such that one can agree on the following when it comes to the MMML-Predict data: 
choose $t$ as the 17\%-quantile of the coninuous model output on the test cohort.

\subsection{Inter-technical variability} 
One could instead suggest to already optimize the output threshold on the cross-validated 
predictions. While this promises a stricter train-test regime at first glance, fixing the output 
threshold once and for all neglects inter-technical variability. An always-present problem in 
Bioinformatics, inter-technical variability refers to the fact that one and the same sample 
measured on different platforms or in different labs, in short: with different protocols, may 
lead to different values for the same feature. If we measure the same $p$ features of the $n$ 
patients in $\mathbf{x} \in \RR^{n \times p}$ again with another protocol, we end up with another predictor 
$\mathbf{z} \in \RR^{n \times p}$.
For gene-expression levels,
we can often well model the discrepancies with two indepedent biases, sample-wise effects $\theta_i$
and feature-wise effects $\omega_j$, leading to
\begin{align}
    \Delta_{ij} = \mathbf{z}_{ij} - \mathbf{x}_{ij} = \theta_i + \omega_j + \epsilon_{ij}
\end{align}
for residues $r_{ij}$. Assuming accurate modeling, i.e. small residues, we can well approximate
\begin{align}
    \mathbf{z}_{ij} \approx \tilde{\mathbf{z}}_{ij} = \mathbf{x}_{ij} + \theta_i + \omega_j.
\end{align}
We can now apply an ordinary linear -- or Gaussian -- model to $\mathbf{z}$ obtaining
\begin{align} \label{eq:inter-tech}
\begin{split}
    \beta_0 + \sum_{j=1}^p \beta_j \mathbf{z}_{ij} &\approx \beta_0 + \sum_{j=1}^p \beta_j \tilde{\mathbf{z}}_{ij} \\
    &= \beta_0 + \sum_{j=1}^p \beta_j \mathbf{x}_{ij} + \sum_{j=1}^p \beta_j \theta_i + \sum_{j=1}^p \beta_j \omega_j.
\end{split}
\end{align}
The second summand is zero under the zero-sum constraint $\sum_{j = 1}^p \beta_j = 0$ and the third 
summand is constant across all samples and can be absorbed by the intercept \cite{transplatform17}. 

Under these (not very 
strict) conditions, going from one protocol to another just leads to a constant shift of the model 
output; even for generalized linear models, i.e. a Gauss model composed with some monotic link function,
inter-technical variability does not change the ordering of the samples. All one needs to do to 
obtain a final classification is calibrating the threshold for the output scores. However, this 
also demonstrates that while generalized linear models can cope with inter-technical variability 
pretty well, there is no point in using the same output threshold across all protocols.

\section{Candidate models}\label{sec:candidate-models}

\subsection{Model-agnostic hyperparameters}\label{subsec:model-agnostic}

Model-agnostic hyperparameters are those hyperparameters we can set and tune for every model. They 
concern the predictor matrix $\mathbf{x} \in \RR^{n \times p}$ and the response vector $y \in 
(\RR \times \{ 0, 1 \})^n \cup \{ 0, 1 \}^n$. We can provide the response in two formats. In the 
first case, the response has two entries per sample: the first one is the time to event, in our 
case progression-free survival, and the second one indicates if the event has occured (1) or 
if the patient was censored before it could occur (0); more on this in subsection 
\ref{subsec:core-models} when we present the Cox model. In the second case, the response is a binary 
vector with 1 indicating the positive class (high-risk DLBCL)
and 0 indicating the negative class (low-risk DLBCL). Note that in the latter case we need 
to discard samples censored before two years, but this is typically only a small proportion of the 
samples in the data set.

\subsubsection{A-priori feature selection}

Of course, not every feature in the data set should be part of the predictor matrix $\mathbf{x}$. Including 
features measured to toward the end of a patinent's therapy or even after it, is cheating and we 
should definitely exclude them. Excluding any of the remaining features is hard to justify: If we 
know for a feature that it alone is associated with the response, we of course include it in $\mathbf{x}$.
On the other hand, if a feature alone shows no link to the response, the right model might still be 
able to leverage it in combination with other features; but at this step, we do not know if such 
a model exists and if it is among our candiate models. The only remaining option for handpicking 
features a priori is to brute-force the problem and try out all $2^p$ combinations of features.
Even if we trust on regularization to do its job for the high-dimensional part (typically several 
hundreds or even thousands of gene expression levels), just \num{10} remaining fetures would still
leave us with $2^{10} = 1024$ combinations to try out -- per model class. While this may be 
computationally feasible, validation would be a statistical fiasco as laid out in subsection 
\ref{subsec:train-val}. Therefore, we decided to reduce this quickly exploding problem to only 
a couple of decisions to 
make and have all the other feature selection happen during model fitting. 

\paragraph{Gene-expression data}
The first and most important decision to make is: do we want to include the high-dimensional
gene-expression part of the data at all? If we do not, we suddenly have $p \ll n$ instead of 
$p \gg n$ and can use way more complex models. 

\paragraph{Features in different formats}
Second, we need to make a couple of more decisions regarding the format of some features.
\begin{itemize}
    \item Concerning the IPI, one can include the five IPI features in their original, continuous 
        form; discretized as by the IPI inventors; the IPI score as a single continuous feature; 
        or the so-called IPI group, a partioning of the IPI score into three groups. Similar widely 
        accepted thresholds may exist for other clinical and genetic features.
    \item For already existing models, we can include their continuous output or also threshold 
        it. E.g., cell-of-origin signatures can be -- and most of the time in data sets already 
        have been -- thresholded into ABC, GCB and unclassified subtypes. The above-mentioned IPI 
        group thresholds the IPI model. 
\end{itemize}

Imagine, a data set provides the five IPI features in continuous format and the continuous output 
of some gene-expression signature for which 
its inventors also provide a preferred threshold. If we want to decide for exactly one format 
per feature, this leaves us with $5 \cdot 2$ possibilities for this simple case. If we want try out 
every combination of non-empty subsets, the number of possibilities jumps to $2^5 \cdot 2^2 = 
128$. 

Our solution here again is to be generous and include all formats of a feature in $\mathbf{x}$ a 
certain model may benefit from. This sentence usually 
simplifies to do: include all widely used formats of a feature in $\mathbf{x}$. E.g., generalized linear models
cannot threshold continuous features, so a feature in its thresholded format may give a rough 
contribution to the prediction while the same feature in its continuous form may further refine it.
Moreover, even decision trees can benefit from a feature additionally provided in its thresholded 
format if this threshold has been inferred on a much bigger data set since the decision tree 
may not be able to find the threshold itself.

\subsubsection{Imputation}

If values in $\mathbf{x}$ are missing, i.e. written as \texttt{NA}, actions we can take fall into two 
categories: we discard the part of data affected by missing values or we replace the missing values 
with some realistic estimate.

\paragraph{Discarding part of the data}

In a first step, we discard a feature if it is not available for a large proportion of the samples.
This enables reasonable computing.

\paragraph{Imputing}

We then mean-impute the remaining missing values. At this point, we model a categorical feature 
with $c$ categories as $c-1$ binary dummy features; if for a sample this categorical feature is not 
available, all $c-1$ dummy features will be \texttt{NA}. For every column in $\mathbf{x}$, we calculate the 
mean of the available features in $\mathbf{x}$ and replace the \texttt{NA} values with it. 
For a missing categorical feature, every dummy feature therefore holds the marginal probability 
that the feature is in the $k$-th category. This is efficient, transparent and easily applicable to 
new data.

\subsubsection{Adding combinations of discrete features to the predictor}

We add all combinations of at most $n_\text{combi}$ categorical features that are positive in a 
share of at least $s_\text{min}$ patients to $\mathbf{x}$; e.g., we add a column ``female \& ABC-type 
tumor'' if at least \num{5}\% of patients have this property. We always choose $s_\text{min}
= 5\%$. We set $n_\text{combi} = 1$ for models that facilitate interactions between features 
themselves, otherwise we set $n_\text{combi} = 3$.

\subsubsection{Tuning the definion of ``high-risk''}

Defining a patient as high-risk if and only if the progression-free survival is less than two years
is clinically accepted, yet quite arbitrary decision. It may be not be the time cutoff that 
separates patients' survival 
best from a biological point of view. Therefore we can provide the fitting 
algorithm a modified response $y$, governed by the time cutoff $T > 0$.
\begin{itemize}
    \item For binary response $y \in \{0, 1\}^n$, we set $y_i = 1$ if the progression-free survival 
        of patient $i$ is less than $T$ and $y_i = 0$ otherwise.
    \item For response $y \in (\RR \times \{0, 1\})^n$ with time to event and censoring, we censor 
        all samples with time to event exceeding $T$ at $T$.
\end{itemize}

\subsection{Core models}\label{subsec:core-models}

All models trained, validated and tested in this thesis at the core consist of ordinary linear, 
logistic and Cox proportional-hazards models with additional properties, as well as random forests.
In this chapter, we want to introduce the design of these models and the hyperparameters governing 
their fitting process.

Formally, we deal with a probability space $(\Omega, \mathcal{A}, P)$ that we do not and cannot 
specify any further because we only get in touch with two random variables 
\begin{align}
    X = (X_1, \ldots, X_p): \Omega \to \RR^p \text{ and } Y: \Omega \to \RR,
\end{align}
the predictor and the response, respectively. More 
precisely, we observe independent training samples $(x_i, y_i) \in \RR^{p+1}, i = 1, \ldots, n$, 
distributed according to $(X, Y)$; $x_i$ is the $i$-th row of the predictor matrix $\mathbf{x}$ and
$y_i$ is the $i$-th entry of the response vector $y$. The i.i.d. test samples follow the same 
distribution as the training samples, but in this subsection everything is about the training and 
hence the training samples; if we say samples in this subsection, we always refer to the training   
samples.

\subsubsection{Generalized linear models}\label{subsubsec:glm}

Here, we work with binary response, i.e. $im(Y) \subset \{0, 1\}$. Both ordinary 
linar models and logistic regression models are generalized linear models (GLMs). In 
a GLM, $Y$ follows an exponential-family distribution and there is an invertible link function 
$g: \RR \to \RR$ and parameters $(\beta_0, \beta) \in \RR^{p+1}$ such that
\begin{align}
    g(E(Y \cond X = x)) = \beta_0 + \sum_{j=1}^p \beta_j x_j \quad \text{for all } x \in \RR^p,
\end{align}
where the $E(Y \cond X = x)$ is the expected value of $Y$ given $X = x$. For the above samples 
$(x_i, y_i)$, we obtain
\begin{align}
    \mu_i = E(Y \cond X = x_i) = g^{-1}\left(\beta_0 + \sum_{j=1}^p \beta_j x_{ij}\right).
\end{align}
Relating $\mu_i$ to $y_i$ via a log-likelihood, will allow us to fit $(\beta_0, \beta)$ to 
$(x_i, y_i)_{i = 1, \ldots, n}$. We can express $\sum_{j=1}^p \beta_j x_{ij}$ in terms of the 
Euclidean scalar product as $x_i^T \beta$, where $x_i^T$ denotes the transpose of the column vector 
$x_i$.

\paragraph{Ordinary linear model}
Here, $g = \text{id}$. $Y$ and $Y \cond X = x_i$ for all $i = 1, \ldots, n$ follow a normal
distribution with fixed standard deviation $\sigma > 0$ (a property called homoscadasticity), but 
varying mean. The log-likelihood of $\mu_i$ is
\begin{align}
\begin{split}
    \ell(\mu_i; y_i) &= \log\left( \frac{1}{\sqrt{2\pi}\sigma} 
        \exp \left( -\frac{1}{2\sigma^2}(y_i - \mu_i)^2 \right) \right) \\
    &= -\frac{1}{2\sigma^2}(y_i - \mu_i)^2 - \log \left( \sqrt{2\pi}\sigma \right).
\end{split}
\end{align}
Up to a constant shift and re-scaling, which does not affect maximizing the log-likelihood, this
is the well-known squared error.

\paragraph{Logistic model}
Here, the linear predictor is the log-odds for $Y = 1$ over $Y = 0$ given $X = x$, i.e.
\begin{align}
    g: \mu \mapsto \log\left( \frac{\mu}{1 - \mu} \right), \text{ hence } g^{-1}: \eta \mapsto
    \frac{1}{1 + \exp(-\eta)}.
\end{align}
Y and $Y \cond X = x_i$ for all $i = 1, \ldots, n$ follow a Bernoulli distribution with parameter 
$p$ and $\mu_i$, respectively. The log-likelihood of $\mu_i$ is
\begin{align}
    \ell(\mu_i; y_i) &= \log\left( \mu_i^{y_i} (1 - \mu_i)^{1 - y_i} \right) 
    = y_i \log(\mu_i) + (1 - y_i) \log(1 - \mu_i). 
\end{align}

\paragraph{}
Using $\mu_i = g^{-1}(\beta_0 + x_i^T \beta)$, in both cases we can express the 
log-likelihood 
for $(\beta_0, \beta)$. For our i.i.d. samples $(x_i, y_i)_{i = 1, \ldots, n}$, we therefore obtain 
\begin{align}
    L(\beta_0, \beta) &= \sum_{i=1}^n \ell(\mu_i; y_i) 
    = \sum_{i=1}^n \ell\left( g^{-1}\left( \beta_0 + x_i^T \beta \right); y_i \right).
\end{align}
We can augment $L$ with two hyperparameters, sample weights $w_i > 0$, $i = 1, \ldots, n$, and the 
zero-sum constraint with respect to zero-sum weights $u_j \geq 0$, $j = 1, \ldots, p$, yielding 
\begin{align} \label{eq:loss-glm-no-lasso}
    L(\beta_0, \beta) &= \sum_{i=1}^n w_i \ell\left( g^{-1}\left( \beta_0 + x_i^T \beta \right); 
    y_i \right) \quad \text{subject to } \sum_{j=1}^p \beta_j u_j = 0.
\end{align}
While we do not deviate from default value $1/n$ for the sample weights in the results chapter 
\ref{chap:results}, we will discuss in chapter \ref{chap:discussion} how we can use them in the 
future in a natural way without getting lost in the vast amount of freedom one has to choose them.

What is the zero-sum constraint good for? Imagine, part of the features suffer from sample-wise 
shifts. We denote these features by $J \subset \{1, \ldots, p\}$ and the shifts by $s_i \in R$, 
$i = 1, \ldots, n$. We now set the zero-sum weights  
\begin{align}
    u_j =
    \begin{cases}
        c & \text{if } j \in J, \\
        0 & \text{else},
    \end{cases}
\end{align}
for some $c > 0$ (usually $c = 1$); we can express this conviently with the help of the indicator 
function $\chi_J$ as $u_j = \chi_J(j) c$. By the zero-sum constraint, $\sum_{j \in J} c \beta_j = 0$  
and hence $\sum_{j \in J} \beta_j = 0$. In this situation, the linear predictor 
\begin{align}
    \beta_0 + \sum_{j=1}^p \beta_j (x_{ij} + s_i \chi_J(j))
    = \beta_0 + x_i^T \beta + \sum_{j \in J} \beta_j s_i = \beta_0 + x_i^T \beta,
\end{align}
as $\sum_{j \in J} \beta_j s_i = s_i \sum_{j \in J} \beta_j = 0$, is invariant. If $J$ consists of 
exactly the features holding gene-expression levels, we can transfer our model to other protocols 
and only need to re-calibrate the intercept, as shown in eq. (\ref{eq:inter-tech}). Moreover, we 
can switch off the zero-sum constraint for all features by setting $u_j = 0$ for all 
$j = 1, \ldots, p$. The zero-sum constraint is cheap in terms of model complexity: it only removes 
one degree of freedom; but it is expensive in terms of computational complexity: enforcing it in
every step of the coordinate descent leads to a considerably higher computation time. However, the 
coordinate descent minimizes eq. (\ref{eq:loss-glm-no-lasso}) with a regularization term add. Before 
we discuss this, we want to introduce a model that is closely related to GLMs.

\subsubsection{Cox proportional-hazards model}\label{subsubsec:cox}

For the Cox proportional-hazards model -- or in short: Cox model -- the response variable $Y$ 
measures the time until the event of interest occurs (``event'') or the time after which at some 
point the 
event occurs (``censoring''). Another binary random variable $\delta: \Omega \to \RR$ encodes 
which of these two options is the case: $\delta(\omega) = 1$ for the event, $\delta(\omega) = 0$ 
for censoring. This pays tribute to an important characteristic of survival trials: some
participants drop out of the trial before the event could happen -- e.g. the trial terminated, the 
patient decided to withdraw or died from another cause -- or the event luckily never happens.

\paragraph{Hazard function} To understand what the Cox model predicts, we need another random 
variable $T: \Omega \to \RR$, the time to the event (this time not affected by censoring), and 
require $T$ to have a density $f_T: \RR \to \RR$. We define the hazard function $h: \RR \to \RR$
via
\begin{align}
    h(t) = \lim_{\Delta t \to 0} \frac{P(t \leq T < t + \Delta t \cond T \geq t)}{\Delta t}.
\end{align}
We can interprete $h(t)$ as the instantaneous rate at which the event is occurring at time $t$, 
given that the event has not occured before time $t$. Let $F_T$ be $T$'s distribution function. Then 
we can write
\begin{align}
    \frac{P(t \leq T < t + \Delta t \cond T \geq t)}{\Delta t} = 
    \frac{F_T(t + \Delta t) - F_T(t)}{\Delta t \cdot (1 - F_T(t))}.
\end{align}
Hence, 
\begin{align}
    h(t) = \lim_{\Delta t \to 0} \frac{F_T(t + \Delta t) - F_T(t)}{\Delta t \cdot (1 - F_T(t))} 
    = \frac{f_T(t)}{1 - F_T(t)} = \frac{f_T(t)}{S_T(t)},
\end{align}
where $S(t) = 1 - F_T(t)$ is the survival function, and $h(t)$ is well-defined as long as $S_T(t)
> 0$.

\paragraph{Conditional hazard} The Cox model wants to get a hand on \textit{conditional} hazards.
Intuitively, it is clear that the population with one value of $X$ can have a vastly different 
hazard function than the population with another value of $X$ if $X$ has predictive power for $T$.
Formally, however, it is hard to define a conditional hazard function: $X$ may very well 
have a continuous distribution, hence we condition on an event of probability $0$ when we write 
$P(t \leq T < t + \Delta t \cond T \geq t, X = x)$ -- it is hard to define this in a natural
and straightforward way. In practice, the measurements in $X$ often fluctuate around their true 
values anyway, so it is sensible to condition on a small neighborhood of $x$, say an $\epsilon$-ball 
around $x$ with respect to the Euclidean norm $|\cdot|_2$ for some small $\epsilon > 0$. This 
amounts to
\begin{align}
    h(t \cond X = x) = \lim_{\Delta t \to 0} \frac{P(t \leq T < t + \Delta t \cond T \geq t, 
    |X - x|_2 < \epsilon)}{\Delta t}
\end{align}
or -- equivalently -- $h(t \cond X = x)$ is the hazard of $T$ restricted to $(\Omega, \mathcal{A}, 
P)$ conditioned on $\Omega_x = \{ \omega \in \Omega: |X(\omega) - x| < \epsilon \}$; unlike before, 
we usually now have $P(\Omega_x) > 0$.

\paragraph{Proportional hazards} The Cox model assumes \textit{proportional} conditional hazards, 
i.e. 
\begin{align}
    h(t \cond X = x) = \lambda_x h_0(t) 
\end{align}
for some (unspecified) baseline hazard $h_0$ and $\lambda_x > 0$. Moreover, it assumes $\lambda_x$ depends 
on $x$ via 
\begin{align}
    \lambda_x = \exp(x^T \beta),
\end{align}
for parameters $\beta \in \RR^p$.

\paragraph{Partial log-likelihood}
Given our independent samples $(x_i, y_i, \delta_i) \in \RR^{p+2}$, $i = 1, \ldots, n$, distributed 
according to $(X, Y, \delta)$, let $E \subset \{ 1, \ldots, n \}$ refer to the non-censored samples 
($\delta_i = 1$) and for $i \in E$, let $R_i$ denote the set of samples where no event has occurred 
until $y_i$, i.e. $y_j \geq y_i$ for all $j \in R_i$. For fixed $i \in E$ and time to event $y_i$,
conditionally on the risk set $R_i$, the probability that the event occurs on sample $i$ as 
observed is
\begin{align}
    \frac{h(y_i \cond X = x_i)}{\sum_{j \in R_i} h(y_i \cond X = x_j)} = 
    \frac{\exp(x_i^T \beta) h_0(y_i)}{\sum_{j \in R_i} \exp(x_j^T \beta) h_0(y_i)} = 
    \frac{\exp(x_i^T \beta)}{\sum_{j \in R_i} \exp(x_j^T \beta)}.
\end{align}
Taking all events into account yields the partial likelihood 
\begin{align}
    \tilde{\ell}(\beta) = \prod_{i \in E} \frac{\exp(x_i^T \beta)}{\sum_{j \in R_i} \exp(x_j^T 
    \beta)}.
\end{align}
Because $T$ is continuous, there is a zero probability for two distinct samples having the same 
event time, so need not deploy one of the more sophisticated likelihoods capable of handling ties.
Analogously to eq. (\ref{eq:loss-glm-no-lasso}), we can equip the log-likelihood with sample 
weights $w_i > 0$ and zero-sum weights $u_j \geq 0$ such that the loss function reads 
\begin{align}
\begin{split}
    L(\beta) &= \sum_{i \in E} w_i x_i^T \beta - \log \left( \sum_{j \in R_i} w_j \exp(x_j^T
    \beta) \right) \\
    & \text{subject to } \sum_{j=1}^p \beta_j u_j = 0.
\end{split}
\end{align}
As with GLMs, we can leverage the zero-sum constraint to reach shift-invariance.

\subsection{Elastic-net regularization}

Modern molecular measurements are a big factor in our hope that MMML-Predict can succeed and improve 
the IPI. These measurements, like gene expression levels, show up as hundreds if not thousands of 
features in our data, meaning we usually have $p > n$ if not $p \gg n$. In this situation, we 
cannot uniquely determine the parameters of any of the models by maximzing the log-likelihood: 
the models have way more degress of freedom than we have samples, find biologically meaningless 
and non-reproducible structures in the training cohort and generalize to poorly to new data.

The solution to this is to restrict the freedom of the model to the extent that it is forced to 
learn the gist from the data. For that reason, we want to penalize model complexity. Elastic-net 
regularization, a widely adopted method proposed by Zou and Hastie \cite{elasticnet05}, does this 
job for us. The elastic net generalizes two well-established regularization methods,
\begin{itemize}
    \item ridge regularization \cite{ridge70}, which forces the parameter vector $\beta$ to 
        stay inside a ball around the origin by limiting the squared $\ell_2$ norm 
        of the parameters, and 
    \item LASSO (least absolute shrinkage and selection operator) regularization \cite{lasso18}, 
        which forces $\beta$ to stay inside a diamont centered at the origin by limiting the 
        $\ell_1$ norm of the parameters,
\end{itemize}
by combining them with weight factor $\alpha \in [0, 1]$ into 
\begin{align}
    p_{\text{enet}, \alpha}(\beta) = \frac{1-\alpha}{2} |\beta|_2^2 + \alpha |\beta|_1.
\end{align}
Muliplited with the regularization hyperparameter $\lambda \geq 0$ we can add it to the negative 
log-likelihood to obtain the regularized loss function
\begin{align}
    \mathcal{L}(\beta_0, \beta) = -L(\beta_0, \beta) + \lambda p_{\text{enet}, \alpha}(\beta).
\end{align}
Note that in the Cox case, $\mathcal{L}$ does not depend on $\beta_0$ and we just ignore $\beta_0$ 
in the above equation. 

\paragraph{Effect on correlated features}
While both ridge and LASSO regularization shrink the coefficients, they act differently in the case 
of correlated features. Ridge regression tends to force the coefficients of correlated features to 
similar values while LASSO tends to pick one of them and set the others to zero. E.g, in the 
extreme case of $k$ identical features, the LASSO arbitrarily picks one of them with, say, 
coefficient $a$; the ridge regression, meanwhile, assigns every of the $k$ features the coefficient 
$a/k$. Relying a very few features comes with the advantage of sparse models, for which we 
can cheaply generate new data, but it is very sensitive to measurement errors in one of the picked 
features; in the latter case, ridge regularization would average out errors with the help of other 
correlated features. Elastic-net regularization balances the sparsity of the LASSO with the 
robustness of the ridge regularization, especially for values of $\alpha$ close to $1$ 
\cite{elasticnet05}.

\paragraph{Feature-wise penalty weights}
We refine the elastic-net regularization one last time and introduce \textit{a priori} defined 
feature weights $v_i \geq 0$ yielding
\begin{align}
    p_{\text{enet}, \alpha}(\beta) = \sum_{j=1}^p v_j \left( \frac{1-\alpha}{2} \beta_j^2 +
    \alpha |\beta_j| \right).
\end{align}
The default value is $w_j = 1$ for all $i = 1, \ldots, p$. Deviating from the default can be useful 
in at least two scenarios. First, sometimes we are so convinced of the predicitve 
power of a feature in our model that we set $v_j = 0$ for it to all but ensure a non-zero 
coefficient for it; the IPI (in some format) could be such a feature. 

\paragraph{Standardizing $X$} 
The second application of the penalty weights is an aspect of a bigger topic: standardizing the
predictor. We can use the weights to standardize $X$ by setting $v_j$ to an estimate of the standard 
deviation of feature $j$. Why 
should we do this? Let us consider two features $X_1$ and $X_2$ with standard deviation $\sigma_1$
and $\sigma_2$, respectively. To change the output of the model by $1$ when the feature changes by 
one standard deviation, $X_1$ needs the coefficient $\beta_1 = 1/\sigma_1$ and $X_2$ needs 
$\beta_2 = 1/\sigma_2$. For $\alpha = 1$, $p_{\text{enet}, \alpha}$ penalizes both coefficients 
equally and it still does so approximately for $\alpha$ slightly below $1$. Standardization strives 
to garuantee equal justice for features on different scales.
Standardization works, however, contrary to the zero-sum idea: thanks to the zero-sum constraint, 
the fitting process and the final model both are invariant under sample-wise shifts on a subset of 
features if we set the zero-sum weights accordingly. Standardization, in contrast, actively changes 
the cost function, which probably results in a different model. Consequently, one should stick to 
the default value $v_j = 1$ for the features for which we wish to have shift-invariance; these 
are usually the gene expression levels, which are on comparable scales anyway.

\subsubsection{Random forests}

More precisely, we talk about classification random forests, hence $Y$ is binary in this subsection 
with $\im(Y) \subset \{ \pm 1 \}$, where $1$ still encodes the positive class. When we say random 
forests in this thesis, we mean classification 
random forests. Random forests are an ensemble of classification trees -- in short: 
trees -- and aggregate the classification of every constituent tree.

\paragraph{Trees}
A tree is a simple function
\begin{align}
    T = \sum_{i=m}^M c_m \chi_{R_m}: \RR^p \to \{0, 1\}
\end{align}
with $c_m \in \{\pm 1\}$ and the $R_m$ being disjoint rectangle sets, i.e. $R_m = \prod_{j=1}^p 
(a_j, b_j]$ for $a_j, b_j \in \RR \cup \{ \pm \infty \}$. It aims to predict the conditional 
majority class 
\begin{align}\label{eq:rf-major}
    \sgn(E(Y \cond X)) = T(X);
\end{align}
$\sgn$ the signum function.

How can we use our samples $(x_i, y_i)$, $i = 1, \ldots, n$, to learn a tree $T$? While we can 
trivially fulfill eq. (\ref{eq:rf-major}) exactly for the samples, this would only lead to terribly 
overfit trees. Instead, we confine ourselves with simpler, rougher trees that will make errors on 
training samples, but generalize better. 
In the algorithm further below, we will only need to calculate an error for the samples inside a 
rectangle set $R \subset \RR^n$. Let $n(R) = |\{ 1 \leq i \leq n: x_i \in R \}|$ be the number of 
samples in $R$ and
\begin{align}
    \hat{p}_R = \frac{|\{ 1 \leq i \leq n: x_i \in R, y_i = 1 \}|}{n(R)}
\end{align}
be the proportion of samples in $R$ with positive outcome. The error measure -- in the context of 
trees usually termed impurity measure -- we use in this thesis
is the Gini impurity $Q(R) = 2 \hat{p}_R (1 - \hat{p}_R)$; it is low for pure $R$ dominated by 
samples with either positive or negative outcome and grows quadratically as the outcome of the 
samples in $R$ gets more and more imbalanced. The impurity measure is a hyperparameter: other 
options include the misclassification error and the cross entropy. To govern the complexity $T$,
we demand that every rectangle set $R_m$ contain at least $n_\text{min}$ samples. Even this 
constraint still leaves with a computationally infeasible number of possible trees, and this is 
why we fit $T$ with a greedy algorithm: it recursively partitions every rectangle set along a 
feature in two subsets in such a way that the impurity measure gets minimal until every resulting 
rectangle set comprises less than $n_\text{min}$ samples. Formalizing binary partioning, for a 
rectangle set $R$, a feature $j \in \{1, \ldots, p\}$ and split point $s \in \RR$, we define 
the pair of half planes
\begin{align}
    R^{(1)}_{j, s} = \{x \in R: x_j \leq s\} \text{ and } R^{(2)}_{j, s} = \{x \in R: x_j > s\}.
\end{align}
With respect to the impurity measure, there are only finitely many split points of interest since 
we only have finitely many samples, and hence we need to consider no more than $n \cdot p$ when 
considering $R$ for partioning. All of this allows us to compactly describe a procedure to grow a 
tree in Alg. \ref{alg:tree}.
  \begin{algorithm}
    \caption{Growing a tree by recursive binary partitioning} \label{alg:tree}
    \begin{algorithmic}[1]
      \Function{Tree}{$(x_i, y_i)_{i = 1, \ldots, n}$; $Q$, $n_\text{min}$}
        \State $\mathcal{R} \gets \emptyset$ \Comment{Final partioning}
        \State $\mathcal{I} \gets \{ \RR \}$ \Comment{Intermediate partioning}
        \While{$\mathcal{I} \neq \emptyset$}
          \State Choose some $R \in \mathcal{I}$
          \State $\mathcal{I} \gets \mathcal{I} \setminus \{ R \}$
          \State $(j_*, s_*) \gets \argmin_{j, s} n\left(R_{j,s}^{(1)}\right) Q\left(R_{j,s}^{(1)}\right) + 
            n\left(R_{j,s}^{(2)}\right) Q\left(R_{j,s}^{(2)}\right)$ \label{alg:tree:split}
          \For{$\ell = 1, 2$}
            \If{$n\left(R_{j_*, s_*}^{(\ell)}\right) < n_{\text{min}}$}
              \State $\mathcal{R} \gets \mathcal{R} \cup \left\{ R_{j_*, s_*}^{(\ell)} \right\}$
              \State $c_R \gets 1$ if $\hat{p}_R > \num{0.5}$ else \num{-1}
            \Else
              \State $\mathcal{I} \gets \mathcal{I} \cup \left\{ R_{j_*, s_*}^{(\ell)} \right\}$
            \EndIf
          \EndFor
        \EndWhile
        \State \Return $\sum_{R \in \mathcal{R}} c_R \chi_R$
      \EndFunction
    \end{algorithmic}
  \end{algorithm}

\paragraph{Forests}
Trees are known to be notoriously noisy, meaning if we fix $x \in \RR^p$ the variance 
$V_\mathbf{z}(T(x))$ is quite high; the index $\mathbf{z}$ makes explicit that we consider $T(x)$ as 
a function -- or better: random variable -- of i.i.d. drawing training samples $\mathbf{z}$ (and 
then fitting a tree to them according to Alg. \ref{alg:tree}). On the other hand, if grown deep,
trees have pretty low bias, meaning for fixed $x \in \RR^p$ the expectation $E_\mathbf{z}(T(x))$ in 
most cases is exactly what we want, $\sgn(E(Y \cond x))$. High variance and low bias makes trees 
ideal candidates for a method called bagging, which averages the predictions of many noisy, 
approximately unbiased models with the goal to reduce their variance. Given trees $T_b$, $b = 1, 
\ldots, B$, we denote the new model by 
\begin{align}
    \overline{T}(x) = \frac{1}{B} \sum_{b=1}^B T_b(x).
\end{align}

Since the expected value is linear, we have 


\subsection{Nested models}\label{subsec:nested-models}

\section{Software}
    \chapter{Results} \label{chap:results}

In this triune chapter, we start with introducing three DLBCL data sets that include 
survival and gene expression features. Next, in intra-trial experiments, we split every of these 
three data sets into 
a train and test cohort, fit models for a variety of hyperparameter tuples in $H$ to the training 
cohort, validate them and test the best on the test cohort; this is less about presenting a 
high-performing model, but more about analyzing validated and tested errors to make $H$ slimmer and 
better for the future. This future plays out in the last part as we train and validate on one of 
the three data sets and test on another and deal with cross-platform variability in inter-trial 
experiments.

\section{The data}

Our DLBCL data sets are taken from papers by Schmitz et al.\ \cite{schmitz18}, Reddy et al.\ 
\cite{reddy17} and Staiger et al.\ \cite{staiger20}; see Table \ref{table:data} for key properties 
and comparison. We will refer to them as Schmitz, Reddy and Lamis test data, respectively.

\begin{table}
    \centering
    \begin{tabular}{lrrr}
        \hline
        & \textbf{Schmitz} & \textbf{Reddy} & \textbf{Staiger} \\
        \hline
        \textbf{prospective trial} & no & no & yes \\
        \textbf{response} & PFS & OS & PFS \\
        \textbf{\# samples} & \num{229} & \num{604} & \num{466} \\
        \textbf{\# genes} & \num{25066} & \num{13302} & \num{145} \\
        \textbf{technology} & RNA-seq & RNA-seq & NanoString \\
        \textbf{high risk [\%]} & \num{36.6} & \num{31.5} & \num{24.3} \\
        \textbf{$\text{prev}(\text{tIPI})$ [\%]} & \num{12.9} & \num{21.6} & \num{17.0} \\
        \textbf{$\text{prec}(\text{tIPI})$ [\%]} & \num{65.2} & \num{54.1} & \num{38.2} \\
        \hline
    \end{tabular}
    \caption{Overview on used data sets. All datasets include the IPI features in thresholded form, 
        gender, cell of origin, and, added by us, the LAMIS signature. The Schmitz and Staiger 
        data use the high-risk definition of MMML-Predict: PFS below two years; 
        because the Reddy data reports only overall, no progression-free survival, we define high risk 
        as overall survival below \num{2.5} years there, leading to a comparable high-risk 
        proportion.}\label{table:data}
\end{table}

\paragraph{Schmitz data}
The data by Schmitz et al.\ includes the five IPI features in their continuous form. 
By heuristically optimizing a novel genetic distinctiveness metric, Schmitz et al.\ 
clustered 574 DLBCL biopsy samples into four subtypes, called MCD, BN2, N1, EZB and ``other''. They 
unblinded the clinical data only after the clustering was complete and the 
model was frozen \cite[Appendix 1, pp. 16--18]{schmitz18}. The following, independent survival analysis 
unveiled significantly differing progression-free survival between the four subtypes (excluding 
``other'') according to a logrank test; BN2 and EZB subtypes have far better prognosis than MCD and 
N1 (2-year PFS rate of \num{81}\% and \num{75}\% as opposed to \num{39}\% and 
\num{20}\%, respectively). The IPI score did not vary significantly between the subtypes, 
indicating that the new classifier gives us additional information to predict survival.

There are two caveats: First, the genetic classifier saw the entire data set during training and 
this runs afoul of a strict train-test regime even if training was 
survival-agnostic. This only affects the intra-trial experiments as in the inter-trial experiments 
we do not include the genetic subtype as a feature. As a more important take away, we always need 
to carefully use features in a data set that are the output of some model trained on this data set. 
Second, this data set is not the result of a 
prospective, representative trial, but opportunistically collected samples from highly renowned 
U.S. hospitals, which preferably treat 
difficult cases. As a result, the high-risk proportion in the Schmitz data is at \num{36.6}\% -- 
compared to \num{24.3}\% in the prospective Lamis test data -- and the IPI reaches a precision of 
\num{65.2}\% for classifying high-risk patients at a prevalence of \num{12.9}\% -- compared to 
\num{38.2}\% at \num{17.0}\% in the Lamis test data. The IPI already meets the MMML-Predict goals, 
but we want to see if we can do even better in a such high-risk regime.

\paragraph{Reddy data}

Compared to a prospective study, also the Reddy data is enriched for high-risk patients and the 
IPI boasts a performance that already satisfies the MMML-Predict goals, even if a bit less 
convincingly than on the Schmitz data. After identifying \num{150} DLBCL driver genes, Reddy et al.\ 
trained a Cox model, termed genomic risk model,
that predicts overall survival (OS) from combinations of genetic events and gene-expression markers 
(cell of origin, MYC, and BCL2) and thresholded it into low, intermediate and low risk. This is a
model whose predictions we cannot use because their split into the train and test cohort is not 
clear, but that inspired us to use combinations of discrete features, cf.\ subsection 
\ref{subsec:model-agnostic}. Nevertheless, with high expression and translocation of MYC, BCL2 and 
BCL6, the data provides some of the input features of the genomic risk model as well as three binary 
clinical features: B symptoms at diagnosis, testicular and central-nervous-system involvement. B 
symptoms refer to the presence of the triad fever, night sweats and unintential weight 
loss.

\paragraph{Lamis test data}

The Lamis test data is composed of \num{466} patients enrolled in prospective clinical trials. 
Staiger et al.\ first determined \num{731} gene pairs with highly correlated gene expression levels 
between their train cohort -- \num{233} DLBCLs with gene expression levels built from the Affymetrix 
GeneChip technology -- and the Lamis test data -- \num{466} DLBCLs with gene expression levels built 
from the NanoString nCounter technology -- with the aid of six paired nCounter-GeneChip samples, cf.\ 
\cite[Supplementary Methods]{staiger20}.
Next, they learned a Cox model on the differences of the (logarithmized) gene expression levels 
from these gene pairs and the five thresholded IPI features using LASSO regularization. 
Afterwards, they removed the five IPI features from the model aiming to make it independent of the 
IPI. One can expand the differences of gene expression levels in the signature to obtain an ordinary 
gene-expression signature with coefficients corresponding to single genes, the Lamis. It is 
based on \num{17} genes, but dominated by just two genes, CSF1 and CPT1A. 

By dichotomizing the 
Lamis scores at the 75\%-quantile into their Lamis group (low or high), Staiger et al.\ present two 
groups on the Lamis test data with significantly differing PFS and OS; 
meanwhile, the IPI features, breaks in MYC, BCL2, BCL6, and cell of origin remained prognostic 
indicators independently of the Lamis group.

Since the Lamis coefficients fulfill the 
zero-sum property, we apply the Lamis unchanged on the two other data sets; according to Eq.\ 
\eqref{eq:inter-tech}, the Lamis scores have a data-set-dependent shift. We also add the Lamis 
group by dichotomizing the Lamis scores at the 75\%-quantile of the respective data set.

\paragraph{Combined data}

We will conduct inter-trial experiments on a big data set consisting of the samples in the Schmitz, 
Reddy and Lamis test data. As for the gene expression features, we map the Ensemble gene IDs used 
in the Reddy data to HGNC gene symbols with the help of the \texttt{BiomaRt} R package \cite{biomart}
to end up with the same gene nomenclature in all three data sets. Intersecting the gene expression 
features leaves us with gene expression levels for \num{119} genes and the features gender, age, 
the five thresholded IPI features, the IPI score, the IPI group, the Lamis score and the Lamis 
group. The resulting data set is \num{1299} samples strong.

\section{Intra-trial experiments}

To gain first insights on our methods, we conduct intra-trial experiments sepratetly on the 
Schmitz, Reddy and Lamis test data. To this end, we split every data set into a train and test 
cohort. We do so uniformly at random, with two constraints: first, a ratio of 3 to 1 between train 
and test cohort and, second, the ratio between high-risk and low-risk patients in train cohort, 
test cohort and overall data set is the same. As shown in Table \ref{table:intra-trial}, the 
performance of the tIPI notably differs between the whole data set and the subsampled test cohort. 

\begin{table}[ht]
    \small
    \centering
    \begin{tabular}{lrrr}
        \hline
         & \textbf{Schmitz} & \textbf{Reddy} & \textbf{Lamis test} \\
            \hline
        \textbf{\# samples} & \num{58} & \num{151} & \num{117} \\
        \textbf{high risk [\%]} & \num{37.0} & \num{31.6} & \num{24.3} \\
        \textbf{(prev./prec.) $\text{IPI} \geq 4$} & (\num{0.170}/\num{0.500}) & (\num{0.192}/\num{0.421}) & (\num{0.139}/\num{0.364}) \\
        \textbf{(prev./prec.) $h^*$} & (\num{0.351}/\num{0.684}) & (\num{0.230}/\num{0.556}) & (\num{0.280}/\num{0.419}) \\
        \textbf{ROC-AUC $h^*$} & \num{0.80} & \num{0.65} & \num{0.66} \\
        \textbf{logrank $h^*$} & \num[scientific-notation=true]{3.69e-4} & 
            \num[scientific-notation=true]{1.82e-3} & 
            \num[scientific-notation=true]{9.38e-4}  \\
        \hline
    \end{tabular}
    \caption{Randomly split a single data set into a train and test cohort; 
        train and validate on the train cohort, test on the test cohort. All 
        numbers refer to the test set.}
\end{table}

\subsection{Model architectures}

We want to give a brief summary of the models we send into the race and take a closer look at the 
best one, $m^*$, on every data set.

\subsubsection{Candidates}

\paragraph{Gene expression levels only}
Models trained in a leave-one-out cross-validation only on the gene expression levels include the 
Gaussian, logistic and Cox model. Regarding noteworthy hyperparameter decisions, we both apply and 
do not apply standardization of the predictor; we do not demand the zero-sum constraint 
because we do not want to transfer our model to other data sets and want to safe computation time;
we regularize with elastic-net penalty factor $\alpha \in \{ \num{0.1}, 1 \}$; as throughout this 
chapter, we have the \texttt{zeroSum} package calculate a sequence of \num{100} regularization strengths 
$\lambda$ for us and stop early if the cross-validated error does not improve for \num{10} 
consecutive decreasing $\lambda$ values; we try out a whole bunch of time cutoffs $T$.

We train and validate a model for every combination of these hyperparameters.
Not taking into account the values of $\lambda$ (which are hard to foresee due to early stopping), 
this adds up to \num{88} models.

Looking at the models with top validated performance (cf. projection upon validation error in Fig. 
\ref{fig:intra-val-test-geo}), for the following we narrow down $T$ to one or 
two values (usually at or slighly below two years) and $\alpha$ to $1$ as the more complex models 
trained for $\alpha = \num{0.1}$ cannot clearly outperform the spare LASSO-trained models. For 
models only predicting from gene expression levels, we do not standardize the predictor any more.

\paragraph{Core models with other features}
We now add all available remaining features: for the 
IPI, we either add the five continuous features (if available), the five threholded features, the 
score as a continuous feature or all of these threee; for the Lamis, % here and for the rest of thesis 
we add the score as a continuous and the group as a discrete feature. For $s_\text{min} = \num{0.05}$ 
and $n_\text{combi} \in \{1, 2, 3, 4 \}$, we add combinations of discrete features to the predictor. 
We both include and exclude gene expression levels in the predictor.

As for the model class, we build on our experience gained from the gene-expression-only models and 
just use the top performing model class of these models for such models that also use gene 
expression levels. For models not predicting from gene-expression levels, we have no prior 
knowledge and thus deploy the full range of core models: Gaussian, logistic, Cox models and random 
forests. As all of these models need to deal with features on different scales, we always 
standardize the predictor.


\paragraph{Nested models}
We nest models according to Alg. \ref{alg:nested-pcv}, where the early model $f_1$ is the best 
validated model among the gene-expression-only models and $f$ has the model class of $f_1$ or is 
a random forest. If $f$ has the same model class as $f_1$, we still tune the LASSO regularization 
strengtht for $f_1$ because we may end up adding several hundreds combined disrete featuers to the 
predictor. As for a-priori feature selection, dito as above.

\subsubsection{Best validated models}

\paragraph{Schmitz and Reddy}
For both the Schmitz and Reddy data, the best model according to the validation error is a nested 
model as in Alg. \ref{alg:nested-pcv} with the early model $f_1$, a Gauss model, predicting from 
the gene-expression levels and the late model $f$ being a Cox model. They only differ in their 
features: For the Schmitz data, $X$ holds the IPI in all its three formats as opposed to the five 
discretized IPI features for the Reddy data. We have $n_\text{combi} = 2$ for Schmitz as opposed to 
$n_\text{combi} = 3$ for Reddy.

On the Schmitz data, $m^*$'s validated precision of \num{89.3}\% drops by more than \num{20} points 
to \num{68.4}\% on the test set. In an even bigger drop, $m^*$'s validated precision of \num{78.6}\% 
declines to \num{55.6}\% on the Reddy test set. These way too optimistic cross-validated errors 
demand further investigation.

\paragraph{Lamis test}
On the Lamis test data, $m^*$ is a very simple model: a logistic model that neither uses gene 
expression levels nor combinations of discrete features ($n_\text{combi} = 1$). Notably, the model
incorporates the Lamis score and group with decisively non-zero coefficient. Here, $m^*$'s 
validated precision of \num{58.2}\% is more in line with the \num{45.9}\% on the test set.

\subsection{Meta analysis}

Now that we have frozen all models and do not add further ones, we can unlock the test set and 
evaluate the test performance of more models. Our main focus rests on the discrepancy between 
validated and test error; as we laid out in section \ref{sec:train-val-test}, the models with the 
best validated errors most likely have an underestimated test error and it is hard to fight this 
effect qualitatively. However, we use several methods to calculate a validated error -- cross 
validation, the pseudo cross validatio of Alg. \ref{alg:nested-pcv} and the OOB error of random 
forests -- and for every model architecture we tune different hyperparameters and a different 
number of hyperparameter tuples. Both validation error and test error of a model are random 
variables and it is hard to infer statistical properties of them based on a single realization of 
these random variables; by grouping the models according to their hyperparameters we gain 
statisticial power and look for patterns in the validation and test error as well as 
discrepancies between them.

\subsubsection{Models predicting from gene expression levels only}

\begin{figure}
    \centering
    \includegraphics[width=\textwidth]{../../results/intra_val_test_geo.pdf}
    \caption{Validation versus test error in intra-trial experiments for models only predicting 
        from gene expression levels for all 
        three data sets in the rows. The left and right column show the same points, but highlight 
        different groupings of the models. The dashed gray line is the 
        identity line. The training survival cutoff $T$, 
        $n_\text{combi}$, elastic-net regularization strength $\lambda$ (for GLMs), 
        $m$ and $n_\text{min}$ (for RFs) have already been optimized with respect to the 
        validation error. The 
        validation error of the $\text{tIPI}$ is in fact also a test error on independent data, 
        namely our training cohort.}
    \label{fig:intra-val-test-geo}
\end{figure}

\paragraph{Lamis test}
Looking at Fig. \ref{fig:intra-val-test-geo}, we see that, for Lamis test, the models only using 
gene expression levels cannot compete with $\text{tIPI}$. While the validated errors of these models are 
below the \num{-36.4}\% negative precision of the IPI, the test errors all are above \num{-30}\%. 
In all cases, the test error is at least 10 points higher than the validated error, a discrepancy 
also indicated by the fact that the identity line is outside the plot area. It is hard to imagine that 
these models even if nested into another model can help outperform the IPI. Hence, for the 
following, the analysis focuses on the other two data sets.

\paragraph{Schmitz and Reddy}
Also for the Schmitz and Reddy data, the dependence between validated and test error is far from 
being monotonic; in both cases, the plot looks noisy and the correlation between validated and test 
error is negative such that the model with the lowest validated error happens to have the highest 
(Schmitz) or second highest (Reddy) test error. The plots do not show errors for all tried out 
hyperparameter tuples in $H$, but only for a high-performing subset with $\lambda$ and $T$ already 
optimized; this optimization was successful in the sense that even the worst models boast a test 
error below that of $\text{tIPI}$.

There is no model class reliably outperforming the others. Both data sets suggest that ridge 
regularization yields models with a test error lower than that of LASSO regularization; validation, 
however, does not reveal this difference. The picture on standardizing $X$ or not is mixed: on the 
Schmitz data, models with non-standardized predictor fluctuate less around the identiy line than 
those with standardized predictor, meaning validation is more reliable for the former model group;
on the Reddy data, meanwhile, we the same phenomenon with roles swapped.

\subsubsection{Models with the full range of features}

\begin{figure}
    \centering
    \includegraphics[width=1.0\textwidth]{../../results/intra_val_test_more.pdf}
    \caption{Validation versus test error in intra-trial experiments for models predicting from the 
        full range of a priori selected features for all three data sets in the rows. The dashed 
        gray line is the identity line. The notation 
        $c_1$-$c_2$ encodes a model of class $c_1$ as the early model $f_1$ nested into a model of 
        class $c_2$ as the late model $f$ according to Alg. \ref{alg:nested-pcv}.
        ``GE'' is an acronym for gene expression and ``RF'' for random forest. Both columns show 
        the same points, but highlight different groupings. The elastic-net 
        regularization strength $\lambda$, training survival cutoff $T$ and the random-forest 
        hyperparameters $B$, $m$ and $n_\text{min}$ have already been tuned. The validation error 
        for the $\text{tIPI}$ is in fact also a test error on independent data, namely our train 
        cohort.}
    \label{fig:intra-val-test-more}
\end{figure}

All analysis here is based on Fig. \ref{fig:intra-val-test-more}.

\paragraph{Schmitz and Reddy}
Again, we analyze the non-prospective, high-risk heavy Schmitz and Reddy data together. As we apply
the full range of a priori selected features, validation and test error for these two data sets 
stay out of touch. Partioning according to the model architecture reveals some patterns. 

Models 
nesting a Gauss model into a GLM as in Alg. \ref{alg:nested-pcv} all have low validated errors, on 
the Reddy data, they even have lower validated error than any other model; the test error is 
always higher than the validated error and usually it is \textit{much} higher. Alg. \ref{alg:nested-pcv} 
yields way too optimistic validated errors if the late model is a GLM, prompting us to no longer 
fit such models in the following experiments. Discarding Alg. \ref{alg:nested-pcv} once and for all
would go too far: when nesting a Gauss model into a random forest, we get validated errors more in 
line with test errors, especially on the Schmitz data. On both data sets, OOB predictions for 
models consisting of a random forest alone very well estimate the test error. In all but one case 
they even underestimate the test error. All in all, OOB predictions seem superior to 
cross-validated predictions, which comes as no surprise when having a closer look at both 
methods: A sample's cross-validated predictions come from one model and we have one model per 
sample in the best case, a leave-one-out cross-validation (which we did for all models involved in 
nested models). In contrast, a sample's OOB prediction comes from a whole forest of models with 
expected size roughly $B/3$ because the probability of a sample not being in a bootrap sample 
is
\begin{align}
    \left( 1 - \frac{1}{n} \right)^n \to \frac{1}{e} \approx \frac{1}{3} \quad \text{as } n \to 
    \infty,
\end{align}
and we can scale it by scaling $B$, which is cheap as we have laid out in subsection 
\ref{subsec:elastic-net}.

As for feature selection, we see that including gene expression levels does not lead to greatly 
improved models. Even the high-performing gene-expression-only models do not benefit very much if 
we train them with additional features. On the Reddy data, the points belonging to models not using 
gene-expression data are all close to or underneath the identity line, meaning the validation errors 
catch the test errors well, their test errors are less widespread and on average lower than those 
of the remaining models. This does not mean that gene expression levels do not provide important 
information: all of the models that do not use the gene expression levels directly from our data 
do so indirectly thanks to the Lamis. We have simply outsourced developing a gene-expression based 
model to the Lamis authors, thereby getting rid of $p \gg n$ and ending up with a more accurate 
validation.

\paragraph{Lamis test}

On the Lamis test data, the validated errors align better with the test errros. As with the two other 
datasets, OOB-based test-error estimates are conservative leading to underestimated test errors 
for models incorporating a random forest. As a result of highly correlated validation and test 
errors, the best validated model is also the best tested model. Compared to above, the validated errors 
of models nesting a Cox model into a GLM overestimate the test errors less; since the early Cox 
model yields pretty inaccurate cross-validated predictions, we hypothesize that the training 
algorithm of the late GLM is less inclined to put as much weight on these predictions as with the 
two other data sets.

Models profit a lot from not just predicting from the gene expression levels: all the 
gene-expression-only cluster into a bulk with test errors higher than those of all other models. 
Key to the success is again the molecular information condensed in the Lamis, which all the 
high-performing models put heavy weight on.

\section{Inter-trial experiments}

The results of the intra-trial experiments are encouraging: we beat the precision of the IPI by 
at least \num{7} points on three data sets, including two where the IPI already did a good job. 
In the following analysis, we saw validated errors often being detached from test errors. To tackle 
this issue, we decided to no longer train certain models. We also noticed that the Lamis is at least 
as capable of catching the molecular information in the gene expression levels as the models
we trained for that purpose ourselves, but allows training more precise models and validating them 
more accurately. Another way to close the validation-test gap is to increase the number of samples 
in both the train and test cohort: it both makes overfitting the validated predictions to the 
training cohort harder and makes it less likely that the test cohort includes cases biologically 
not covered at all in the train cohort. We will now go along this path as we use every of the 
three data sets in their entirety for training and validation and then test on the other two.

\subsection{Model architectures}

Here, we want to sketch the architectures of the candidate models and take a closer look at the 
best validated model $m_i^*$ trained on every data set.

\subsubsection{Candidates}

Even with combined discrete features, the number of features the model in this section predict from 
never exceeds \num{300} meaning computation is comparatively cheap. Consequently, we can afford 
a leave-one-out cross-validation for every model that is not a random forest and \num{1000} trees 
for every random forest.

\paragraph{Gene-expression levels only}
This time, we demand that every Gauss, logistic and Cox models fulfill the zero-sum constraint for 
the gene-expression levels it deals with as features since every data set has its own protocol for 
measuring gene-expression levels. We have gene-expression levels for \num{119} genes in the 
combined data and these genes are a subset of the \num{145} genes for which the Lamis test data 
includes gene-expression levels; even though the gene-expression only models trained on the Lamis 
data disappointed in both validation and testing, we once more train LASSO-regularized Gauss, 
logistic and Cox models predicting only from gene-expression levels hoping they can exploit the 
improved data situation with more samples. In line with the zero-sum idea, we do not standardize 
the predictor and have $T$ range from \num{1} to \num{2.6} years in steps of \num{0.2} years for 
models trained on the Schmitz and Lamis-test samples; for the Reddy samples, here and for the rest 
of this section, we shift $T$ by
\num{0.5} years to the future accounting for the fact that on the Reddy data, absent PFS, we need 
to classify OS below \num{2.5} years as opposed to PFS below \num{2} years.

\paragraph{Core models with other features}
As in the intra-trial experiments, we generously add all remaining features to the predictor. Only 
for the Lamis, we vary its format and add just the score, just the group or both. We augment the 
predictor with combinations of discrete features according to $s_\text{min} = \num{0.05}$ and 
$n_\text{combi} \in \{ 1, 2, 3 \}$. We restrict the training survival cutoff $T$ to range between 
\num{1.4} and \num{2} years in steps of \num{0.2} years for the Schmitz and Lamis test data. We 
both include and exclude the expression levels in the predictor of the Cox and logistic 
models we fit for all combinations of these values for the hyperparameters.

Random forests have trouble dealing with features systematically shifted features between data sets, so 
we always exclude the gene-expression levels from the predictor and of the Lamis formats only 
include the Lamis group in the predictor. Moreover, random forests can realize combinations of 
categorical features themselves, so we set $n_\text{combi} = 1$ for them.

\subsubsection{Best validated models}

Table \ref{table:inter-trial} presents the performance of the best models validated and trained on 
every cohort, $m_i^*$, $i \in \{ \text{Schmitz}, \text{Reddy}, \text{Lamis test} \}$. We now cycle 
through $i$.

\begin{table}
    \centering
    \begin{subtable}{\textwidth}
        \centering
        \begin{tabular}{lrrr}
            \hline
            & \textbf{Schmitz} & \textbf{Reddy} & \textbf{Lamis test} \\
            \hline
            \textbf{Schmitz} & \num{12.9}, \num{65.2} & \num{17.7}, \num{59.6} & \num{17.1}, \num{50.7} \\
            \textbf{Reddy} & \num{17.8}, \num{71.1} & \num{21.6}, \num{54.1} & \num{18.0}, \num{53.2} \\
            \textbf{Lamis test} & \num{17.4}, \num{75.7} & \num{22.5}, \num{50.4} & \num{17.0}, \num{38.2} \\
            \hline
        \end{tabular}
        \caption{Prevalence and precision.
            Diagonal entries $(i, i)$ hold prevalence, precision of the $\text{tIPI}$ on 
            cohort $i$. Off-diagonal entries $(i, j)$ hold prevalence, precision on cohort $j$ of 
            the best model trained and validated on cohort $i$, $m_i^*$.}\label{subtab:inter-prev-prec}
    \end{subtable}

    \vspace{0.5cm}
    \begin{subtable}{\textwidth}
        \centering
        \begin{tabular}{lrrr}
            \hline
            & \textbf{Schmitz} & \textbf{Reddy} & \textbf{Lamis test} \\
            \hline
            \textbf{Schmitz} & \num{65.2} & \num{48.6} & \num{38.7} \\
            \textbf{Reddy} & \num{54.1} & \num{54.1} & \num{41.5} \\
            \textbf{Lamis test} & \num{58.5} & \num{40.9} & \num{38.2} \\
            \hline
        \end{tabular}
        \caption{Lower limit of \num{95}\%-confidence interval of the precision.
            Diagonal entries hold the precision of the $\text{tIPI}$ on cohort $i$. Off-diagonal 
            entries hold the lower limit of the \num{95}\%-confidence interval of the precision on 
            cohort $j$ of the best model trained and validated on cohort $i$, $m_i^*$.}
            \label{subtab:inter-prec-ci}
    \end{subtable}

    \vspace{0.5cm}
    \begin{subtable}{\textwidth}
        \centering
        \begin{tabular}{lrrr}
            \hline
            & \textbf{Schmitz} & \textbf{Reddy} & \textbf{Lamis test} \\
            \hline
            \textbf{Schmitz} & \num{1.61}, \num{1.37}-\num{1.91}, \num{0} & 
                \num{1.62},\num{1.45}-\num{1.80}, \num{0} & 
                \num{1.59}, \num{1.39}-\num{1.83}, 0 \\
            \textbf{Reddy} & \num{53.36}, \num{18.73}-\num{152.08}, 0 & 
                \num{1.58}, \num{1.40}-\num{1.78}, \num{0} & 
                \num{23.87}, \num{9.89}-\num{57.61}, 0 \\
            \textbf{Lamis test} & \num{1.46}, \num{1.32}-\num{1.60}, 0 & 
                \num{1.32}, \num{1.24}-\num{1.41}, \num{0} & 
                \num{1.46}, \num{1.25}-\num{1.70}, \num{0} \\
            \hline
        \end{tabular}
        \caption{Hazard ratio, its \num{95}\%-confidence interval and p-value for the null 
            hypothesis of the hazard ratio being equal to one. Diagonal entries show these 
            properties for the $\text{tIPI}$, off-diagonal entries for the best model trained and 
            validated on cohort $i$, $m_i^*$. All p-values are below 
            \num[scientific-notation]{5e-6}.}
            \label{subtab:inter-hr}
    \end{subtable}
    \caption{Statistics of inter-trial experiments. Rows $i$ always refer to the training cohort, 
        columns $j$ to the test cohort. Diagonal entries $(i, i)$ hold some statistic about the 
        IPI.}
    \label{table:inter-trial}
\end{table}

\paragraph{Schmitz}
$m^*_\text{Schmitz}$ is a Cox model predicting from discrete features and combinations of up to 
three of them; in particular, no gene-expression levels are directly involved and the model confines 
itself with the Lamis \text{group}. All in all, the sparse model uses \num{7} features. During 
training, we provided the model with a fairly low survival cutoff of \num{1.4} years.

When tested on the Reddy data, $m^*_\text{Schmitz}$ beats the IPI's precision by 5 points. This 
is not enough to show significant superiority in the sense that the \num{95}\%-confidence interval 
of our model's precision does not include the precision of the IPI.

On the Lamis test set, meanwhile, $m^*_\text{Schmitz}$ accomplishes all objectives: with 
\num{50.7}\% precision, it tops the IPI's precision by more than 12 points, narrowly surpassess the 
pychologically important \num{50}\% threshold, and the \num{95}\%-confidence interval of our model's 
precision excludes the IPI's precision. It is remarkable that the $m^*_\text{Schmitz}$, coming from 
the high-risk, non-prospective, RNA-seq environment of the Schmitz data with the Lamis measured with the 
RNA-seq technology, well adapts to the prospective, NanoString regime of the Lamis test data and 
does so better than the models we trained in the previous section on part of the Lamis test data.

\paragraph{Reddy}
$m^*_\text{Reddy}$ looks quite similar to $m^*_\text{Schmitz}$. Instead of the Lamis group, the 
logistic model uses the Lamis score making it its only continuous feature. Its predictor contains 
combinations of up to two discrete features. Hyperparameter tuning yielded a training survival 
cutoff of \num{2.3} years, somewhat below the \num{2.5} cutoff separating patients into high-risk 
and low-risk DLBCLs. 

On the Schmitz data, $m^*_\text{Reddy}$ is more precise than the IPI, but fails to outperform the 
IPI significantly. It yields a fairly high hazard ratio of \num{53.4} 
(\num{95}\%-CI \num{18.7}-\num{152.1}).

On the Lamis test set, $m^*_\text{Reddy}$ boasts a precision of \num{53.2} that is by \num{2.5} 
points higher than that of $m^*_\text{Schmitz}$ and by \num{15} points higher than that of the IPI. 
Consequently, the lower boundary \num{95}\%-confidence interval of its precision at \num{41.5}\% is 
clearly above the IPI's precision and the hazard ratio rises to a considerable \num{23.9}
(\num{95}\%-CI \num{9.9}-\num{57.6}). Even more than $m^*_\text{Schmitz}$, $m^*_\text{Reddy}$ 
defies systematic differences between data sets: comparing the Reddy to the Lamis test data, we 
notice the contrasts non-prospective versus prospective, 
RNA-seq versus NanoString nCounter technology to measure gene expression levels and, most strikingly, 
classifying OS below \num{2.5} years versus PFS below \num{2} years.

\paragraph{Lamis test}
The simplest of the three picked models is $m^*_\text{Lamis test}$. A Cox model, it predicts from 
the Lamis score and five more discrete features -- no combinations of discrete features involved, 
$n_\text{combi} = 1$. With $T = \num{1.4}$, it uses the same low training survival cutoff as 
$m^*_\text{Schmitz}$.

On the Reddy data, our model's precision at \num{50.7}\% stays below the IPI's at \num{54.1}\%. 
Speaking for the whole thesis, this renders the Reddy data a challenging test cohort with a 
hard-to-beat IPI, but -- as we have just seen -- a very helpful training cohort.

Meanwhile on the Schmitz data, the precision of $m^*_\text{Lamis test}$ at \num{75.7}\% exceeds that 
of the IPI by more than \num{10}\%, but falls short of outrivaling the IPI significantly. Still, 
this demonstrates that transferring models between the Schmitz and Lamis data works in both 
directions.

\subsection{Meta analysis}

We again freeze all of our models and unlock the test for all of them to analyze the discrepancy 
between validated and test error and to gain insights on how stable thresholding the models on the 
test cohort is.

\subsubsection{Validated and test errors}

\begin{figure}
    \centering
    \includegraphics[width=1.0\textwidth]{../../results/inter_val_test.pdf}
    \caption{Validation versus test error in inter-trial experiments. The dashed gray line is the 
        identity line. The elastic-net regularization strength $\lambda$, training survival cutoff 
        $T$, $n_\text{combi}$ and the random-forest hyperparameters $B$, $m$ and $n_\text{combi}$ 
        have already been tuned. The validation error for the $\text{tIPI}$ is in fact also a test 
        error on independent data, namely our train cohort.}
    \label{fig:inter-val-test}
\end{figure}

Fig. \ref{fig:inter-val-test} compares validation to test error for all candidate models 
participating in the inter-trial experiments; note that again several hyperparameters have already 
been optimized. 

We have emphasized that the risk profiles of the three data sets differ, especially when comparing 
the Schmitz and Reddy to the Lamis test data, so we cannot expect the validation-test-error tuples 
to perfectly align at the identity line. The motivation to increase the sample size was rather 
to observe test errors that are more monotonic in their validation errors. Indeed, Fig. 
\ref{fig:inter-val-test} shows test errors that way more correlated with their validation errors 
than in the intra-trial experiments. As a result, the model with minimal validation error in all 
cases is close to being the one with minimal test error. In more detail, with the exception of 
the models trained on the Lamis test and tested on the Reddy data, the precision of the best 
validated model and that of the best tested model deviate by no more than \num{2}\% points and in 
three of the six cases the best validated and tested model coincide. This is a solid foundation 
to train and validate more models.

As for the model class, we see that no model class prevails clearly. Ignoring the tIPI, the random 
forest always finishes last in validation and never reaches a top position in testing; in fact, 
it claims the last testing place in four out of the six plots. At the top positions, Cox and 
logistic models compete closely: Among the models trained on the Reddy data, a logistic model 
narrowly secures the pole position in validation; testing on both the Schmitz and Lamis test 
data reveals how crucial this victory was as the closest validation competitors of 
$m^*_\text{Reddy}$ cleary lag begind in terms of the test error. Among the models trained on the 
Lamis test data, the same holds true with roles of Cox and logistic swapped. This shows that it 
was crucial to try out both Cox and logistic models although they are closely related; thanks to 
more correlated validation and errors, we need to worry less about overfitted validated 
predictions and might even consider sending analogous Gauss models into the race.

Regarding the question if the predictor should directly contain the gene-expression levels or not,
Fig. \ref{fig:inter-val-test} speaks a plain language. In testing, the top ranking models are 
always such models that do not use gene-expression levels; often the test errors of the models whose 
predictor excludes gene-expression features separate from the rest (cf.\ plots A.1, A.2, B.1). In 
validation however, this distinction can be less obvious, especially for those models trained on 
the Reddy data (cf.\ plots B.1, B.2). In summary, we suffer a lot from the curse of high dimensions 
if we include gene-expression features in the predictor, but gain nothing in terms of test 
performance. Indeed, the models that make the choice of the best validated model on the Reddy data 
close and lucky with regard to testing all use gene-expression features. As the Lamis does a very 
good job at capturing gene-expression information and was trained on an independent data set, we 
conclude that we should just include the Lamis -- and maybe other widely accepted molecular 
signatures -- as an ordinary feature in the predictor and leave aside any other gene-expression 
levels.

\subsubsection{Output thresholding}
    \chapter{Discussion} \label{chap:discussion}

The previous chapter showed that our best models can significantly outperform the IPI in both 
a high-risk and representative-risk setting. Moreover, we demonstrated that we can reliably validate 
our trained models and pick a near-optimal if not optimal one if we restrict the methods from 
chapter \ref{chap:methods} appropriately and increase the sample size.

\section{Heuristics for candidate models}

Key to ensuring realistic validated errors is excluding those candidate models from $H$ that sneak 
in a validated error considerably below their test error. In this thesis, we developed a series 
of heuristics that aim to do precisely that.

\paragraph{Nesting models}
We have seen that we should nest models predicting from hundreds or even thousands of 
gene-expression levels together with more features into another model with care. Our way to do 
this, which uses the cross-validated predictions of the early model to train the late model, 
delivers way too optimistic validated errors for the late model. The early model, usually involving 
much more features than training samples, has plenty of freedom to overfit and it will exploit this 
freedom when we tune its hyperparameters in a cross-validation. The algorithm of the late model then 
recognizes the output of the early model as a very predictive feature and is misled to make it 
a prominent feature in the final model. On independent test data, the input for the late model 
systematically differs from the training, causing it to generalize poorly.

Instead, we need to present the late model training data identically distributed to the test 
data. An easy and successful way to this is to not train and tune the early model ourselves, but have 
other people do this on an independent data set. In other words, we use already-existent signatures. 
Specifically, we did this with cell of origin and the LAMIS. The LAMIS fulfills the 
zero-sum constraint such that transferring it to a new data set with gene-expression levels 
measured under another protocol only results in a constant shift. When using the output of such 
a signature as a feature for one of our models, we have two options regarding its format.
First, we can threshold the signature -- as with cell of origin into ABC and GCB or the LAMIS into 
a high and low group --, thereby rendering the signature protocol-independent and ready to input 
into any late model; this, of course, also works with violated zero-sum constraint and 
$m^*_\text{Schmitz}$ from section \ref{sec:inter-trial} 
does this quite successfully. Second, we can leave the signature output untouched and provide it as a 
continuous feature to a GLM, for whose linear predictor a constant shift in an input feature results 
in constant shift of its output. The most convincing model of this thesis, $m^*_\text{Reddy}$ does 
this with the LAMIS score.

\paragraph{Gene-expression levels}
As a result, we refrain from incorporating gene-expression levels directly into the models trained 
by us, but only use their valuable information condensed into the output of already-existent models. 
With the curse of high dimensions gone, we have observed much more trustworthy validated errors. 
Furthermore, this reduces the training time and enables us to use more complex models.

\paragraph{More heuristics}

The inter-trial experiments suggest that model performance benefits from using a lower training 
survival cutoff $T$. Random forests, whose OOB predictions proved to yield a very accurate estimate 
of the test error, fare worse than GLMs; unlike GLMs, they cannot deal with systemically shifted 
features, which makes them unsuited to handle the continuous output of gene-expression signatures
across protocols.

\section{Increasing the sample size}

In addition to restricting the hyperparameter-tuple space $H$ carefully, we raised the number of 
samples by combining three data sets. We hoped to get both more representative train and test 
cohorts with less systematic differences and hence turn validated errors into better estimates 
for test errors. Whatever the reason, in the latter we succeeded. To combine the data, we had to 
make sacrifices: we discarded the features that are 
not present in all three data sets -- and this included valuable features like MYC translocations -- 
and we often trained models to predict PFS and then tested them on OS or vice versa. Still, our 
best validated models defied this as well as systematic, especially technological and prognostic 
differences between the data sets and, at roughly \num{15}\% prevalence, outperformed the IPI 
significantly on the prospective Staiger data.

In the inter-trial experiments, with more samples in the test cohort, we could lower the prevalence 
of our selected models to \num{10}\%, drastically raise the precision and still retain enough 
statistical power to significantly defeat the IPI in even more cases. For the best validated models, 
we have seen an overwhelmingly monotonic relationship between the prevalence and the precision, 
which made the \num{15}\%-quantile or -- even better -- \num{10}\%-quantile of the model output a 
near-optimal threshold. 

The results of this thesis strongly suggest that sample size is a crucial factor 
for our problem. Furthermore, we should be ready to make sacrifices -- less features, 
even a different kind of the response -- to increase the number of samples in both train and test 
cohort. 

\section{The future of MMML-Predict}\label{subsec:discussion-mmml}

With $m^*_\text{Schmitz}$ and even more $m^*_\text{Reddy}$, this thesis presents two 
models that meet all requirements of MMML-Predict: a prevalence of at least \num{10}\% and a 
precision above \num{50}\% and significantly above the IPI on an independent, prospective data set. 

And still, every percentage point we gain in prevalence will convince more clinicians, researchers 
and pharmaceutical companies to pay attention to the high-risk group identified by the MMML-Predictor.
Every percentage point we gain in precision will not just spur more interest of the above people and 
companies, but will also convince more patients to let the MMML-Predictor guide their treatment and 
will avoid both unnecessary and failed therapies. This thesis did not suggest that model performance 
has already saturated, so we can go for even more in the future.

How might we do this? This thesis demonstrated that we can use data from already-existent trials, 
even if they are not prospective and differ technologically, to train and validate a model and 
then successfully test it on data from a prospective trial as MMML-Predict is about to conduct one.
In the intra-trial experiments, we saw validation and test performance out of touch. We suspect 
that too small, therefore non-representative and systematically differing train and test cohorts 
could have played a key role in this. For the final MMML-Predictor, one might thus consider 
using more than \num{100} of the \num{300} samples registered for MMML-Predict, maybe even all 
\num{300}, for a sufficiently large, statistically more powerful test cohort. One might combine 
a series of available data sets into a big training cohort. With combining data sets come two 
caveats. 

First, we need to intersect over the sets of features, which amounts to discarding plenty 
of features. Nevertheless, many modern DLBCL data sets include gender, age, the five thresholded 
IPI features as clinical features and gene-expression levels as molecular features. More modern data 
sets also hold the double-hit and triple-hit status, which refers to the simultaneous translocation 
of MYC and either BCL2 or BCL6, or MYC, BCL2 and BCL6, respectively. From gene-expression 
features, we can calculate already-existent molecular signatures as well as the double-expressor and 
triple-expressor, which describe the phenomena analogous to double- and triple-hit status with 
overexpression instead of translocation of the three involved genes.

Second, we need to take care of batch effects. Among the features mentioned above, this concerns 
the gene-expression levels. The first option is to get rid of protocol effects by once and for all
thresholding the output of molecular signatures. Most signatures come with canonical thresholds, 
such as the group for the LAMIS, and we can calculate them on the respective data set even before 
combining. In this case, we can deploy quite any model as we no longer need to fight the curse 
of high dimensions or batch effects. The second option is to only apply zero-sum signatures (or 
scale the gene-expression levels to an $\ell_1$ norm of \num{1} across all samples). Using the 
output of these signatures as continuous features of a GLM again results in a protocol-dependent 
shift in output of the GLM. For training, we can add the data set the sample was taken from as a 
categorical 
feature to the predictor so we can correct for the data-set-dependent shifts in the loss function. 
After training, we remove the data-set feature from the model. In testing, 
we threshold the continuous output of the GLM at the $\alpha$-quantile for some $\alpha$, as we did 
in this thesis. A shortcoming of this testing procedure is that we always need a sufficiently large 
test cohort to be able estimate the $\alpha$-quantile reliably. An alternative approach might 
involve internal standards, i.e.\ a small number of samples one has measured for a test cohort for 
which we already know a good threshold and that we can measure again for any new protocol to shift 
the threshold accordingly. Even for MMML-Predict, this is a problem for the distant future.

Our last idea concerns the sample weights in loss functions. Almost every loss function is derived 
from the log likelihood of i.i.d.\ samples and thus sums over the training samples. Therefore, they 
offer to weight every summand with a sample weight, as we have seen in the loss functions of the 
presented GLMs in Eq.\ \eqref{eq:loss-glm-no-lasso} and \eqref{eq:cox-log-lh}. One can even provide 
sample weights for random forests. We have a huge amount of 
freedom in how we choose the sample weights and trying out too many choices risks torpedoing 
validation. We will now describe a single, natural choice. Let $q$ denote the 
proportion of high-risk samples in the training data. In the prospective Staiger data, we had 
$q = \num{24.3}$, which renders our classification problem imbalanced. Setting $w_i = 1/q$ for 
high-risk and $w_i = 1/(1-q)$ for low-risk samples perfectly balances the classification task in 
the sense that the sum of sample weights belonging to high-risk samples equals the sum of sample 
weights belonging to low-risk samples.

Looking back at this thesis, we can say with confidence: the state of MMML-Predict is strong and 
it is getting stronger.
    \bibliography{../lit.bib}

    \newpage
    \section*{Declaration of authorship}
    \begin{german}
        Ich habe die Arbeit selbständig verfasst, keine anderen als die angegebenen Quellen und 
        Hilfsmittel benutzt und bisher keiner anderen Prüfungsbehörde vorgelegt. Außerdem 
        bestätige ich hiermit, dass die vorgelegten Druckexemplare und die vorgelegte elektronische 
        Version der Arbeit identisch sind, dass ich über wissenschaftlich korrektes Arbeiten und 
        Zitieren aufgeklärt wurde und dass ich von den in § 26 Abs.\ 5 vorgesehenen Rechtsfolgen 
        Kenntnis habe. 
    \end{german}

        \vspace{1.5cm}
        \noindent Regensburg, \today \hfill \underline{\hspace{6cm}}

\end{document}
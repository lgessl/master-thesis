\chapter{Methods} \label{chap:methods}

Finding the best possible model for our given task will not be possible from just theoretical 
considerations; we will have to fit several models to our data and demonstrate the performance of 
the chosen one convincingly. Section \ref{sec:train-val-test} will lay the state-of-the-art 
train-validate-test paradigm for this. In section \ref{sec:candidate-models}, we will introduce 
the candidate models and the hyperparameters governing their fitting process. We will start with 
model-agnostic hyperparameters before we go on to present well-known model types and finally 
introduce a method that lets us train compositions of multiple models.

\section{Training, validation and testing}\label{sec:train-val-test}

The design of MMML-Predict with its train and test cohort, where the people developing the predictor 
never get to see the test cohort, pays tribute to the standard two-step approach in supervised-
learning tasks: We use the train cohort to fit multiple models (training) and choose the model
among these models that we have confidence performs best on new data (validation). We then touch 
the test cohort for the first time as we evaluate the chosen model on it to persuade the outside 
world we have come up with model worth deploying.

Since we want to have the same conditions for data used in this thesis as later in MMML-predict,
we randomly split given data $(\mathbf{x}, y)$ with predictor $\mathbf{x}$ and response $y$ into 
a train cohort $(\mathbf{x}_\text{train}), y_\text{train}$ and a test cohort $(\mathbf{x}_\text{test}, y_\text{test})$.
The following assumes a single, split data set.

\subsection{Training and validation}\label{subsec:train-val}

To be able to discuss some probabilistic caveats of validation later, we introduce some formal 
notation. We start with a set of tuples of hyperparameters $H$, where every $h \in H$ defines a model
up to its parameters; determining the parameters of a model, by definition, is the job of the 
algorithm optimizing a given loss function, we refer to this as fitting. There is a one-to-one 
mapping between $H$ and the set of candidate models.

For every hyperparameter tuple $h \in H$, we
\begin{enumerate}
    \item fit the model to the train cohort subject to $h$ in $k$-fold cross-validation. This means 
        we randomly assign the training samples into $k$ subsets, called folds, and then actually 
        fit $k+1$ models, one model on all samples and for every $i = 1, \ldots, k$, we train a model
        on all samples except the $i$-th fold and obtain its cross-validated predictions on the 
        $i$-th fold; these are predictions on new data. Doing this for all folds, we obtain a 
        cross-validated prediction for every sample and hence a vector of cross-validated predictions
        $\hat{y}_\text{train} = \text{cv}(h)$ of the same shape as $y_\text{train}$. There may be 
        other methods than ordinary cross
        validation in this step, but their purpose is always the same: yield a prediction for every 
        sample that looks like it was made on new data.
    \item We use the cross-validated predictions to calculate the validation error 
        $\text{err}(y_\text{train}, \hat{y}_\text{train})$.
\end{enumerate}

Finally, we select the model $m^*$ defined by the hyperparameter tuple 
\begin{align}
    h^* = \argmin_{h \in H} \ \text{err}(y_\text{train}, \text{cv}(h)).
\end{align}

We note that for all $h$, the validation error is a random variable, which e.g. depends on the 
random assignment of the fold, the train cohort and possibly random involved in the loss-function
optimizer. It is a well known property of independent,
identically distributed (i.i.d.) real random variables $X_i, i \in \NN$, that their extreme values 
are notoriously unstable in the sense that for all $t \in \RR$ with $P(X_1 \geq t) < 1$, 
\begin{align}
    \Pr\left( \min_{1 \leq i \leq n} X_i \geq t \right) = \Pr(X_1 \geq t)^n \to 0 \quad
    \text{as } n \to \infty.
\end{align}

While not all $\text{err}(y_\text{train}, \text{cv}(h)), h \in H$, are i.i.d. -- after all, they 
all depend on 
$(\mathbf{x}_\text{train}, y_\text{train})$ and some model are better suited for the problem than others --,
we can have situations where at least a subset of $H$ delivers validation errors close to being i.i.d.

\paragraph{Models merely guessing}
Imagine models not suited to deal with the problem at all. Their cross-validated predictions are 
merely guesses on $y_\text{train}$ and therefore i.i.d. There is a non-zero probability to guess
all true outcomes on the train cohort correctly and hence it is only a matter of the number of 
such merely guessing models until which get the minimal possible error for one of them. The number of 
such models can indeed grow large e.g. if we keep trying ever new hyperparameters for a model class 
that can per se not describe the dependence between $\mathbf{x}$ and $y$ at all.
As a practical advice, we should not spend too much time und put too much hope into improving 
bad models by hyperparamter tuning.

\paragraph{General scenario}
In general, the validation errors of the candidate models are independent to some extent because we fit them
subject to differing hyperparameters. As a result, the validation error of every model can 
fluctuate freely around its expected value to some extent. The larger $H$ gets, the 
more severely underestimated validation errors we will have in the ranking. This 
increases the odds that $m^*$ is just a model with a tremendously underestimated validation error, 
leading to a bad surprise on the test cohort. As a practical advice, we should be careful with 
trying too many model families and vastly differing hyperparameters for a given base model 
because both may give us quite independent models.

\paragraph{}
This underscores the need for a smart and lean choice of the candidate models or hyperparameter 
tuples, respectively. Exclusion criteria for candidate models may be theoretical considerations
and experience by both us and others. Concerning experience, this means we will not move away too 
far from default hyperparameters and we will develop heuristics in chapter \ref{chap:results} as 
move from data set to the next.

\subsection{Testing}

We calculate the predictions $m^*(X_\text{test}) = \hat{y}_\text{test}$ of the best validated model 
$m^*$ on the test cohort and estimate its performance on independent data via 

\begin{align}
    \text{err}(y_\text{test}, \hat{y}_\text{test}).
\end{align}

\subsection{Choice of the error function}\label{sec:error-function}

Taking into account the calculations in section \ref{sec:intro-mmml},
we want to optimize the precision of the high-risk group, which must comprise at least 15\% of the 
samples in our data. Since an error should be the lower, the better the model is, we choose 
$\text{err}$ to be the minimum of the negative precisions with a prevalence of at least 17\%. 

Usually, we need 
to take a minimum over \textit{several} precisions because most models do not output the final 
classification.
Instead they return a continuous score where a higher score means a higher probability of being
in the positive class, in our case being high-risk, and this continuous score needs thresholding
via $\hat{y}_i > t$ for some $t \in \RR$.
We notice that as we increase $t$ and thereby decrease the prevalence, the obtained 
adjacent precision values more and more depend on one another; for high $t$ and low 
prevalences, fluctuations of up to 50 percentage points would be possible, but requiring a 
prevalence of at least 17\% caps such fluctuations at 
\begin{align}
    \frac{1}{\ceil*{\num{0.17} \cdot n_\text{test}}}
\end{align}
for $n_\text{test}$ samples in the test cohort. This gives us an error function that is tailored 
for our problem, sufficiently far-sighted and robust.

As indicated by the notation $\text{err}(y_\text{train}, \hat{y}_\text{train})$ and 
$\text{err}(y_\text{test}, \hat{y}_\text{test})$, we optimize $t$ on the true outcomes 
of the train cohort when validating and we optimize it again on the true outcomes of the test cohort 
when testing. Strictly speaking, $t$ is a hyperparameter, which we optimize on the 
test cohort during testing; this sounds delicate. The results however will show that the optimal 
choice for $t$ (or at least an 
almost optimal choice) on both train and test cohort correponds to a prevalence of 
slightly above 17\% such that one can agree on the following when it comes to the MMML-Predict data: 
choose $t$ as the 17\%-quantile of the coninuous model output on the test cohort.

\subsection{Inter-technical variability} 
One could instead suggest to already optimize the output threshold on the cross-validated 
predictions. While this promises a stricter train-test regime at first glance, fixing the output 
threshold once and for all neglects inter-technical variability. An always-present problem in 
Bioinformatics, inter-technical variability refers to the fact that one and the same sample 
measured on different platforms or in different labs, in short: with different protocols, may 
lead to different values for the same feature. If we measure the same $p$ features of the $n$ 
patients in $\mathbf{x} \in \RR^{n \times p}$ again with another protocol, we end up with another predictor 
$Z \in \RR^{n \times p}$.
For gene-expression levels,
we can often well model the discrepancies with two indepedent biases, sample-wise effects $\theta_i$
and feature-wise effects $\omega_j$, leading to
\begin{align}
    \Delta_{ij} = Z_{ij} - \mathbf{x}_{ij} = \theta_i + \omega_j + \epsilon_{ij}
\end{align}
for residues $r_{ij}$. Assuming accurate modeling, i.e. small residues, we can well approximate
\begin{align}
    Z_{ij} \approx \tilde{Z}_{ij} = \mathbf{x}_{ij} + \theta_i + \omega_j.
\end{align}
We can now apply an ordinary linear -- or Gaussian -- model to $Z$ obtaining
\begin{align}
    \beta_0 + \sum_{j=1}^p \beta_j Z_{ij} &\approx \beta_0 + \sum_{j=1}^p \beta_j \tilde{Z}_{ij} \\
    &= \beta_0 + \sum_{j=1}^p \beta_j \mathbf{x}_{ij} + \sum_{j=1}^p \beta_j \theta_i + \sum_{j=1}^p \beta_j \omega_j.
\end{align}
The second summand is zero under the zero-sum constraint $\sum_{j = 1}^p \beta_j = 0$ and the third 
summand is constant across all samples and can be absorbed by the intercept \cite{transplatform17}. 

Under these (not very 
strict) conditions, going from one protocol to another just leads to a constant shift of the model 
output; even for generalized linear models, i.e. a Gauss model composed with some monotic link function,
inter-technical variability does not change the ordering of the samples. All one needs to do to 
obtain a final classification is calibrating the threshold for the output scores. However, this 
also demonstrates that while generalized linear models can cope with inter-technical variability 
pretty well, there is no point in using the same output threshold across all protocols.

\section{Candidate models}\label{sec:candidate-models}

\subsection{Model-agnostic hyperparameters}\label{subsec:model-agnostic}

Model-agnostic hyperparameters are those hyperparameters we can set and tune for every model. They 
concern the predictor matrix $\mathbf{x} \in \RR^{n \times p}$ and the response vector $y \in 
(\RR \times \{ 0, 1 \})^n \cup \{ 0, 1 \}^n$. We can provide the response in two formats. In the 
first case, the response has two entries per sample: the first one is the time to event, in our 
case progression-free survival, and the second one indicates if the event has occured (1) or 
if the patient was censored before it could occur (0). In the second case, the response is a binary 
vector with 1 indicating the positive class (high-risk DLBCL)
and 0 indicating the negative class (low-risk DLBCL). Note that in the latter case we need 
to discard samples censored before two years, but this is typically only a small proportion of the 
samples in the data set.

\subsubsection{A-priori feature selection}

Of course, not every feature in the data set should be part of the predictor matrix $\mathbf{x}$. Including 
features measured to toward the end of a patinent's therapy or even after it, is cheating and we 
should definitely exclude them. Excluding any of the remaining features is hard to justify: If we 
know for a feature that it alone is associated with the response, we of course include it in $\mathbf{x}$.
On the other hand, if a feature alone shows no link to the response, the right model might still be 
able to leverage it in combination with other features; but at this step, we do not know if such 
a model exists and if it is among our candiate models. The only remaining option for handpicking 
features a priori is to brute-force the problem and try out all $2^p$ combinations of features.
Even if we trust on regularization to do its job for the high-dimensional part (typically several 
hundreds or even thousands of gene expression levels), just \num{10} remaining fetures would still
leave us with $2^{10} = 1024$ combinations to try out -- per model class. While this may be 
computationally feasible, validation would be a statistical fiasco as laid out in subsection 
\ref{subsec:train-val}. Therefore, we decided to reduce this quickly exploding problem to only 
a couple of decisions to 
make and have all the other feature selection happen during model fitting. 

\paragraph{Gene-expression data}
The first and most important decision to make is: do we want to include the high-dimensional
gene-expression part of the data at all? If we do not, we suddenly have $p \ll n$ instead of 
$p \gg n$ and can use way more complex models. 

\paragraph{Features in different formats}
Second, we need to make a couple of more decisions regarding the format of some features.
\begin{itemize}
    \item Concerning the IPI, one can include the five IPI features in their original, continuous 
        form; discretized as by the IPI inventors; the IPI score as a single continuous feature; 
        or the so-called IPI group, a partioning of the IPI score into three groups. Similar widely 
        accepted thresholds may exist for other clinical and genetic features.
    \item For already existing models, we can include their continuous output or also threshold 
        it. E.g., cell-of-origin signatures can be -- and most of the time in data sets already 
        have been -- thresholded into ABC, GCB and unclassified subtypes. The above-mentioned IPI 
        group thresholds the IPI model. 
\end{itemize}

Imagine, a data set provides the five IPI features in continuous format and the continuous output 
of some gene-expression signature for which 
its inventors also provide a preferred threshold. If we want to decide for exactly one format 
per feature, this leaves us with $5 \cdot 2$ possibilities for this simple case. If we want try out 
every combination of non-empty subsets, the number of possibilities jumps to $2^5 \cdot 2^2 = 
128$. 

Our solution here again is to be generous and include all formats of a feature in $\mathbf{x}$ a 
certain model may benefit from. This sentence usually 
simplifies to do: include all widely used formats of a feature in $\mathbf{x}$. E.g., generalized linear models
cannot threshold continuous features, so a feature in its thresholded format may give a rough 
contribution to the prediction while the same feature in its continuous form may further refine it.
Moreover, even decision trees can benefit from a feature additionally provided in its thresholded 
format if this threshold has been inferred on a much bigger data set since the decision tree 
may not be able to find the threshold itself.

\subsubsection{Imputation}

If values in $\mathbf{x}$ are missing, i.e. written as \texttt{NA}, actions we can take fall into two 
categories: we discard the part of data affected by missing values or we replace the missing values 
with some realistic estimate.

\paragraph{Discarding part of the data}

In a first step, we discard a feature if it is not available for a large proportion of the samples.
This enables reasonable computing.

\paragraph{Imputing}

We then mean-impute the remaining missing values. At this point, we model a categorical feature 
with $c$ categories as $c-1$ binary dummy features; if for a sample this categorical feature is not 
available, all $c-1$ dummy features will be \texttt{NA}. For every column in $\mathbf{x}$, we calculate the 
mean of the available features in $\mathbf{x}$ and replace the \texttt{NA} values with it. 
For a missing categorical feature, every dummy feature therefore holds the marginal probability 
that the feature is in the $k$-th category. This is efficient, transparent and easily applicable to 
new data.

\subsubsection{Adding combinations of discrete features to the predictor}

We add all combinations of at most $n_\text{combi}$ categorical features that are positive in a 
share of at least $s_\text{min}$ patients to $\mathbf{x}$; e.g., we add a column ``female \& ABC-type 
tumor'' if at least \num{5}\% of patients have this property. We always choose $s_\text{min}
= 5\%$. We set $n_\text{combi} = 1$ for models that facilitate interactions between features 
themselves, otherwise we set $n_\text{combi} = 3$.

\subsubsection{Tuning the definion of ``high-risk''}

Defining a patient as high-risk if and only if the progression-free survival is less than two years
is clinically accepted, yet quite arbitrary decision. It may be not be the time cutoff that 
separates patients' survival 
best from a biological point of view. Therefore we can provide the fitting 
algorithm a modified response $y$, governed by the time cutoff $T > 0$.
\begin{itemize}
    \item For binary response $y \in \{0, 1\}^n$, we set $y_i = 1$ if the progression-free survival 
        of patient $i$ is less than $T$ and $y_i = 0$ otherwise.
    \item For response $y \in (\RR \times \{0, 1\})^n$ with time to event and censoring, we censor 
        all samples with time to event exceeding $T$ at $T$.
\end{itemize}

\subsection{Core models}\label{subsec:core-models}

All models trained, validated and tested in this thesis at the core consist of ordinary linear, 
logistic and Cox proportional-hazards models with additional properties as well as random forests.
In this chapter, we want to introduce the design of these models and the hyperparameters governing 
their fitting process.

Formally, we deal with a probability space $(\Omega, \mathcal{A}, P)$ that we do not and cannot 
specify any further because we only get in touch with two random variables 
\begin{align}
    X = (X_1, \ldots, X_n): \Omega \to \RR^p \text{ and } Y: \Omega \to \RR,
\end{align}
the predictor and the response, respectively. More 
precisely, we observe independent samples $(x_i, y_i) \in \RR^{p+1}, i = 1, \ldots, n$, 
distributed according to $(X, Y)$; $x_i$ is the $i$-th row of the predictor matrix $\mathbf{x}$ and
$y_i$ is the $i$-th entry of the binary response vector $y$. MMML-Predict is all about getting 
a hand on the conditional distribution $P(Y \cond X = x)$ for all $x \in \RR^p$.

\subsubsection{Generalized linear models}\label{subsubsec:glm}

Both ordinary linar models and logistic regression models are generalized linear models (GLMs). In 
a GLM, $Y$ follows an exponential-family distribution and there is an invertible link function 
$g: \RR \to \RR$ and parameters $(\beta_0, \beta) \in \RR^{p+1}$ such that
\begin{align}
    g(E(Y \cond X = x)) = \beta_0 + \sum_{j=1}^p \beta_j x_j \quad \text{for all } x \in \RR^p.
\end{align}
The linear predictor $\beta_0 + \sum_{j=1}^p \beta_j x_j$ predicts the expected
value of $Y$ given $X = x$, $E(Y \cond X = x) = \mu$, via a link function. 

\begin{itemize}
    \item For the ordinary linear model, the link function is the identity and $Y \cond X = x$ 
        follows a normal distribution with mean $\mu$ and variance $\text{Var}(Y)$.
    \item For the logistic model, $g: \mu \mapsto \ln\left( \frac{\mu}{1 - \mu} \right)$ and 
        $Y \cond X = x$ follows a Bernoulli distribution with parameter $\mu$.
\end{itemize}

In both cases, the distribution of $Y \cond X = x$ is fully determined by the mean $\mu$ and hence 
by $(\beta_0, \beta)$. Hence, the log-likelihood for 

\subsection{Nested models}\label{subsec:nested-models}

\section{Software}
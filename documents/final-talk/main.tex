\documentclass[10pt, aspectratio=169]{beamer}
% hi
\usetheme{metropolis}
\metroset{block=fill}
\usepackage{appendixnumberbeamer}

\usepackage{booktabs}
\usepackage[scale=2]{ccicons}

\usepackage{siunitx}

\usepackage{xspace}
\newcommand{\themename}{\textbf{\textsc{metropolis}}\xspace}

\usepackage{amsmath}
\def\RR{\mathbb{R}}
\def\Pr{\mathrm{Pr}}
\usepackage{algorithm}
\usepackage{algpseudocode}

\usepackage{natbib}
\bibliographystyle{plainnat}

\usepackage{graphicx}
\usepackage{xcolor}
\usepackage{hyperref}

\definecolor{mDarkTeal}{HTML}{23373b}
\definecolor{mLightBrown}{HTML}{EB811B}
\definecolor{mDarkBrown}{HTML}{B85002}
\definecolor{mLightGreen}{HTML}{14B03D}
\setbeamertemplate{itemize/enumerate subbody begin}{\normalsize}

\title{Beating the International Prognostic Index for high-risk DLBCL patients}
\subtitle{Master thesis final report}
\date{July 9th, 2024}
\author{Lukas Geßl}
\institute{Chair of Statistical Bioinformatics, Regensburg University}

\begin{document}

\maketitle

\section{Recap: The goal of this thesis}

\begin{frame}{}
  \begin{itemize}
    \item The MMML-Predict project wants to develop a cost-efficient classifier 
    that filters DLBCL patients with progression-free survival $\leq 2$ years more reliably 
    than the International Prognostic Index for non-Hodgkin's lymphoma (IPI).
    \item The IPI is a simple risk score (0-5) based on five clinical features one can
    measure cheaply and without batch effects. The cohorts IPI $\geq i, i = 0, 1, \ldots, 5$
    lack precision ($< 50 \%$) or are too small to be clinically relevant
    (prevalence $< 10 \%$) (at least in the DSNHL trials).
    \item Our classfier should label at least 15\% of patients as high-risk with a precision
    of at least $\max(50\%, \text{precision of IPI} \geq 4)$.
    \item Unlike the IPI, the new classifier can incorporate the whole range of 
    modern features (like transcriptomic, genetic, clinical data, already-existent 
    signatures) measured at diagnosis and even dynamic features measured during 
    the treatment.
  \end{itemize}
\end{frame}

\section{How to find and sell the best model}
\begin{frame}{A two-step approach}
  \begin{description}
    \item[Validation] Of those models we have trained, we want to find and choose
       the model that performs best on new data to the best of \alert{our} knowledge.
    \item[Testing] We need to demonstrate the performance of the chosen model to 
      \alert{outside} people on new, independent data.
  \end{description}

  To this end, we split the data into a train cohort (for validation) and test 
  cohort (for testing); we even do so multiple times so particulary lucky or 
  unlucky splits will not mislead us. 
\end{frame}

\begin{frame}{Validation}
  We start with a set of tuples of hyperparameters $H$, where every $h \in H$ 
  defines a model up to its parameters (the latter is the job of the fitting 
  algorithm).

  For every hyperparameter tuple $h \in H$ and every split $s \in S$ into train 
  cohort $(X_{\text{train}, s}, y_{\text{train}, s})$ and test cohort 
  $(X_{\text{test}, s}, y_{\text{test}, s})$,

  \begin{enumerate}
    \item fit the model to train cohort $(X_{\text{train}, s}, y_{\text{train}, s})$
    in a cross-validation. This yields a tuple $(y_{\text{train}, s}, \hat{y}_s)$
    of true and cross-validated predicions. We describe this with a mapping 
    $\text{cv}: (h, s) \mapsto (y_{\text{train}, s}, \hat{y}_s)$.
    \item Then calculate the model's cross-validation error via $\text{err}(
      y_{\text{train}, s}, \hat{y}_s)$.
  \end{enumerate}

  The model's final cross-validation error is
  \begin{align*}
    \text{Err}_{\text{cv}}: H \to \RR: h \mapsto \frac{1}{|S|} \sum_{s \in S} 
      (\text{err} \circ \text{cv})(h, s).
  \end{align*}
\end{frame}

\begin{frame}{Validation and testing}
  We select the model with hyperparameter tuple
  \begin{align*}
    h^* = \arg \min_{h \in H} \text{Err}_{\text{cv}}(h)
  \end{align*}
  and call it $m^*$.

  As for testing, we obtain $m^*$'s predictions $\hat{y}_s$ on the test cohort 
  $(X_{\text{test}, s}, y_{\text{test}, s})$ for all $s \in S$ and estimate 
  its performance on independent data via
  \begin{align*}
    \frac{1}{|S|} \sum_{s \in S} \text{err}(y_{\text{test}, s}, \hat{y}_s).
  \end{align*}
\end{frame}

\begin{frame}{Validation and testing}
  For our problem, we choose $\text{err}(y, \hat{y})$ as the maximum of the precisions
  with a prevalence of at least 15\% (model output usually needs thresholding).

  Strictly speaking, the threshold for the model output is another hyperparameter, 
  but it is a platform-dependent one, so we will not fix it. On a new data set,
  one might then take the 15\% quantile of the model output as the threshold.
\end{frame}

\section{Let's talk about $H$: candidate models}

\begin{frame}{Model-agnostic hyperparameters ...}
  ... apply for every model. In our case, they concern the predictor matrix 
  $X \in \RR^{n \times p}$ and the response vector $y \in \{ 0, 1 \}^n \cup 
  (\RR \times \{0, 1 \})^n$ (binarized response and Cox response with time-to-event 
  and censoring status).

  \begin{itemize}
    \item We add all combinations of at most $n_{\text{combi}}$ discrete features 
      that are positive in a share of at least $s_{\text{min}}$ patients to $X$; e.g. 
      we add a column "female and ABC-type tumor" if at least 5\% of patients 
      have this property.
    \item For $T > 0$, we provide the fitting algorithm a modified reponse $y$, namely
      \begin{itemize}
        \item for the binary response, we set $y_i = 1$ if the patient's progression-free 
          survival is $< T$ years, $y_i = 0$ otherwise,
        \item for the Cox response, we censor all samples with time to event 
          exceeding $T$ at $T$.
      \end{itemize}
    \item A-priori feature selection: which features do we include in $X$ in the 
      first place?
  \end{itemize}
\end{frame}

\begin{frame}{The most model-specific hyperparameter: model class}
  At the core, our models consist of 
  \begin{itemize}
    \item Cox proportional-hazards,
    \item logistic regression and 
    \item ordinary linear regression
  \end{itemize}
  models with a loss function optionally equipped with
  \begin{itemize}
    \item $\ell_1$ or $\ell_2$ regularization,
    \item the zero-sum constraint which promises to make the model transferable
      between different data sets.
  \end{itemize}

  Moreover, we deploy random forests.
\end{frame}

\begin{frame}{Nested models}
  Given some "early" models $f_i: \RR^p \to \RR, i = 1, \ldots, m$, we can nest 
  them into another, "late" model $f: \RR^m \to \RR$ and get a new model 
  $f \circ (f_1, \ldots, f_m)$.

  \begin{itemize}
    \item Often, the early models have been trained on another data set, so we 
      observe their output as features in our data set (like ABC/GCB).
    \item If we need to fit some of the early models to our data, getting good 
      models and reliable cross-validated predictions gets a bit more tricky. 
  \end{itemize}
  On the next slide, we will deal with a special case of the latter case, where 
  exactly one early model, say $f_1$, needs to be learned from our data.

  Typically, we train the early (generalized linear) model on the high-dimensional 
  part of the data (like gene expression) and use its output together with the 
  remaining features as input for the (possibly more sophisticated) late model.
\end{frame}

\begin{frame}{}
  \begin{algorithm}[H]
    \caption{Nested pseudo cross validation}
    \begin{algorithmic}[1]
      \State \textbf{Input:} Predictor matrix $X$, response $y$, hyperparameter tuple $h = (h_1, h_2)$
      \State Fit $f_1$ to $X$ and obtain $\hat{f}_1: \RR^p \to \RR$
      \State \textbf{Output:} $f: \RR^p \to \RR$
    \end{algorithmic}
  \end{algorithm}
\end{frame}

\section{How this plays out on real data}

\section{Take aways and outlook}

\begin{frame}[standout]
  Thank you for your attention! \par Questions?
\end{frame}

\appendix

\begin{frame}[allowframebreaks]{References}
  \bibliography{lit}
\end{frame}

\end{document}